\section{29/04/2024}

\subsection{Baer's theorem}

\begin{theorem}[Baer]
	\index{Baer's theorem}
	\label{thm:Baer}
	Let $G$ be a finite group and $H$ be a subgroup of $G$. Then $H\subseteq
	F(G)$ if and only if $\langle H,xHx^{-1}\rangle$ is nilpotent for all 
	$x\in G$.
\end{theorem}

\begin{proof}
	If $H\subseteq F(G)$, then $xHx^{-1}\subseteq F(G)$ for all $x\in G$, since
	$F(G)$ is normal in $G$. Thus $\langle H,xHx^{-1}\rangle$ is nilpotent, as it
	is a subgroup of $F(G)$.

	Conversely, assume that $\langle H,xHx^{-1}\rangle$ is nilpotent for all  $x\in G$. Since $H\subseteq \langle H,xHx^{-1}\rangle$, $H$ is nilpotent. By
	Theorem~\ref{thm:F(G)subnormalidad}, it is enough to see that $H$ is subnormal
	in $G$. Suppose that $H$ is not subnormal in
	$G$. In particular, $H$ is properly contained in some proper subgroup $K$ of $G$. Since $\langle
	H,kHk^{-1}\rangle$ is nilpotent for all $k\in K$, $H$ is subnormal in $K$ if we assume that $G$ is a minimal counterexample. 
	By Theorem~\ref{thm:zipper}, there exists a unique
	maximal subgroup $M$ of $G$ containing $H$. 
    
    Assume first that $G=\langle H,xHx^{-1}\rangle$ for some $x\in G$. Since $G$
    is nilpotent, $H$ subnormal in $G$ by Theorem~\ref{thm:subnormal}, a
    contradiction. 

    Assume now that $\langle H,xHx^{-1}\rangle\ne G$ for all $x\in G$. For each 
	$x\in G$, there exists a maximal subgroup containing $\langle
	H,xHx^{-1}\rangle$. Since $H\subseteq \langle H,xHx^{-1}\rangle$ and $H$
	is contained in a unique maximal subgroup, we conclude that $\langle
	H,xHx^{-1}\rangle\subseteq M$ for all $x\in G$. In particular, the normal closure 
	$H^G$ of $H$ is properly contained in $G$. By the inductive hypothesis, 
	$H$ is subnormal in $H^G$ and $H^G$ is normal in $G$, we conclude that 
	$H$ is subnormal in $G$, a contradiction. 
\end{proof}

\subsection{Zenkov's theorem}

\begin{theorem}[Zenkov]
    \index{Zenkov's!theorem}
    \label{thm:Zenkov}
    Let $G$ be a finite group and $A$ and $B$ be abelian subgroups of $G$. Let
    $M\in\{A\cap gBg^{-1}:g\in G\}$ such that no $A\cap gBg^{-1}$ is properly
    contained in $M$. Then $M\subseteq F(G)$.
\end{theorem}

\begin{proof}
    Suppose the result is not true and let $G$ be a minimal counterexample. 
	Without loss of generality, we may assume that $M=A\cap B$. 

	Assume first that $G=\langle A,gBg^{-1}\rangle$ for some $g\in G$. Since $A$
	and $B$ are both abelian, 
 \[
 A\cap gBg^{-1}\subseteq Z(G)
 \]
 and hence 
	\[
		A\cap gBg^{-1}=g^{-1}(A\cap gBg^{-1})g\subseteq A\cap B=M.
	\]
	By the minimality of $G$, 
    \[
    M=A\cap gBg^{-1}\subseteq Z(G)\subseteq F(G)
    \]
	by Corollary~\ref{cor:Z(G)subsetF(G)}.

	Assume now that $G\ne \langle A,gBg^{-1}\rangle$ for all $g\in G$.
	Let $g\in G$, $H=\langle A,gBg^{-1}\rangle\ne G$ and $C=B\cap H$.
	Since $A\subseteq H$, we obtain that 
 	$M=A\cap B=A\cap C$ and 
	$A\cap hCh^{-1}=A\cap hBh^{-1}$
	for all $h\in H$. This implies that no 
	$A\cap hCh^{-1}$ is properly contained in $A\cap C$. 
    By the inductive hypothesis on $H$, 
 	\[
		M=A\cap B=A\cap C\subseteq F(H).
	\]

    We now prove that every Sylow $p$-subgroup $P$ of $M$ is contained in $F(G)$. Moreover, 
    $M\subseteq F(G)$.
	If $P\in\Syl_p(M)$, then $P\subseteq M\subseteq F(H)$. Since $O_p(H)$ is 
    the only Sylow $p$-subgroup of $F(H)$, $P\subseteq O_p(H)$. Since 
	$P\subseteq M\subseteq B$, 
	\[
	gPg^{-1}\subseteq gBg^{-1}\subseteq H.
	\]
	Thus $O_p(H)(gPg^{-1})$ is a $p$-subgroup of $H$ containing 
	$\langle P,gPg^{-1}\rangle$. Hence $\langle P,gPg^{-1}\rangle$
	is nilpotent for all $g\in G$, since it is a $p$-group. By Baer's theorem~\ref{thm:Baer}, 
    $P\subseteq F(G)$ for all Sylow $p$-subgroup $P$ of $M$. 
\end{proof}

\begin{corollary}
	\label{cor:Zenkov}
	Let $G$ be a non-trivial finite group and $A$ be an abelian subgroup of $G$ such that 
 	$|A|\geq(G:A)$. Then $A\cap F(G)\ne\{1\}$.
\end{corollary}

\begin{proof}
	Let $g\in G$. We may assume that $G\ne A$. Then $(gAg^{-1})A\ne G$ by Lemma~\ref{lem:H=G}. Since 
	$|gAg^{-1}||A|=|A|^2\geq |A|(G:A)=|G|$, 
	\[
		|G|>|gAg^{-1}A|
		=\frac{|A||gAg^{-1}|}{|A\cap gAg^{-1}|}
		\geq \frac{|G|}{|A\cap gAg^{-1}|}.
	\]
	Hence $A\cap gAg^{-1}\ne\{1\}$ for all $g\in G$. By Zenkov's theorem, 
    the trivial subgroup is not contained in 
    the set $\{A\cap gAg^{-1}:g\in G\}$. A minimal subset 
    of this set then belongs to $F(G)$. 
\end{proof}

\begin{corollary}
	Let $G=NA$ be a finite group, where $N$ is a normal subgroup of $G$, $A$ is an abelian subgroup of $G$ and 
	$C_A(N)=\{1\}$. If $F(N)=\{1\}$, then $|A|<|N|$. 
\end{corollary}

\begin{proof}
	Since $N$ is normal in $G$, 
	\[
    N\cap F(G)=F(N)=\{1\}
    \]
    by Corollary~\ref{cor:McapF(G)}. Thus $[N,F(G)]=\{1\}$, since 
	both $N$ and $F(G)$ are normal in $G$. Since 
	\[
	|A|\geq |N|\geq \frac{|N|}{|N\cap A|}=(NA:A)=(G:A),
	\]
	$A\cap F(G)\ne\{1\}$ by Corollary~\ref{cor:Zenkov}. If $1\ne a\in
	A\cap F(G)$, then $a\in C_A(N)=\{1\}$, a contradiction. 
\end{proof}

\subsection{Brodkey's theorem}

\begin{theorem}[Brodkey]
	\index{Brodkey's!theorem}
	\label{thm:Brodkey}
	Let $G$ be a finite group such that there exists an abelian $P\in\Syl_p(G)$. Then
    there exist $S,T\in\Syl_p(G)$ such that $S\cap T=O_p(G)$.
\end{theorem}

\begin{proof}
	By applying Zenkov's theorem (Theorem~\ref{thm:Zenkov}) with $A=B=P$, 
	\[
    P\cap gPg^{-1}\subseteq F(G)
    \]
    for some $g\in G$. Since $O_p(G)$ is the only Sylow $p$-subgroup of 
	$F(G)$, $P\cap gPg^{-1}\subseteq O_p(G)$.
	Hence $P\cap gPg^{-1}=O_p(G)$, since $O_p(G)$ is contained in every Sylow $p$-subgroup 
	of $G$. 
\end{proof}

\begin{corollary}
	\label{corollary:GP2}
	Let $G$ be a finite group. If there exists an abelian $P\in \Syl_p(G)$, 
	\[
	(G:O_p(G))\leq (G:P)^2. 
	\]
\end{corollary}

\begin{proof}
	By Brodkey's theorem, there exist $S,T\in\Syl_p(G)$
	such that $S\cap T=O_p(G)$. Then 
	\[
		|G|\geq |ST|=\frac{|S||T|}{|S\cap T|}=\frac{|P|^2}{|O_p(G)|},
	\]
	which implies the claim. 
\end{proof}

\begin{corollary}
	Let $G$ be a finite group. If there exists an abelian $P\in\Syl_p(G)$ such that 
	$|P|>\sqrt{|G|}$, then $O_p(G)\ne\{1\}$.
\end{corollary}

\begin{proof}
	Since $(G:P)^2<|G|$, the previous corollary implies that 
	$O_p(G)\ne\{1\}$.
\end{proof}

% \begin{exercise}
% 	\label{xca:G/Z(G)}
% 	Sea $G$ un grupo y sea Sea $K\subseteq Z(G)$. Demuestre que si $G/K$ es
% 	cíclico entonces $G$ es abeliano.
% \end{exercise}

% \begin{sol}{xca:G/Z(G)}
% 	Sean $g,h\in G$ y sea $\pi\colon G\to G/K$ el morfismo canónico. Como $G/K$
% 	es cíclico, existe $x\in G$ tal que $G/K=\langle xK\rangle$. Sean $k,l$ tales que 
% 	$\pi(g)=x^k$, $\pi(h)=x^l$. Entonces existen $z_1,z_2\in K$ tales que 
% 	$g=x^kz_1$, $h=x^lz_2$. Luego $[g,h]=[x^k,x^l]=1$. 
% \end{svgraybox}


\subsection{Lucchini's theorem}

\begin{theorem}[Lucchini]
	\index{Lucchini's!theorem}
	\label{thm:Lucchini}
	Let $G$ be a finite group and $A$ be a proper cyclic subgroup of $G$. If 
	$K=\Core_G(A)$, then $(A:K)<(G:A)$.
\end{theorem}

\begin{proof}
	We proceed by induction on $|G|$. Let $\pi\colon G\to G/K$ be the canonical map. Note that $\Core_{G/K}\pi(A)$ is trivial. 

	Assume first that $K\ne\{1\}$. Since $\pi(A)$ is a proper cyclic subgroup of 
	$G/K$ and $K\subseteq A$, the inductive hypothesis implies that 
	\[
		(A:K)=|\pi(A)|=(\pi(A):\pi(K))<(\pi(G):\pi(A))=\frac{(G:K)}{(A:K)}=(G:A).
	\]

	Assume now that $K=\{1\}$. We want to prove that $|A|<(G:A)$. Suppose that 
	$|A|\geq (G:A)$. Since $A\ne G$, $A\cap F(G)\ne\{1\}$ by Corollary~\ref{cor:Zenkov}. In particular, $F(G)\ne\{1\}$. Let $E$ be a minimal normal subgroup of 
	such that $E\subseteq F(G)$. By Theorem~\ref{thm:Hirsch}, $E\cap Z(F(G))\ne\{1\}$.  Since 
	$E\cap Z(F(G))$ is normal in $G$ and $E$ is minimal, $E\cap Z(F(G))=E$, that is 
	$E\subseteq Z(F(G))$. In particular, $E$ is abelian. By the minimality of 
	$E$, there is a prime number $p$ such that $x^p=1$ for all $x\in E$. 

	\begin{claim}
		$A\cap F(G)$ is a normal subgroup of $EA$.
	\end{claim}

	Since $E$ is normal in $G$, $EA$ is a subgroup of $G$. Since $A\cap
	F(G)\subseteq A$, $A\cap F(G)$ is a subgroup of $EA$.  Since $F(G)$ is 
	normal in $G$, $a(A\cap F(G))a^{-1}=A\cap F(G)$ for all $a\in A$. Moreover, 
    $E\subseteq Z(F(G))$ and $A\cap F(G)\subseteq F(G)$ imply that 
	$x(A\cap F(G))x^{-1}=A\cap F(G)$ for all $x\in E$. 

	\begin{claim}
		$EA\ne G$.
	\end{claim}

	If $G=EA$, then, since $A\cap F(G)$ is a normal subgroup of $G$
	contained in $A$, we conclude that $\{1\}\ne A\cap F(G)\subseteq K=1$, a 
	contradiction. 
%como $F(G)$ es normal en $G$, 
%	\[
%	A\cap F(G)=g(A\cap F(G))g^{-1}=gAg^{-1}\cap F(G)\subseteq gAg^{-1}
%	\]
	%para todo $g\in G$. 
%    Luego $1\ne A\cap F(G)\subseteq K$, una contradicción pues $K=1$.

	\medskip
	Let $p\colon G\to G/E$ the canonical map. By the correspondence theorem,
	there exists a normal subgroup $M$ of $G$ such that $E\subseteq M$ and 
	$p(M)=\Core_{G/E}(p(A))$. Since $EA\ne G$, $p(A)$ is a proper cyclic subgroup 
	of $p(G)$. Since $p(A)\simeq A/A\cap E\simeq EA/E$ and $p(M)\simeq
	M/E$, the inductive hypothesis implies that 
	$(EA:M)<(G:EA)$, as 
	\[
	\frac{|EA/E|}{|M/E|}
	=(p(A):p(M))
	<(p(G):p(A))
	=\frac{|G/E|}{|EA/E|}.
	\]

	\begin{claim}
		$MA=EA$. 
	\end{claim}

	Since $E\subseteq M$, $EA\subseteq MA$. Conversely, if $m\in M$, 
	then, since $p(m)\in\Core_{G/E}(p(A))$, we obtain that 
	$p(m)\in p(A)$. Thus $m\in EA$. 

	\medskip
	Let $B=A\cap M$. Since $(AE:M)<(G:EA)$, 
	\[
	(A:B)=|A/A\cap M|=|AM/M|=(EA:M).
	\]
	By the inductive hypothesis, 
	\begin{equation}
		\label{eq:(M:B)leq|B|}
	\begin{aligned}
		(M:B)&=(M:A\cap M)=(MA:A)\\
		&=(EA:A)
		=\frac{(G:A)}{(G:EA)}
		<\frac{(G:A)}{(AE:M)}
		=\frac{(G:A)}{(A:B)}\leq |B|, 
	\end{aligned}
	\end{equation}
	as $|A|\geq (G:A)$. 

	\begin{claim}
		$M=EB$.
	\end{claim}

	Since $E\cup B\subseteq M$, $EB\subseteq M$. Conversely, if 
	$m\in M$, then $m=ea$ for some $e\in E$ and $a\in A$. Since $e^{-1}m=a\in
	A\cap M=B$ (because $E\subseteq M$), $m\in EB$.

	\begin{claim}
		$M$ is not abelian. 
	\end{claim}

	Suppose that $M$ is abelian. The map $f\colon M\to M$, $m\mapsto
	m^p$, is a group homomorphism such that $E \subseteq\ker f$. Since $M=EB$,
	$f(M)\subseteq f(B)\subseteq B\subseteq A$. Since $M$ is normal in $G$,
	$f(M)$ is normal in $G$. Thus $f(M)=\{1\}$, as $K=\Core_G(A)=\{1\}$ is the largest normal subgroup of $G$ contained in $A$. In particular, since $B$ is normal in $M=EB$, $M/B$ is a $p$-group. Since $B\subseteq M$,  $f(B)=\{1\}$. Moreover, since 
	$B\subseteq A$ is cyclic, $|B|\leq p$. By using~\eqref{eq:(M:B)leq|B|}, 
	$(M:B)<|B|\leq p$. This implies that $M=B\subseteq A$ and $M=E=1$ (because 
	$M$ is normal in $G$ and $\Core_G(A)=K=\{1\}$ is the largest normal subgroup of $G$ containing $A$), a contradiction. 
	
	\begin{claim}
		$Z(M)$  is cyclic. 
	\end{claim}

	Since $M$ is not abelian and $M/E=EB/E\simeq B/E\cap B$ is cyclic,
	$E\not\subseteq Z(M)$, that is $E\cap
	Z(M)\subsetneq E$. Thus  
	\begin{equation}
		\label{equation:EcapZ(M)}
		E\cap Z(M)=\{1\}
	\end{equation}
	by the minimality of $E$. Hence  
	\[
	Z(M)=Z(M)/Z(M)\cap E\simeq p(Z(M))\subseteq p(M)=\Core_{G/E}p(A)\subseteq p(A)
	\]
	and therefore $Z(M)$ is cyclic, since $p(A)$ is cylic. 

	\medskip
	Since $B\subseteq A$ is abelian and $(M:B)<|B|$, $B\cap F(M)\ne1$ by Corollary~\ref{cor:Zenkov}. Then $[E,F(M)]=1$ (because $E\subseteq
	Z(F(G))$ and $F(M)\subseteq F(G)$ by Corollary~\ref{cor:McapF(G)}).
	Hence $B\cap F(M)\subseteq Z(M)$, since $M=BE$, $[B\cap F(M),E]\subseteq
	[F(M),E]=1$ and $[B\cap F(M),B]=1$ as $B$ is abelian. Since 
	$Z(M)$ is cyclic, $B\cap F(M)$ is characteristic in $Z(M)$. Since 
	$Z(M)$ is normal in $G$, $\{1\}\ne B\cap F(M)$ is a normal subgroup of $G$
	contained in $A$, a contradiction. 
\end{proof}

\subsection{Horosevskii's theorem}

To conclude this section, we present a striking application of Lucchini's theorem.

\begin{corollary}[Horosevskii]
	\index{Horosevskii's!theorem}
	Let $G$ be a finite non-trivial group and $\sigma\in\Aut(G)$. Then 
	$|\sigma|<|G|$.
\end{corollary}

\begin{proof}
	Let $A=\langle\sigma\rangle$ act by automorphisms on $G$ and 
	 $\Gamma=G\rtimes A$. The group operation of $\Gamma$ is 
	\[
	(g,\sigma^k)(h,\sigma^l)=(g\sigma^k(h),\sigma^{k+l}).
	\]
	Identity $A$ with $\{1\}\times A$ and $G$ with $G\times\{1\}$.
    Let $K=\Core_\Gamma(A)\subseteq A$. 
    % \[
    % (1,\sigma^i)(g,1)=(g,1)(1,\sigma^i)\text{ for all $g\in G$}
    % \Longleftrightarrow 
    % (g,\sigma^i)=(\sigma^i(g),\sigma^i)\text{ for all $g\in G$}
    % \Longleftrightarrow 
    % i=0.
    % \]   
    Since both $K$ and $G$ are normal in $\Gamma$, 
    \[ 
    [K,G]\subseteq K\cap G\subseteq A\cap G=\{1\}.
    \]
    Thus  
    $K\subseteq A\cap C_\Gamma(G)=\{1\}$.
    By Lucchini's theorem, 
    $(A:K)<(\Gamma:A)$, that is
	\[
		|\sigma|=|A|=(A:K)<(\Gamma:A)=|G|.\qedhere
	\]
\end{proof}
