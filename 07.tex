\section{28/03/2024}

\subsection{The Schur--Zassenhaus theorem}

% Para leer este capítulo es conveniente haber entendido el capítulo \ref{derivaciones}, ya
% que demostraremos el teorema de Schur--Zassenhaus gracias al uso de algunos trucos que
% involucran derivaciones. También es necesario utilizar el subgrupo de Frattini, capítulo \ref{Frattini}. 
% Daremos una aplicación 
% del teorema de Schur--Zassenhaus a grupos súper-resolubles, estudiados en el capítulo \ref{super_resoluble}. 

Recall that a group $Q$ \textbf{acts by automorphisms} on a group $K$ if 
there exists a map $Q\times K\to K$, $(q,k)\mapsto q\cdot k$, 
such that 
\begin{enumerate}
    \item $1\cdot a=a$ for all $a\in K$, 
    \item $x\cdot (y\cdot a)=(xy)\cdot a$ for all $x,y\in Q$ and $a\in K$, 
    \item $x\cdot 1=1$ for all $x\in Q$, and 
    \item $x\cdot (ab)=(x\cdot a)(x\cdot b)$ for all $x\in Q$ and $a,b\in K$, 
\end{enumerate}
For example, if $K$ is a normal subgroup of $G$, 
then $G$ acts by automorphisms on $K$ by conjugation. 

\begin{definition}
\index{1-cocycle}
Let $Q$ and $K$ be groups, where $Q$ acts by automorphisms on $K$. 
A map 
$\varphi\colon Q\to K$ is said to be a \textbf{1-cocycle} if 
\[
	\varphi(xy)=\varphi(x)(x\cdot\varphi(y))
\]
for all $x,y\in Q$.  
\end{definition}

Let $Q$ and $K$ be groups, where $Q$ acts by automorphisms on $K$. 
The set of 1-cocycles $Q\to K$ will be denoted by 
\[
Z^1(Q,K)=\{\delta\colon Q\to K:\text{$\delta$ is a 1-cocycle}\}.
\]

\begin{example}
Let $Q$ be a group acting by automorphisms on $K$. 
The semidirect product $K\rtimes Q$ 
is a group $G$ that contains a normal subgroup isomorphic to $K$ 
and a subgroup isomorphic to such that 
$G=KQ$ and $K\cap Q=\{1\}$. Under the obvious identifications, 
$Q$ acts on $K$ by conjugation. For each $k\in K$, the map 
$Q\to K$, $x\mapsto [k,x]=kxk^{-1}x^{-1}$, is a 1-cocycle. 
\end{example}

\begin{exercise}
\label{xca:1cocycle}
Let $\varphi\colon Q\to K$ be a 1-cocycle. Prove the following statements:
\begin{enumerate}
	\item $\varphi(1)=1$.
	\item $\varphi(y^{-1})=(y^{-1}\cdot\phi(y))^{-1}=y^{-1}\cdot\phi(y)^{-1}$.
	\item The set $\ker\varphi=\{x\in Q:\varphi(x)=1\}$ is a subgroup of $Q$. 
\end{enumerate}
\end{exercise}

\begin{lemma}
\label{lem:1cocycle}
Let $G$ be a group with a normal subgroup $N$. 
If $\varphi\colon G\to N$ is a 1-cocycle (where $G$ acts on $N$ by conjugation)
with kernel 
\[
K=\ker\varphi=\{g\in G:\varphi(g)=1\}, 
\]
then 
$\varphi(x)=\varphi(y)$ if and only if $xK=yK$. In particular,
$(G:K)=|\varphi(G)|$. 
\end{lemma}

\begin{proof}
If $\varphi(x)=\varphi(y)$, then, since  
\[
\varphi(x^{-1}y)
=\varphi(x^{-1})(x^{-1}\cdot\varphi(y))
=\varphi(x^{-1})(x^{-1}\cdot\varphi(x))
=\varphi(x^{-1}x)=\varphi(1)
=1,
\]
we obtain that $xK=yK$. Conversely, if $x^{-1}y\in K$, then, since 
\[
1=\varphi(x^{-1}y)=\varphi(x^{-1})(x^{-1}\cdot \varphi(y)),
\]
we obtain that $\varphi(y)=x\cdot\varphi(x^{-1})^{-1}$. We conclude that 
$\varphi(x)=\varphi(y)$.

The second claim now is clear, as $\varphi$ is constant in each coclass of $K$ 
and takes $(G:K)$ different values. 
\end{proof}

\begin{lemma}
	\label{lem:d}
	Let $G$ be a finite group, $N$ be an abelian normal subgroup of $G$ and $S$, $T$ and $U$
    be transversals of $N$ in $G$. Let 
	\[
	d(S,T)=\prod st^{-1}\in N,
	\]
	where the product runs over all elements $s\in S$ and $t\in T$ such that 
	$sN=tN$. The following statements hold: 
	\begin{enumerate}
		\item $d(S,T)d(T,U)=d(S,U)$.
		\item $d(gS,gT)=gd(S,T)g^{-1}$ for all $g\in G$.
		\item $d(nS,S)=n^{(G:N)}$ for all $n\in N$.
	\end{enumerate}
\end{lemma}

\begin{proof}
	If $s\in S$, $t\in T$ and $u\in U$ are such that $sN=tN=uN$, then, since $N$ is 
	abelian and $(st^{-1})(tu^{-1})=su^{-1}$, we obtain that 
	\[
		d(S,T)d(T,U)=\prod (st^{-1})(tu^{-1})=\prod su^{-1}=d(S,U).
	\]

	Since $sN=tN$ if and only if $gsN=gtN$ for all $g\in G$, 
	\[
	g\left(\prod st^{-1}\right)g^{-1}=\prod gst^{-1}g^{-1}=\prod (gs)(gt)^{-1}=d(gS,gT).
	\]

	Finally, since $N$ is normal in $G$, $nsN=sN$ for all $n\in N$. Thus 
	\[
		d(nS,S)=\prod (ns)s^{-1}=n^{(G:N)}.\qedhere
	\]
\end{proof}

Recall that a subgroup $K$ of $G$ admits a \textbf{complement} $Q$ 
if $G$ factorizes as 
$G=KQ$ with $K\cap Q=\{1\}$. 
A typical example is the semidirect product $G=K\rtimes Q$, where $K$ is a normal subgroup of 
$G$ and $Q$ is a subgroup of $G$ such that $K\cap Q=\{1\}$. 

\begin{exercise}
\label{xca:complementos}
Let $Q$ act by automorphisms on $K$. Prove that there is a bijection 
between the set of complements of $K$ in $K\rtimes Q$ and the set 
$Z^1(Q,K)$.
\end{exercise}

% \begin{sol}{xca:complementos}
% 	El grupo $Q$ actúa en $K$ por conjugación, entonces $\delta\in\Der(Q,K)$ si
% 	y sólo si $\delta(xy)=\delta(x)x\delta(y)x^{-1}$, $x,y\in Q$. En este caso,
% 	las fórmulas del ejercicio anterior quedan así:
% 	$\delta(1)=1$, $\delta(x^{-1})=x^{-1}\delta(x)^{-1}x$.
	
% 	Sea $\mathcal{C}$ el conjunto de complementos de $K$ en $K\rtimes Q$.  Sea
% 	$C\in\mathcal{C}$. Si $x\in Q$, sabemos que 
% 	existen únicos $k\in K$ y $c\in C$ tales que $x=k^{-1}c$. Queda bien
% 	definida entonces la función $\delta_C\colon Q\to K$, $x\mapsto k$. Vale
% 	que $\delta(x)x=c\in C$. 
	
% 	Veamos que $\delta_C\in\Der(Q,K)$. Si $x,x_1\in Q$, escribimos $x=k^{-1}c$
% 	y $x_1=k_1^{-1}c_1$, donde $k,k_1\in K$ y $c,c_1\in C$. Como $K$ es normal
% 	en $K\rtimes Q$, podemos escribir a $xx_1$ como $xx_1=k_2c_2$, donde
% 	$k_2=k^{-1}(ck_1^{-1}c^{-1})\in K$, $c_2=cc_1\in C$. Luego 
% 	\[
% 		\delta(xx_1)xx_1=cc_1=\delta(x)x\delta(x_1)x_1
% 	\]
% 	implica que $\delta(xx_1)=\delta(x)x\delta(x_1)x^{-1}$. 
% 	Tenemos así una función $F\colon\mathcal{C}\to\Der(Q,K)$, $F(C)=\delta_C$.

% 	Vamos a construir ahora $G\colon\Der(Q,K)\to\mathcal{C}$. 
% 	Para
% 	cada $\delta\in\Der(Q,K)$ vamos a definir un complemento $\Delta$ de $K$ en $K\rtimes Q$: 
% 	\[
% 	\Delta=\{\delta(x)x:x\in Q\}.
% 	\]

% 	Veamos que $\Delta$ es un subgrupo de $K\rtimes Q$. Como $\delta(1)=1$,
% 	$1\in X$. Si $x,y\in Q$ entonces
% 	$\delta(x)x\delta(y)y=\delta(x)x\delta(y)x^{-1}xy=\delta(xy)xy\in \Delta$.
% 	Por último si $x\in Q$ entonces
% 	$(\delta(x)x)^{-1}=x^{-1}\delta(x)^{-1}xx^{-1}=\delta(x^{-1})x^{-1}$.
	
	
% 	Veamos que $\Delta\cap K=\{1\}$. Si $x\in Q$ es tal que $\delta(x)x\in K$
% 	entonces, como $\delta(x)\in K$, $x\in K\cap Q=\{1\}$. Si $g\in G$ entonces
% 	existen únicos $k\in K$, $x\in Q$ tales que $g=kx$. Escribimos
% 	$g=k\delta(x)^{-1}\delta(x)x$. Como $k\delta(x)^{-1}\in K$ y $\delta(x)x\in
% 	\Delta$, se concluye que $G=K\Delta$. Queda bien definida entonces la
% 	función $G\colon\Der(Q,K)\to\mathcal{C}$, $G(\delta)=\Delta$.

% 	Veamos ahora que $G\circ F=\id_{\mathcal{C}}$. 
% 	Sea $C\in\mathcal{C}$. Entonces 
% 	\[
% 	G(F(C))=G(\delta_C)=\{\delta_C(x)x:x\in
% 	Q\}=C,
% 	\]
% 	por construcción. (Vimos que $\delta_C(x)x\in C$. Recíprocamente,  si $c\in
% 	C$, escribimos $c=kx$ para únicos $k\in K$, $x\in Q$ y luego $x=k^{-1}c$
% 	que implica $c=\delta_c(x)x$.)

% 	Por último veamos que $F\circ G=\id_{\Der(Q,K)}$. Sea $\delta\in\Der(Q,K)$.
% 	Entonces 
% 	\[
% 	F(G(\delta))=F(\Delta)=\delta_{\Delta}.
% 	\]
% 	Queremos demostrar que $\delta_\Delta=\delta$.  Sea $x\in Q$. Existe
% 	$\delta(y)y\in\Delta$ para algún $y\in Q$ tal que $x=k^{-1}\delta(y)y$.
% 	Luego $\delta_{\Delta}(x)x=\delta(y)y$ y luego $\delta(x)=\delta(y)$ por la
% 	unicidad de la escritura.
% \end{sol}

We are now ready to prove the first version of the
Schur--Zassenhaus theorem. 

\begin{theorem}[Schur--Zassenhaus]
	\index{Schur--Zassenhaus!theorem}
	\label{thm:SchurZassenhaus:abeliano}
	Let $G$ be a finite group and $N$ be an abelian normal subgroup of $G$. If 
 	$|N|$ and $(G:N)$ are coprime, then $N$ admits a complement in $G$. Moreover, 
    all complements of $N$ are conjugate. 
\end{theorem}

\begin{proof}
	Let $T$ be a transversal of $N$ in $G$ and $\theta\colon G\to N$,
	$\theta(g)=d(gT,T)$. Since $N$ is abelian, Lemma~\ref{lem:d} implies that 
	$\theta$ is a 1-cocycle, where $G$ acts on $N$ by conjugation: 
	\begin{align*}
		\theta(xy)&=d(xyT,T)
		=d(xyT,xT)d(xT,T)\\
		&=(xd(yT,T)x^{-1})d(xT,T)=(x\cdot\theta(y))\theta(x).
	\end{align*}

	\begin{claim}
		$\theta|_N\colon N\to N$ is surjective. 
	\end{claim}

	If $n\in N$, Lemma~\ref{lem:d} implies that 
	$\theta(n)=d(nT,T)=n^{(G:N)}$. Since $|N|$ and $(G:N)$ are coprime, 
	there exist $r,s\in\Z$ such that $r|N|+s(G:N)=1$. Thus 
	\[
		n=n^{r|N|+s(G:N)}=(n^s)^{(G:N)}=\theta(n^s).
	\]

	Let $H=\ker\theta$. We prove that $H$ is a complement of $N$. 
	By Exercise~\ref{xca:1cocycle}, $H$ is a subgroup of $G$. By Lemma~\ref{lem:1cocycle}, 
	\[
		|N|=|\theta(G)|=(G:H)=\frac{|G|}{|H|}. 
	\]
	
	Since $N\cap H$ is a subgroup of $N$ and a subgroup of $H$, $N\cap H=\{1\}$, as the numbers 
	$|N|$ and $(G:N)=|H|$ are coprime. Since $|NH|=|N||H|=|G|$, we conclude that 
	$G=NH$. Hence $H$ is a complement of~$N$. 

	We now prove that two complements of $N$ are conjugate. 
	Let $K$ be a complement of $N$ in $G$. Since $NK=G$ and $N\cap K=\{1\}$, $K$ is a transversal of $N$. 
 Let $m=d(T,K)\in N$. Since the restriction map $\theta|_N$ is surjective, 
	there exists $n\in N$ such that $\theta(n)=m$. By Lemma~\ref{lem:d}, 
	\[
	kmk^{-1}=kd(T,K)k^{-1}=d(kT,kK)=d(kT,K)=d(kT,T)d(T,K)=\theta(k)m
	\]
    for all $k\in K$. 
	Since $N$ is abelian,
	$\theta(n^{-1})=m^{-1}$. Thus 
	\begin{align*}
		\theta(nkn^{-1})&=\theta(n)n\theta(kn^{-1})n^{-1}
		=m\theta(kn^{-1})\\
		&=m\theta(k)k\theta(n^{-1})k^{-1}
		=m\theta(k)km^{-1}k^{-1}=1.
	\end{align*}
	Therefore $nKn^{-1}\subseteq H=\ker\theta$. Since 
	$|K|=(G:N)=|H|$, we conclude that $nKn^{-1}=H$.
\end{proof}


\begin{theorem}[Schur--Zassenhaus]
	\index{Schur--Zassenhaus!theorem}
	\label{thm:SchurZassenhaus}
	Let $G$ be a finite group and $N$ be a normal subgroup of $G$. If $|N|$ and 
	$(G:N)$ are coprime, then $N$ admits a complement in $G$. 
\end{theorem}

\begin{proof}
	We proceed by induction on $|G|$. If there is a proper subgroup $K$ of 
	$G$ such that $NK=G$, then, since $(K:K\cap N)=(G:N)$ and $|N|$ are coprime,
	$(K:K\cap N)=(G:N)$ is coprime with $|K\cap N|$. Since $K\cap N$ is normal in $K$,
	the inductive hypothesis implies that $K\cap N$ admits a complement in $K$. Thus there exists 
    a subgroup $H$ of $K$ such that $|H|=(K:K\cap N)=(G:N)$. 

    Assume that there is no proper subgroup $K$ of $G$ such that 
	$NK=G$. We may assume that $N\ne\{1\}$ (otherwise, $G$ would be a complement of $N$ in $G$). Since $N$ is contained in 
    every maximal subgroup of $G$ (because, if there is a maximal subgroup $M\subsetneq G$ such that 
	$N\not\subseteq M$, then $NM=G$), it follows that $N\subseteq\Phi(G)$. By Frattini's theorem~\ref{thm:Frattini}, 
    $\Phi(G)$ is nilpotent. Thus $N$ is nilpotent and then $Z(N)\ne\{1\}$. Let $\pi\colon G\to
	G/Z(N)$ be the canonical map. Since $N$ is normal in $G$ and $Z(N)$ is characteristic in $N$, 
    $Z(N)$ is normal in $G$.  Moreover, 
	\[
	(\pi(G):\pi(N))=\frac{|\pi(G)|}{|\pi(N)|}=\frac{|G/Z(N)|}{|N/N\cap Z(N)|}=(G:N)
	\]
	is coprime with $|N|$. Then $(\pi(G):\pi(N))$ is coprime with $|\pi(N)|$. By the inductive hypothesis, 
	$\pi(N)$ admits a complement in $G/Z(N)$, say $\pi(K)$
	for some subgroup $K$ of $G$. Hence $G=NK$, as 
	$\pi(G)=\pi(N)\pi(K)=\pi(NK)$. 
	Since $K=G$ (because there is no $K$ such that $G=NK$), 
	$\pi(N)$ is abelian, as 
	\[
		\pi(Z(N)=\pi(N)\cap\pi(K)=\pi(N)\cap\pi(G)=\pi(N).
	\]
	Thus $N\subseteq Z(N)$ is abelian. By Theorem~\ref{thm:SchurZassenhaus:abeliano}, the subgroup $N$ 
    admits a complement. 
\end{proof}

\begin{theorem}[Schur--Zassenhaus conjugation theorem]
	\label{thm:SchurZassenhaus:conjugation}
    Let $G$ be a finite group and $N$ be a normal subgroup of $G$ such that 
    $|N|$ and 
	$(G:N)$ are coprime. If either $N$ or $G/N$ is solvable, then 
    all complements of $N$ in $G$ are conjugate. 
\end{theorem}

%\begin{proof}
%	Sea $G$ un contraejemplo minimal, es decir: existen complementos $K_1$ y
%	$K_2$ a $N$ en $G$ que no son conjugados.
%
%	\begin{claim}
%		$N$ es minimal en $G$.
%	\end{claim}
%
%	Si $M\subseteq N$ es minimal normal en $G$, $M\ne1$ pues $N\ne1$. Sea
%	$\pi\colon G\to G/M$ el morfismo canónico. El grupo $\pi(G)$ contiene un
%	subgrupo normal $\pi(N)$ de índice coprimo con $|\pi(N)|$. Además
%	$\pi(K_1)$ y $\pi(K_2)$ complementan a $\pi(N)$. Como $|G|$ es minimal,
%	$\pi(K_1)$ y $\pi(K_2)$ son conjugados en $\pi(G)$, es decir: existe $x\in G$ tal que 
%	$\pi(K_1)=\pi(xK_2x^{-1})$.
%
%\end{proof}

\begin{proof}
	Let $G$ be a minimal counterexample to the theorem, that is there are complements $K_1$ and 
	$K_2$ of $N$ in $G$ such that $K_1$ and $K_2$ are not conjugate. 

	\begin{claim}
		Every subgroup $U$ of $G$ satisfies the assumptions of the theorem with respect to the normal subgroup 
        $U\cap N$.
%		Sea $U$ un subgrupo de $G$. Entonces $U$ satisface las hipótesis del
%		teorema con respecto al subgrupo normal $U\cap N$. Si $U$ contiene un
%		complemento $H$ para $N$ en $G$, entonces $H$ complementa a $U\cap N$
%		en $U$.
	\end{claim}
	
	Since $N$ is normal in $G$, $U\cap N$ is normal in $U$. Moreover, $|U\cap N|$ and 
	$(U:U\cap N)$ are coprime, as $|U\cap N|$ divides $|N|$ and $(U:U\cap
	N)=(UN:N)$ divides $(G:N)$.  If $G/N$ is solvable, then $U/U\cap N$
	is solvable, as $U/U\cap N$ is isomorphic to a subgroup of $G/N$. If $N$ is 
	solvable, then so is $U\cap N$.
%
%	Como $|H|$ divide a $|U|$ y $|H|$ es coprimo con $|U\cap N|$, se tiene que
%	$|H|$ divide a $(U:U\cap N)$. Como además $(U:U\cap N)$ divide a
%	$(G:N)=|H|$, se concluye que $|H|=|U:U\cap N|$. Luego $H$ complementa a
%	$U\cap N$ en $U$.

	\begin{claim}
		If there is a normal subgroup $L$ of $G$ such that $\pi(N)$ is normal in 
		$\pi(G)$, where $\pi\colon G\to G/L$ is the canonical map, then 
    	$\pi(G)$ satisfies the theorem's assumptions with respect to $\pi(N)$.
		In this case, if $H$  is a complement of $N$ in $G$, then $\pi(H)$ 
		is a complement of $\pi(N)$ in $\pi(G)$.
	\end{claim}

	If $N$ is solvable, then so is $\pi(N)$. If $G/N$ is solvable, then so is 
	$\pi(G)/\pi(N)\simeq G/NL$. Moreover, 
	$(\pi(G):\pi(N))=\frac{|G/L|}{|N/N\cap L|}$ divides $(G:N)$. 
	
	If $H$ is a complement of $N$ in $G$, $|\pi(H)|$ and $|\pi(N)|$ are 
	coprime. Then $\pi(H)$ is a complement of $\pi(N)$, as 
	$\pi(G)=\pi(N)\pi(H)=\pi(NH)$ and 
	$\pi(N)\cap\pi(H)=\{1\}$. 

	\begin{claim}
		$N$ is minimal normal in $G$.
	\end{claim}

	Let $M\ne\{1\}$ be a normal subgroup of $G$ such that $M\subseteq N$. Let $\pi\colon G\to G/M$ be the canonical map. 
	Then $\pi(G)$ satisfies the theorem's assumptions with respect to the normal subgroup 
	$\pi(N)$. By the minimality of $|G|$, there exists 
	$x\in G$ such that $\pi(xK_1x^{-1})=\pi(K_2)$. The subgroup 
	$U=MK_2$ satisfies the theorem's assumptions with respect to the normal subgroup 
	$U\cap N$. Since $xK_1x^{-1}\cup K_2\subseteq U$,
	we conclude that both $xK_1x^{-1}$ and $K_2$ complement $U\cap N$ in $U$.
	Hence $MK_2=G$, as $xK_1x^{-1}$ and $K_2$ are not conjugate and $G$ is a minimal counterexample. 
	minimal. Therefore $M=N$, as 
	\[
		\frac{|K_2|}{|M\cap K_2|}=(MK_2:M)=(G:M)=\frac{|NK_2|}{|M|}=(N:M)|K_2|.
	\]

	\begin{claim}
		$N$ is not solvable and $G/N$ is solvable. 
	\end{claim}
	
	Otherwise, by Lemma~\ref{lem:minimal_normal}, $N$ is abelian (because it is minimal normal). This contradicts
    Theorem~\ref{thm:SchurZassenhaus:abeliano}, as it states that 
	$K_1$ and $K_2$ are conjugate. 
 
	\medskip
	Let $p\colon G\to G/N$ be the canonical map and $S$ be a subgroup such that $p(S)$
	is minimal normal in $p(G)=G/N$.  By Lemma~\ref{lem:minimal_normal},
	$p(S)$ is a $p$-group for some prime number $p$. Since $G=NK_1=NK_2$ and $N\subseteq
	S$, Dedekind's lemma~\ref{lem:Dedekind} implies that 
	\[
	S=N(S\cap K_1)=N(S\cap K_2).
	\]
	Hence both $S\cap K_1$ and $S\cap K_2$
	complement $N$ in $S$. Since $p(S)=p(S\cap K_1)=p(S\cap K_2)$  is a $p$-group, 
 	$p$ divides $|S|$. The theorem's assumptions hold for $S$ with respect to the normal subgroup $N$, 
    so $|N|$ and $(S:N)$ are coprime. If $p\mid |N|$, then 
	$p\nmid (S:N)=|S\cap K_1|=|S\cap K_2|$, a contradiction. Thus $p\nmid |N|$ and 
	hence $p\nmid |S|$. This implies that both $S\cap K_1$ and $S\cap K_2$ are Sylow 
	$p$-subgroups of $S$, as 
	\[
		|S\cap K_1|=(S:N)=|S\cap K_2|.
	\]
	By Sylow's theorem, there exists $s\in
	S$ such that 
    \[
	S\cap sK_1s^{-1}=S\cap K_2.
	\]
	In particular, $S\ne G$ by the minimality of $G$.
	Let 
	\[
		L=S\cap K_2=S\cap sK_1s^{-1}\ne\{1\}.
	\]
	Since $S$ is normal in $G$, $sK_1s^{-1}\cup K_2\subseteq N_G(L)$ (because $L$
	is both normal in $sK_1s^{-1}$ and in $K_2$). The subgroups $sK_1s^{-1}\subseteq
	N_G(L)$ and $K_2\subseteq N_G(L)$ complement $N\cap N_G(L)$ in $N_G(L)$. Hence 
	$N_G(L)=G$ by the minimality of $G$ (if $N_G(L)\ne G$, then both 
	$sK_1s^{-1}$ and $K_2$ are conjugate in $G$ because they are conjugate in  $N_G(L)$). Therefore 
	$L$ is normal in $G$. 
	
	Let $\pi_L\colon G\to G/L$ be the canonical map. Since both 
	$\pi_L(K_1)$ and $\pi_L(K_2)$ complement $\pi_L(N)$ in $\pi_L(G)$, the minimality of 
	$|G|$ implies that there exists $g\in G$ such that $\pi_L(gK_1g^{-1})=\pi_L(K_2)$, that is 
	there exists $g\in G$ such that $(gK_1g^{-1})L=K_2L$.  Hence $gK_1g^{-1}\cup
	K_2\subseteq \langle K_2,L\rangle=K_2$, because $L\subseteq K_2$. In conclusion, 
	$gK_1g^{-1}=K_2$, a contradiction to the minimality of $|G|$. 
%	Sea $L$ un subgrupo maximal normal de $G$ tal que $N\subseteq L$. Por
%	definición $L\ne G$. Como $L\cap K_1$ y $L\cap K_2$ complementan a $N$ en
%	$L$, la minimalidad de $G$ implica que existe $x\in G$ tal que 
%	\[ 
%	L\cap K_2=x(L\cap K_1)x^{-1}=L\cap xK_1x^{-1}.
%	\]
%	Sea $D=L\cap K_2$. Como $L$ es normal en $G$, $D$ es normal en $K_2$ y en
%	$xK_1x^{-1}$. Como $K_2$ y $xK_1x^{-1}$ son complementos para 
%	$N$ en $G$ y además
%	$xK_1x^{-1}\cup K_2\subseteq N_G(D)$, la minimalidad de $G$ implica que $N_G(D)=G$.
%	Si $N$ es resoluble, $N\ne1$ (pues de lo contrario $G=H=K$ y no hay nada
%	para demostrar). Sea $L\subseteq N$ un subgrupo minimal normal de $G$. Por
%	el lema~\ref{lemma:minimal_normal}, $L$ es abeliano\dots
%
%	Si $G/N$ es resoluble,\dots 
\end{proof}


By the Feit--Thompson theorem, in the previous theorem 
we do not need to assume that either $N$ or $G/N$ is solvable. Since every group of odd order is solvable 
and $|N|$ and $(G:N)$ are coprime, one of these groups should have odd order. 

\begin{theorem}
	\label{thm:solvable_maximal}
	Let $G$ be a finite solvable group and $p$ a prime number dividing $|G|$. There exists a maximal 
    subgroup $M$ of $G$ of index a power of $p$. 
\end{theorem}

\begin{proof}
	We proceed by induction on $|G|$. If $G$ is a $p$-group, the result clearly holds. So we may assume that $|G|$ is divisible by at least two different prime numbers. 
    Let $p$ be a prime dividing $|G|$, $N$ be a minimal normal subgroup of $G$ and 
    $\pi\colon G\to G/N$ be the canonical map. Since $G$ is solvable, by Lemma~\ref{lem:minimal_normal}, 
    $N$ is a $q$-group for some prime $q$. Since $G/N$ is solvable, if $p$ divides 
	$(G:N)$, then, by the inductive hypothesis, $G/N$ has a maximal subgroup 
 	$M_1$ of index a power of $p$. By the correspondence theorem, 
  $M=\pi^{-1}(M_1)$ is a maximal subgroup of $G$ of index a power of $p$. 
  $p$. If $p$ does not divide $(G:N)$, then $p$ divides $|N|$. Thus 
	$N\in\Syl_p(G)$. Since $N$ is normal in $G$ and $|N|$ and $|G/N|$ are coprime, by 
	Schur--Zassenhaus theorem~\ref{thm:SchurZassenhaus}, 
	there exists a complement $K$ of $N$ in $G$, that is $G=NK$ and $N\cap K=\{1\}$. Let 
	$M$ be a maximal subgroup containing $K$. Then $(G:M)$ is a power of $p$. 
\end{proof}

We now discuss an application to finite super-solvable groups. 

\begin{definition}
	\index{Group!lagrangian}
	A finite group $G$ is said to be \textbf{lagrangian} if for each $d$ dividing $|G|$ 
	there exists a subgroup of $G$ of order $d$.
\end{definition}

The group $\Alt_4$ is not lagrangian, as it has no subgroups of order six. 

\begin{theorem}
	Every finite super-solvable group is lagrangian. 
\end{theorem}

\begin{proof}
	Let $p$ be a prime number dividing $|G|$. Since subgroups of super-solvable groups are super-solvable, it is enough to 
    show that there exists a subgroup of index $p$. 
	Since $G$ is solvable, there exists a maxima subgroup $M$ of index 
	$p^{\alpha}$ by Theorem~\ref{thm:solvable_maximal}. Since maximal subgroups of super-solvable groups have prime index 
    by Theorem~\ref{thm:super_structure}, we conclude that $\alpha=1$.
\end{proof}

See \cite{MR294497} for an elementary proof. 

\subsection{*Hall's theory for solvable groups}

As an application of the Schur--Zassenhaus theorem, 
we present Hall's theory of solvable groups. 
For an elementary presentation, see \cite{MR600654}. 

\begin{definition}
\index{$\pi$-number}
\index{$\pi$-group}
\index{$\pi$-subgroup}
Let $G$ be a finite group and $\pi$ be a set of prime numbers. We say that 
$G$ is a \textbf{$\pi$-group} if every prime dividing $|G|$ belongs to $\pi$. 
Similarly, a $\pi$-subgroup of $G$ is a subgroup of $G$ that is also a $\pi$-group.  
\end{definition}

For a set $\pi$ of prime numbers, 
we define a $\pi$-number as an integer whose prime divisors 
belong to $\pi$. The set of prime numbers not belonging to $\pi$ will be denoted 
as $\pi'$. Thus a $\pi'$-number is an integers not divisible by 
the prime numbers of $\pi$. 

\begin{definition}
	\index{Hall!subgroup}
	Let $G$ be a group and $\pi$ be a set of prime numbers. A subgroup $H$ of $G$ 
    is a \textbf{Hall $\pi$-subgroup} if $H$ is a $\pi$-subgroup of $G$ and 
    $(G:H)$ is a $\pi'$-number.
\end{definition}

We now prove that a finite solvable group of order $nm$ with $\gcd(n,m)=1$ 
always admits a subgroup of order $m$. 

\begin{theorem}[Hall's!existence theorem]
	\index{Hall!existence theorem}
 	\label{theorem:HallE}
	Let $\pi$ be a set of prime numbers and $G$ be a finite solvable group. 
    Then $G$ has a Hall $\pi$-subgroup. 
\end{theorem}

\begin{proof}
	Assume that $|G|=nm>1$ and $\gcd(n,m)=1$. We want to show that $G$ admits a
	subgroup of order $m$. We proceed by induction on $|G|$. Let $K$ be a minimal
	normal subgroup of $G$ and $\pi\colon G\to G/N$ be the canonical map. Since $G$
	is solvable, $K$ is a $p$-group (Lemma~\ref{lem:minimal_normal}).
	
	There are two cases to consider. Assume first that $p$ divides $m$. Since
	$|G/K|<|G|$, the inductive hypothesis and the correspondence theorem imply that
	there exists a subgroup $J$ of $G$ containing $K$ such that $\pi(J)$ is a
	subgroup of 
    $\pi(G)=G/K$ of order $m/|K|$. Then $J$ has order $m$, as 
    \[
	m/|K|=|\pi(J)|=\frac{|J|}{|K\cap J|}=(J:K).
	\]

	Assume now that $p$ does not divide $m$. By the inductive hypothesis and the
	correspondence theorem, there exists a subgroup $H$ of $G$ containing $K$ such
	that $\pi(H)$ is a subgroup of $G/K$ of order $m$. Since $|H|=m|K|$, $K$ is
	normal in $H$ and $|K|$ is coprime with $|H:K|$, the Schur--Zassenhaus theorem
	(Theorem~\ref{thm:SchurZassenhaus}) implies that there exists a complement $J$
	of $K$ in $H$. Hence $J$ is a subgroup of $G$ such that $|J|=m$.
\end{proof}

\begin{example}
	The group $\Alt_5$ contains a Hall $\{2,3\}$-subgroups isomorphic to 
 	$\Alt_4$.
\end{example}

\begin{example}
	The simple group $\PSL_3(2)$ of order $168$ does not contain Hall $\{2,7\}$-subgroups.
\end{example}

\begin{theorem}[Hall's conjugation theorem]
	\index{Hall!conjugation theorem}
	\label{theorem:HallC}
	Let $G$ be a finite solvable group and $\pi$ be a set of prime numbers. 
    Then all two Hall $\pi$-subgroups of $G$ are conjugate. 
\end{theorem}

\begin{proof}
	We may assume that $G\ne\{1\}$. We proceed by induction on $|G|$.  Let $H$
	and $K$ be Hall $\pi$-subgroups of $G$. Let $M$ be a minimal normal subgroup of 
    $G$ and $\pi\colon G\to G/M$ be the canonical map. Since $G$ is solvable, 
	$M$ is a $p$-group for some prime number $p$ (Lemma~\ref{lem:minimal_normal}). 
    Since $\pi(H)$ and $\pi(K)$ are both Hall 
	$\pi$-subgroups of $G/M$, by the inductive hypothesis, 
    the subgroups $\pi(H)$ and $\pi(K)$ are 
	conjugate in $G/M$. Thus there exists $g\in G$ such that $gHMg^{-1}=KM$. 

	There are two cases to consider. Assume first that $p\in\pi$. Since $|HM|$ and 
	$|KM|$ are $\pi$-numbers and $|H|=|K|$ is the largest $\pi$-number dividing $|G|$, 
    we conclude that $H=HM$ and $K=KM$. In particular, $gHg^{-1}=K$. 

	Assume now that $p\not\in\pi$. Then $K$ admits a complement $M$ in 
 	$KM$, as $K\cap M=\{1\}$. We claim that $gHg^{-1}$ complements $M$ in $KM$. Since 
	$M$ is normal in $G$, 
 	\[
	(gHg^{-1})M=gHMg^{-1}=KM,
	\]
	and $gHg^{-1}\cap M=\{1\}$, as $p\not\in\pi$. These complements are conjugate 
    by the Schur--Zassenhaus theorem~\ref{thm:SchurZassenhaus:conjugation}.
\end{proof}

\begin{corollary}
	Let $G$ be a finite group, $N$ a normal subgroup of $G$ and $n=|N|$. 
 	Assume that either $N$ of $G/N$ is solvable. 
    If $|G:N|=m$ is coprime with $n$ and 
    $m_1$ divide a $m$, then every subgroup of $G$ of order $m_1$ 
    is contained in some subgroup of order $m$. 
\end{corollary}

\begin{proof}
	Let $H$ be a complement of $N$ in $G$. Then $|H|=m$. Let $H_1$
	be a subgroup of $G$ such that $|H_1|=m_1$. 
	Since $\gcd(n,m)=1$, $m_1=|H_1|=|H\cap NH_1|$, as 
	\[
	\frac{|H||N||H_1|}{|H\cap NH_1|}=
	\frac{|H||NH_1|}{|H\cap NH_1|}=|H(NH_1)|=|G|=|NH|=|N||H|.
	\]
	Since both $H_1$ and $H\cap NH_1$ are complements of $N$ in $NH_1$, and both 
    groups have orders coprime with $n$, there exists 
	$g\in G$ such that $H_1=g(H\cap NH_1)g^{-1}$. Thus  
	$H_1\subseteq gHg^{-1}$ and hence $|gHg^{-1}|=m$. 
\end{proof}


\subsection{Subnormality}

\begin{definition}
	\index{Subgroup!subnormal}
	Let $G$ be a group. A subgroup $H$ of $G$ is said to be \text{subnormal} in $G$ if there is a sequence 
    of subgroups 
	\[
		H=H_0\subseteq H_1\subseteq\cdots\subseteq H_k=G		
	\]
	with $H_i$ normal in $H_{i+1}$ for all $i\in\{0,\dots,k-1\}$. 
\end{definition}

\begin{example}
	Let $G=\Sym_4$. Then $K=\{\id,(12)(34),(13)(24),(14)(23)\}$ is normal in $G$. 
	The subgroup $L=\{\id,(12)(34)\}$ is subnormal in $G$ (and not normal). 
	es subnormal. 
\end{example}

\begin{exercise}
\label{xca:correspondence_subnormality}
    Prove that the correspondence theorem preserves subnormality. 
\end{exercise}

\begin{theorem}
	\label{thm:subnormal}
	Let $G$ be a finite group. Then $G$ is nilpotent if and only if every subgroup of $G$ is subnormal in $G$. 
\end{theorem}

\begin{proof}
	Assume first that every subgroup of $G$ is subnormal in $G$. Let $H$ be a subnormal subgroup of $G$, where 
	\[
		H=H_0\subseteq H_1\subseteq\cdots\subseteq H_k=G
	\]
	with $H_i$ normal in $H_{i+1}$. Without loss of generality, we may assume that 
	$H\subsetneq H_1$. Since $H\subsetneq H_1\subseteq N_G(H)$, 
	$G$ is nilpotent by Exercise~\ref{xca:normalizadora}.

	Assume now that $G$ is nilpotent. Let $H$ be a subgroup of $G$.
	We proceed by induction on $(G:H)$. If $(G:H)=1$, then $H=G$ and the theorem holds. If 
	$H\ne G$, since $H\subsetneq N_G(H)$ by Lemma~\ref{lem:normalizadora}, 
	\[
		(G:N_G(H))<(G:H).
	\]
	By the inductive hypothesis, $N_G(H)$ is subnormal in $G$. Since $H$ is 
	normal in $N_G(H)$, we conclude that $H$ is subnormal in $G$.
\end{proof}

\index{Subgroup!central}

\begin{corollary}
	Let $G$ be a group and $K$ be a central subgroup of $G$ (that is, $K\subseteq Z(G)$).
	Then $G$ is nilpotent if and only if $G/K$ is nilpotent. 
\end{corollary}

\begin{proof}
	If $G$ is nilpotent, then so is $G/K$. Conversely, let 
	$\pi\colon G\to G/K$ be the canonical map and $U$ be a subgroup of $G$. Since 
	$G/K$ is nilpotent, Theorem~\ref{thm:subnormal} implies that 
	$\pi(U)$ is a subnormal subgroup of $G/K$. By the correspondence theorem, 
 	$UK$ is a subnormal subgroup of $G$. Since $K$ is central, $U$ is normal in 
	$UZ$. Hence $U$ is subnormal in $G$ and therefore $G$ is nilpotent by Theorem~\ref{thm:subnormal}.
\end{proof}

\begin{theorem}
	\label{thm:F(G)subnormalidad}
	Let $G$ be a finite group and $H$ be a subgroup of $G$. Then $H$ is nilpotent and subnormal in $G$ if and only if 
    $H\subseteq F(G)$.
\end{theorem}

\begin{proof}
	Assume first that $H\subseteq F(G)$. Since $F(G)$ is nilpotent by Theorem~\ref{thm:Fitting}, 
    so is $H$. Moreover, since $H$ is subnormal in $F(G)$ (Theorem~\ref{thm:subnormal}) 
    and $F(G)$ is normal in $G$, $H$ is
	subnormal in $G$.

	Assume now that $H$ is nilpotent and subnormal in $G$. We proceed by induction on $|G|$. 
    If $H=G$, then the result holds. Assume then that $H\ne G$. Since $H$ is subnormal in $G$, there is a sequence 
	\[
		H=H_0\subseteq H_1\subseteq\cdots\subseteq H_k=G
	\]
    of subgroups of $G$ with $H_i$ normal in $H_{i+1}$ for all $i$. 
	Let $M=H_{k-1}$. Since $M\ne G$ and $M$ is normal in $G$, 
	$H\subseteq F(M)$ by the inductive hypothesis. Thus $H\subseteq F(M)=M\cap
	F(G)\subseteq F(G)$ by Corollary~\ref{cor:McapF(G)}.
\end{proof}

Before proving another important theorem of Wielandt, we need a lemma. 

\begin{lemma}
	\label{lem:McapN=1}
	Let $M$ and $N$ be normal subgroups of $G$ such that $M\cap N=\{1\}$.
	Then $M\subseteq C_G(N)$.
\end{lemma}

\begin{proof}
	Let $m\in M$ and $n\in N$. Then $[n,m]=(nmn^{-1})m\in M$, since $M$ is normal in $G$ and 
	Moreover, $[n,m]=n(mn^{-1}m^{-1})\in N$, since $N$ is normal in 
	$G$. Thus $[n,m]\in M\cap N=\{1\}$.
\end{proof}

\begin{exercise}
\label{xca:characteristically_simple}
\index{Group!characteristically simple}
    A group $G$ is said to be \textbf{characteristically simple} if $G$ is non-trivial 
    and has no proper characteristic subgroups. 
    Prove that any minimal normal subgroup of $G$ is characteristically simple. 
\end{exercise}

\begin{definition}
    \index{Socle}
    Let $G$ be a group. If $G$ admits minimal normal subgroups, the \textbf{socle} of $G$ 
    is defined as the subgroup $\Soc(G)$ of $G$ generated by all minimal normal subgroups of $G$. If $G$ admits no minimal normal subgroups, 
    then $\Soc(G)=\{1\}$. 
\end{definition}

For example, $\Soc(\Z)=\{0\}$ and $\Soc(\SL_2(3))\simeq C_2$. 

\begin{exercise}
\label{xca:Soc_direct_product}
% Robinson 3.3.11 with $\Omega=Inn(G)$
    Prove that the socle of a group is a direct product of minimal normal subgroups. 
\end{exercise}

\begin{exercise}
\label{xca:caracteristically_simple}
% Robinson 3.3.15, page 87
Prove the following statements:
    \begin{enumerate}
        \item A direct product of isomorphic simple groups is characteristically simple.
        \item A characteristically simple group with at least one minimal normal subgroup is a direct product of isomorphic simple groups. 
    \end{enumerate}
\end{exercise}

\begin{theorem}[Wielandt]
	\label{thm:MsubsetNG(S)}
    \index{Wielandt's!theorem}
	Let $G$ be a finite group. If $S$ is a subnormal group of $G$ and 
	$M$ is a minimal normal subgroup of $G$, then $M\subseteq N_G(S)$.
\end{theorem}

\begin{proof}
	We proceed by induction on $|G|$. If $S=G$ the result holds. So assume that 
	$S\ne G$.  Since $S$ is subnormal in $G$, there exists a sequence 
	\[
		S=S_0\subseteq S_1\subseteq\cdots\subseteq S_{k-1}\subseteq S_k=G
	\]
    of subgroups of $G$ such that $S_i$ is normal in $S_{i+1}$ for all $i$. 
	Let $N=S_{k-1}$. 

	If $M\cap N\ne\{1\}$, then $M\subseteq N$ (because since $M$ and $N$ are both normal in $G$, 
	$M\cap N=M$ by the minimality of $M$). We claim that 
	$M\subseteq\Soc(N)$.  Since $M\ne\{1\}$ and $M$ is normal in $N$,
	$M\cap\Soc(N)\ne\{1\}$. Moreover, since $\Soc(N)$ is characteristic in $N$ and $N$ is 
	normal in $G$, it follows that $\Soc(N)$ is normal in $G$. Hence $M\cap\Soc(N)$ is a normal subgroup of $G$. 
	Since $\{1\}\ne M\cap\Soc(N)\subseteq M$, we conclude that 
	$M\cap\Soc(N)=M$ by the minimality of $M$. By the inductive hypothesis, 
	every minimal normal subgroup of $N$ normalizes $S$. Thus 
	$\Soc(N)\subseteq N_N(S)\subseteq N_G(S)$ and therefore 
	\[
	M\subseteq\Soc(N)\subseteq N_G(S).
	\]
	If $M\cap N=1$, Lemma~\ref{lem:McapN=1} implies that 
	\[
	M\subseteq C_G(N)\subseteq C_G(S)\subseteq N_G(S). \qedhere
	\]
\end{proof}

\begin{corollary}
    Let $G$ be a finite group and $S$ be a subnormal subgroup of $G$. Then 
    \[
    \Soc(G)\subseteq N_G(S).
    \]
\end{corollary}

\begin{proof}
	By Theorem~\ref{thm:MsubsetNG(S)}, every minimal normal subgroup of $G$ is contained in $N_G(S)$. Then 
	$\Soc(G)=\langle M:M\text{ minimal normal subgroup of $G$}\rangle\subseteq N_G(S)$.
\end{proof}

\begin{theorem}[Wielandt]
	\label{thm:STsubnormal}
    \index{Wielandt's lattice theorem}
	Let $G$ be a finite group and $S$ and $T$ be subnormal subgroups of $G$. Then $S\cap T$ and 
	$\langle S,T\rangle$ are subnormal in $G$.
\end{theorem}

\begin{proof}
	We first prove that $S\cap T$ is subnormal in $G$. Since subnormality is a transitive relation, it is enough to see that 
	$S\cap T$ is subnormal in $T$.
	Since $S$ is subnormal in $G$, there exists a sequence 
	\[
		S=S_0\subseteq S_1\subseteq \cdots\subseteq S_k=G
	\]
    of subgroups of $G$ such that $S_i$ is normal in $S_{i+1}$ for all $i$. 
	Each $S_{j-1}\cap T$ is normal in $S_j\cap T$. Then $S\cap T$ is 
	subnormal in $T$.
	
	
	We now prove that $\langle S,T\rangle$ is subnormal in $G$ 
	We proceed by induction on $|G|$. Assume that $G\ne\{1\}$. Let $M$ be a minimal normal subgroup of $G$ and 
    $\pi\colon G\to G/M$ be the canonical map. Since 
	both $\pi(S)$ and $\pi(T)$ subnormal in $G/M$ and $|G/M|<|G|$,
	the inductive hypothesis implies that 
	\[
	\pi(\langle S,T\rangle M)=\pi(\langle S,T\rangle)=\langle \pi(S),\pi(T)\rangle
	\]
	is subnormal in $G/M$. By the correspondence theorem, $\langle S,T\rangle M$ is 
	subnormal in $G$. Theorem~\ref{thm:MsubsetNG(S)}
	implies that $M\subseteq N_G(S)$ and $M\subseteq N_G(T)$. Hence $M\subseteq
	N_G(\langle S,T\rangle)$. Since $\langle S,T\rangle$ is normal in
	$\langle S,T\rangle M$ and $\langle S,T\rangle M$ is subnormal in $G$, we conclude that 
	$\langle S,T\rangle$ is subnormal in $G$.
\end{proof}


%\section{Torres de automorfismos}
%
%\begin{exercise}
%	\label{exercise:Inn(G)}
%	Sea $G$ un grupo. Demuestre las siguientes afirmaciones:
%	\begin{enumerate}
%		\item La conjugación $\gamma\colon G\to\Inn(G)$, $g\mapsto \gamma_g$,
%			es morfismo con núcleo $Z(G)$.
%		\item $\Inn(G)$ es normal en $\Aut(G)$.
%		\item Si $Z(G)=1$ entonces $C_{\Aut(G)}(\Inn(G))=1$.
%		\item Si $Z(G)=1$ entonces $Z(\Aut(G))=1$.
%	\end{enumerate}
%\end{exercise}
%
%\begin{svgraybox}
%	\begin{enumerate}
%		\item Es morfismo pues
%			\[
%			\gamma_{gh}(x)=(gh)x(gh)^{-1}=\gamma_g\gamma_h(x)
%			\]
%			y el núcleo es
%			\[
%			\ker\gamma=\{g\in G:\gamma_g=\id\}=\{g\in G:gxg^{-1}=x\text{ para todo $x\in G$}\}=Z(G).
%			\]
%		\item Es trivial pues $f\gamma_gf^{-1}=\gamma_{f(g)}$.
%		\item Sea $f\in C_{\Aut(G)}(\Inn(G))$. Como
%			$f\gamma_gf^{-1}=\gamma_{f(g)}$, 
%	Sea $\sigma\in\Aut(\Aut(G))$. Sabemos por el
%	ejercicio~\ref{exercise:Inn(G)} que $\Inn(G)$ es normal en $\Aut(G)$, y
%	entonces $\sigma(\Inn(G))$ es normal en $\Aut(G)$. El grupo
%	\[
%	\Inn(G)\simeq G/Z(G)\simeq G
%	\]
%	es simple y $\sigma(\Inn(G))\cap \Inn(G)$ es normal en $\Inn(G)$; entonces
%	hay dos posibilidades: $\Inn(G)\cap\sigma(\Inn(G))=1$ o bien
%	$\Inn(G)\cap\sigma(\Inn(G))=\sigma(\Inn(G))$. Basta demostrar que
%	$\Inn(G)\cap\sigma(\Inn(G))\ne1$. 
%
%	Si $\Inn(G)\cap\sigma(\Inn(G))=1$ entonces, al usar la normalidad de
%	$\sigma(\Inn(G)$ y de $\Inn(G)$ en $\Aut(G)$, tendríamos
%	$[\Inn(G),\sigma(\Inn(G))]\in \Inn(G)\cap \sigma(\Inn(G))=1$.  Luego, por
%	el el ejercicio ~\ref{exercise:Inn(G)}, 
%	\[
%	G\simeq \Inn(G)\simeq\sigma(\Inn(G))\subseteq
%	C_{\Aut(G)}(\Inn(G))=1,
%	\]
%	una contradicción.
%\end{proof}
%
%\begin{theorem}
%	\label{theorem:simple=>completo}
%	Sea $G$ un grupo simple no abeliano. Entonces $\Aut(G)$ es completo.
%\end{theorem}
%
%\begin{proof}
%	Como $Z(G)=1$, el ejercicio~\ref{exercise:Inn(G)} implica que
%	$Z(\Aut(G))=1$. Por el lema~\ref{lemma:Inn(G)char}, $\Inn(G)$ es
%	característico en $\Aut(G)$. Queremos demostrar que 
%	$\Aut(G)\subseteq \Inn(G)$.
%	
%	Sean $\sigma\in\Aut(\Aut(G))$, $g\in G$ y $\gamma_g\in\Inn(G)$. Como
%	$\sigma(\Inn(G))\subseteq\Inn(G)$, existe $\alpha(g)\in G$ tal que
%	$\sigma(\gamma_g)=\gamma_{\alpha(g)}$. Queda definida entonces una función
%	$\alpha\colon G\to G$ tal que $\sigma(\gamma_g)=\gamma_{\alpha(g)}$ para
%	todo $g\in G$.
%
%	\begin{claim}
%		$\alpha\in\Aut(G)$. 
%	\end{claim}
%
%	Veamos que $\alpha$ es inyectiva: si $\alpha(g)=\alpha(h)$ entonces
%	\[
%	\sigma(\gamma_g)=\gamma_{\alpha(g)}=\gamma_{\alpha(h)}=\sigma(\gamma_h)
%	\]
%	y luego $\gamma_g=\gamma_h$; esto implica que $h=g$ pues
%	$\ker(\gamma)=Z(G)=1$.  Luego $\alpha$ es biyectiva. Además $\alpha$ es
%	morfismo pues 
%	\[
%	\gamma_{\alpha(gh)}=\sigma(\gamma_{gh})=\sigma(\gamma_h\gamma_h)=\sigma(\gamma_h)\sigma(\gamma_h)=\gamma_{\alpha(g)}\gamma_{\alpha(h)}.
%	\]
%
%	\begin{claim}
%		$\sigma=\gamma_\alpha$.
%	\end{claim}
%
%	Sea $\tau=\sigma\gamma_{\alpha}^{-1}$ y sea $h\in G$. Entonces
%	\[
%		\tau(\gamma_h)=\sigma\gamma_{\alpha}^{-1}\gamma_h
%		=\sigma(\alpha^{-1}\gamma_h\alpha)
%		=\sigma\gamma_{\alpha^{-1}(h)}
%		=\gamma_{\alpha\alpha^{-1}(h)}=\gamma_h.
%	\]
%	Si $\beta\in\Aut(G)$ y $g\in G$ entonces 
%	\[
%		\beta\gamma_g\beta^{-1}
%		=\gamma_{\beta(g)}
%		=\tau(\gamma_{\beta(g)})
%		=\tau(\beta)\tau(\gamma_g)\tau(\beta)^{-1}
%		=\tau(\beta)\gamma_g\tau(\beta)^{-1}.
%	\]
%	Como entonces $\tau(\beta)\beta^{-1}\in C_{\Aut(G)}(\Inn(G))=1$ para todo
%	$\beta$, $\tau=\id$ y luego $\sigma=\gamma_{\alpha}$.
%\end{proof}
%
%Si $G$ es un grupo con centro trivial entonces $G\hookrightarrow\Aut(G)$ pues 
%\[
%G\simeq
%G/Z(G)\simeq\Inn(G)\subseteq\Aut(G).
%\]
%Como también $\Aut(G)$ tiene centro trivial, al iterar este procedimiento
%obtenemos una sucesión
%\begin{equation}
%	\label{equation:Aut(G)}
%G\hookrightarrow\Aut(G)\hookrightarrow\Aut(\Aut(G))\hookrightarrow\cdots
%\end{equation}
%Como aplicación del concepto de subnormalidad veremos un teorema de Wielandt
%que afirma que la sucesión~\eqref{equation:Aut(G)} se estabiliza. 
%
%\begin{lemma}
%	\label{lemma:CG(S)=1}
%	Sea $G$ un grupo y sea $S=S_1\triangleleft
%	S_2\triangleleft\cdots\triangleleft S_r=G$.  Si $C_{S_{i+1}}(S_i)=1$ para
%	todo $i\in\{1,\dots,r-1\}$ entonces $C_G(S)=1$. 
%\end{lemma}
%
%\begin{proof}
%	Procederemos por inducción en $r$. El caso $r=2$ es trivial pues
%	$C_{G}(S)=C_{S_2}(S_1)=1$. Supongamos entonces que $r>2$. Al usar la
%	hipótesis inductiva al grupo $S_{r-1}$ obtenemos 
%	$C_G(S)\cap S_{r-1}=C_{S_{r-1}}(S)=1$.
%	Como $S_1$ es normal en $S_2$, $C_{G}(S)$ también es normal en $S_2$ pues
%	si $x\in C_G(S)$, $s_1\in S_1$, $s_2\in S_2$ entonces $s_2^{-1}s_1s_2\in
%	S_1$ y luego 
%	\[
%		[s_2xs_2^{-1},s_1]=s_2x(s_2^{-1}s_1s_2)x^{-1}s_2^{-1}s_1^{-1}=1.
%	\]
%	La normalidad de $S_{r-1}$ en $G$ implica que 
%	$[C_G(S),S_2]\subseteq C_G(S)\cap S_{r-1}=1$.
%	Luego $C_G(S)\subseteq C_G(S_2)$
%	Al usar la hipótesis inductiva en la sucesión 
%	$S_2\triangleleft\cdots\triangleleft S_r=G$, se concluye que $C_G(S)=1$.
%\end{proof}
%
%\begin{lemma}
%	\label{lemma:CG(N)=Z(N)}
%	Sea $N$ un subgrupo normal de un grupo finito $G$ tal que $C_G(N)\subseteq
%	N$. Entonces $|G|$ divide a $|Z(N)||\Aut(N)|$. En particular, $|G|$ divide
%	al factorial de $|N|$.
%\end{lemma}
%
%\begin{proof}
%	Al hacer actuar a $G$ en $N$ por conjugación obtenemos un morfismo
%	$\rho\colon G\to\Aut(N)$ con núcleo
%	\[
%	\ker\rho=\{g\in G:gng^{-1}=n\text{ para todo $n\in N$}\}=C_G(N).
%	\]
%	Como $C_G(N)\subseteq N$, $\ker\rho=C_G(N)=Z(N)$ y luego $G/Z(N)$ es
%	isomorfo a un subgrupo de $\Aut(N)$. 
%
%	Por el teorema de Lagrange, $|Z(N)|$ divide a $|N|$. Como 
%	$\Aut(N)$ actúa fielmente en el conjunto $N\setminus\{1\}$, se tiene 
%	un morfismo inyectivo $\Aut(N)\to\Sym_{|N|-1}$. Luego $|G|$ divide a $|N|!=|N|(|N|-1)!$.
%	% recordemos que actuar fielmente quiere decir que $f\cdot n=n$ para todo $n$ implica que $f=\id$. 
%\end{proof}
%
%Recordemos que si $G$ es un grupo, existe un único subgrupo normal minimal
%$G^{\infty}$ con la siguiente propiedad: el cociente $G/G^{\infty}$ es
%nilpotente.
%
%\begin{lemma}
%	\label{lemma:Ginf=Sinf}
%	Sea $G$ un grupo finito tal que $G=SF$ para algún subgrupo $S$ subnormal en
%	$G$ y algún subgrupo nilpotente $F$ normal en $G$. Entonces
%	$G^{\infty}=S^{\infty}$.
%\end{lemma}
%
%\begin{proof}
%	Sin perder generalidad podemos suponer que $S\ne G$.
%\end{proof}