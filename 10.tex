\section{02/05/2024}


\subsection{*Wielandt's automorphism tower theorem}

We now present without proof a beautiful theorem of Wielandt. 
For $G$ a finite group with trivial center, let 
$A_1=G$ and $A_{k+1}=\Aut(A_k)$ for $k\geq1$. Note that 
identifying $G$ with $\Inn(G)$, one gets a sequence
\begin{equation}
\label{eq:automorphism_groups}
A_1\subseteq A_2\subseteq A_3\subseteq\cdots 
\end{equation}
where $A_i$ is normal in $A_{i+1}$. 

\begin{definition}
    \index{Group!complete}
    A group $G$ is said to be \textbf{complete} if $Z(G)=\{1\}$ and
    $\Aut(G)=\Inn(G)$. 
\end{definition}

\begin{example}
For example, the group $\Sym_3$ is complete:
\begin{lstlisting}
gap> G := SymmetricGroup(3);;
gap> IsTrivial(Center(G));
true
gap> AutomorphismGroup(G)=InnerAutomorphismGroup(G);
true    
\end{lstlisting}
In particular, the sequence \eqref{eq:automorphism_groups}
stabilizes. 
\end{example}

\begin{example}
    Let $G=\Sym_3\times\Sym_3$. Then $|G|=36$ and $Z(G)=\{1\}$. Moreover, 
    $|\Aut(G)|=72$ and $|\Aut(\Aut(G))|=144$. Since the group
    $\Aut(\Aut(G))$ is complete, the sequence \eqref{eq:automorphism_groups} stabilizes. Let us do this 
    with the computer: 
\begin{lstlisting}
gap> G := DirectProduct(SymmetricGroup(3),SymmetricGroup(3));;
gap> A1 := G;;
gap> A2 := AutomorphismGroup(G);;
gap> A3 := AutomorphismGroup(A2);;
gap> Order(A2);
72
gap> Order(A3);
144
gap> IsTrivial(Center(A3));
true
gap> AutomorphismGroup(A3)=InnerAutomorphismGroup(A3);
true    
\end{lstlisting}
\end{example}

Let $G$ be a group and $g\in G$. Let
$\gamma_g\colon G\to G$, $x\mapsto gxg^{-1}$, denote the
conjugation map. Then 
\[
\Inn(G)=\{\gamma_g:g\in G\}
\]
is a normal subgroup of $\Aut(G)$. Moreover, $G/Z(G)\simeq\Inn(G)$. 

% Rotman 
\begin{exercise}
\label{xca:Aut}
    Let $G$ be a non-abelian simple group, $A=\Aut(G)$ and
    $I=\Inn(G)$. Prove the following statements:
    \begin{enumerate}
        \item $C_A(I)=\{\id\}$. 
        \item $f(I)=I$ for all $f\in \Aut(A)$.
        \item Every $f\in\Aut(A)$ is inner. 
        \item $\Aut(G)$ is complete. 
    \end{enumerate}    
\end{exercise}

% \begin{sol}{xca:Aut}
%     \begin{enumerate}
%     \item 
%     \end{enumerate}
% \end{sol}

The following result is known as the \textbf{Wielandt automorphism tower theorem}. 

\begin{theorem}[Wielandt]
    \index{Wielandt's automorphism tower theorem}
    \label{thm:Wielandt:automorphism}
    Let $G$ be a finite group with trivial center. Up to isomorphism, there
    are finitely many groups among the terms of the sequence \eqref{eq:automorphism_groups}.  
\end{theorem}

\begin{proof}
    See \cite[Theorem 9.10]{MR2426855}.
\end{proof}

\subsection{The transfer map}

If $H$ is a subgroup of $G$, recall that 
a \textbf{transversal} of $H$ in $G$ is a complete
set of coset representatives of $G/H$. 

\begin{lemma}
	\label{lem:d:transfer}
	Let $G$ be a group and $H$ be a subgroup of $G$ of finite index.  Let $R$
	and $S$ be transversals of $H$ in $G$ and let $\alpha\colon H\to H/[H,H]$
	be the canonical map. Then 
	\[
		d(R,S)=\prod \alpha(rs^{-1}),
	\]
	where the product is taken over all pairs 
	$(r,s)\in R\times S$ such that $Hr=Hs$,
	is well-defined and satisfies the following properties:
	\begin{enumerate}
		\item $d(R,S)^{-1}=d(S,R)$.
		\item $d(R,S)d(S,T)=d(R,T)$ for all transversal $T$ of $H$ in $G$.
		\item $d(Rg,Sg)=d(R,S)$ for all $g\in G$.
		\item $d(Rg,R)=d(Sg,S)$ for all $g\in G$.
	\end{enumerate}
\end{lemma}

\begin{proof}
	The product that defines $d(R,S)$ is well-defined since $H/[H,H]$ is 
	an abelian group. The first three claim are trivial. Let us prove
	4). By 2), 
	\[
		d(Rg,Sg)d(Sg,S)d(S,R)=d(Rg,S)d(S,R)=d(Rg,R).
	\]
	Since $H/[H,H]$ is abelian, 1) and 3) imply that 	
	\[
		d(Rg,Sg)d(Sg,S)d(S,R)=d(R,S)d(S,R)d(Sg,S)=d(Sg,S).\qedhere
	\]
\end{proof}

\begin{theorem}
	\label{thm:transfer}
	Let $G$ be a group and $H$ be a finite-index subgroup of $G$. The map 	
	\[
		\nu\colon G\to H/[H,H],\quad
		g\mapsto d(Rg,R),
	\]
	does not depend on the transversal $R$ of $H$ in $G$ and is a group
	homomorphism. 
\end{theorem}

\begin{proof}
	The previous lemma implies that the map does not depend on the transversal used. 
	Moreover, $\nu$ is a group homomorphism, as 
	\begin{align*}
		\nu(gh)&=d(R(gh),R)
		=d(R(gh),Rh)d(Rh,R)
		=d(Rg,R)d(Rh,R)=\nu(g)\nu(h).\qedhere
	\end{align*}
\end{proof}

The theorem justifies the following definition: 

\begin{definition}
\index{Transfer homomorphism}
	Let $G$ be a group and $H$ be a finite-index subgroup of $G$. The
	\textbf{transfer map} of $G$ in $H$ is the group homomorphism 
	\[
		\nu\colon G\to H/[H,H],
		\quad
		g\mapsto d(Rg,R),
	\]
	of Theorem~\ref{thm:transfer}, where $R$ is some transversal of $H$ in $G$.
\end{definition}

We need methods for computing the transfer map. If $H$ is a subgroup of 
$G$
and $(G:H)=n$, let $T=\{x_1,\dots,x_n\}$ be a transversal of $H$. For $g\in G$ let  
\[
	\nu(g)=\prod \alpha(xy^{-1}),
\]
where the product is taken over all pairs $(x,y)\in (Tg)\times T$ such that $Hx=Hy$
and $\alpha\colon H\to H/[H,H]$ is the canonical map. 
If we write 
$x=x_ig$ for some $i\in\{1,\dots,n\}$, then  
$Hx_ig=Hx_{\sigma(i)}$ for some permutation $\sigma\in\Sym_n$. Thus 
\[
	\nu(g)=\prod_{i=1}^n\alpha(x_igx_{\sigma(i)}^{-1}).
\]
The cycle structure of $\sigma$ turns out to be important. 
For example, if $\sigma=(12)(345)$ and $n=5$, then a direct calculation shows that 
\[
\prod_{i=1}^5\alpha\left(x_igx_{\sigma(i)}^{-1}\right)
=\alpha(x_1g^2x_1^{-1})\alpha(x_3g^3x_3^{-1}).
\]
This is precisely the content of the following lemma. 



% \begin{lemma}
% 	\label{lem:transfer}
% 	Let $G$ be a group and $H$ be a subgroup such that $(G:H)=n$. Let 
% 	$T$ be a transversal of $H$ in $G$. 
% 	For each $g\in G$ there exist $k$ and 
% 	positive integers 
% 	$n_1,\dots,n_k$ such that $n_1+\cdots+n_k=n$ and elements 
% 	$t_1,\dots,t_k\in T$ such that  
% 	\[
% 		\nu(g)=\prod_{i=1}^k \alpha(t_ig^{n_i}t_i^{-1}),
% 	\]
% 	where $\alpha\colon H\to H/[H,H]$ is the canonical map.
% \end{lemma}

\begin{lemma}
	\label{lem:transfer}
	Let $G$ be a group and $H$ be a subgroup of index $n$. Let 
	$T=\{t_1,\dots,t_n\}$ be a transversal of $H$ in $G$.  For each $g\in G$ there exist 
	$m\in\Z_{>0}$ and elements 
	$s_{1},\dots,s_{m}\in T$ and positive integers $n_1,\dots,n_m$
    such that 
	$s_i^{-1}g^{n_i}s_i\in H$,
	$n_1+\cdots+n_m=n$ and 
	\[
		\nu(g)=\prod_{i=1}^m \alpha(s_i^{-1}g^{n_i}s_i).
	\]
\end{lemma}

\begin{proof}
	For each $i$ there exist $h_1,\dots,h_n\in H$ and $\sigma\in\Sym_n$ such that 
	$gt_i=t_{\sigma(i)}h_i$. Write $\sigma$ as a product of disjoint cycles, say 
	\[
		\sigma=\alpha_1\cdots\alpha_m.
	\]

	Let $i\in\{1,\dots,n\}$ and write 
	$\alpha_i=(j_{1}\cdots j_{n_i})$. Since   
	\[
		g t_{j_k}=t_{\sigma(j_k)}h_{j_k}=\begin{cases}
			t_{j_1}h_{j_k} & \text{si $k=n_i$},\\
			t_{j_{k+1}}h_{j_k} & \text{otherwise},
		\end{cases}
	\]
	then 
	\begin{align*}
	t_{j_1}^{-1}g^{n_i}t_{j_1}
	&=t_{j_1}^{-1}g^{n_i-1}gt_{j_1}\\
	&=t_{j_1}^{-1}g^{n_i-1}t_{j_2}h_{j_1}\\
	&=t_{j_1}^{-1}g^{n_i-2}gt_{j_2}h_{j_1}\\
	&=t_{j_1}^{-1}g^{n_i-2}t_{j_3}h_{j_2}h_{j_1}\\
	&\phantom{=}\vdots\\
	&=t_{j_1}^{-1}gt_{j_{n_i}}h_{n_{i-1}}\cdots h_{j_2}h_{j_1}\\
	&=t_{j_1}^{-1}t_{j_1}h_{j_{n_i}}\cdots h_{j_2}h_{j_1}\in H. 	
	\end{align*}
	Thus $s_i=t_{j_1}\in T$. It only remains to note that $\nu(g)=h_1\cdots h_n$. 
\end{proof}

% \begin{proof}
% 	There exists $\sigma\in\Sym_n$ such that 
% 	\[
% 	\nu(g)=\prod_{i=1}^n \alpha( t_igt_{\sigma(i)}^{-1}). 
% 	\]
% 	Write $\sigma$ as a product of $k$ disjoint cycles
% 	$\sigma=\alpha_1\cdots\alpha_k$, where each $\alpha_j$ is a cycle of length 
% 	$n_j$. For every cycle of the form $(i_1\cdots i_{n_j})$
% 	we reorder the product in such a way that 
% 	\[
% 		\alpha(x_{i_1}gx_{i_2}^{-1})\alpha(x_{i_2}gx_{i_3}^{-1})\cdots \alpha(x_{i_{n_j}}gx_{i_1}^{-1})=\alpha(x_{i_1}g^{n_1}x_{i_1}^{-1}).
% 	\]
% 	There exist $t_1,\dots,t_k\in T$ such that 
% 	$\nu(g)=\prod_{j=1}^k \alpha(t_ig^{n_i}t_i^{-1}$). 
% \end{proof}

Gauss's lemma in number theory gives conditions for an integer to be a quadratic residue. The lemma appears in some proof of the quadratic reciprocity. Gauss's Lemma is basically a computation of the transfer homomorphism. 

\begin{exercise}[Gauss' lemma]
	\index{Gauss'!lemma}
	Let $p$ be a prime number. Let $G=\F_p^\times$ and $H=\{-1,1\}$. 
    \begin{enumerate} 
    \item Prove that the transfer homomorphism 
	\[
		\nu\colon G\to H,\quad
		\nu(x)=x^{\frac{p-1}{2}}=\legendre{x}{p}=\begin{cases}
			1 & \text{if $x$ is a square},\\
			-1 & \text{otherwise}.
		\end{cases}.
	\]
	\item For a transversal $T=\{1,2,\dots,\frac{p-1}{2}\}$ and elements $x\in G$ and $t\in T$, let 
 	\[
	\epsilon(x,t)=\begin{cases}
		1 & \text{si $xt\in T$},\\
		-1 & \text{si $xt\not\in T$}.
	\end{cases}
	\]
	Prove that  
	\[
	\legendre{x}{p}=\prod_{t\in T}\epsilon(x,t).
	\]
    \end{enumerate}
\end{exercise}

\subsection{Other applications of the transfer homomorphism}

\begin{lemma}
	\label{lem:sigma}
	Let $G$ be a group, $H$ be a finite-index subgroup and $n=(G:H)$. 
    Let $S=\{s_1,\dots,s_n\}$ and $T=\{t_1,\dots,t_n\}$ be transversals of $H$ in $G$. 
    For each $g\in G$, there exist unique $h_1,\dots,h_n\in H$ and 
	$\sigma\in\Sym_n$ such that 
	\[
		gt_i=s_{\sigma(i)}h_i,\quad
		i\in\{1,\dots,n\}.
	\]
\end{lemma}

\begin{proof}
	If $i\in\{1,\dots,n\}$, there exists  a unique $j\in\{1,\dots,n\}$ such that 
    $gt_i\in
	s_jH$. Thus $gt_i=s_jh_i$ for a unique $h_i\in H$. Thus we have constructed a 
	$\sigma\colon\{1,\dots,n\}\to\{1,\dots,n\}$, $\sigma(i)=j$.  We need to show that 
	$\sigma\in\Sym_n$. It is enough to prove that $\sigma$ is injective. If
	$\sigma(i)=\sigma(k)=j$, since $gt_i=s_jh_i$ and $gt_k=s_jh_k$, we obtain that 
	$t_i^{-1}t_k=h_i^{-1}h_k\in H$. Hence $i=k$, since $t_iH=t_kH$.
\end{proof}


\begin{theorem}
	\label{thm:P_nonabelian}
	Let $G$ be a finite group and $p$ be a prime number dividing $|[G,G]\cap
	Z(G)|$. If $P\in\Syl_p(G)$, then $P$ is non-abelian. 
\end{theorem}

\begin{proof}
	Assume that $P$ is abelian. Let $T=\{t_1,\dots,t_n\}$ be a transversal of $P$ in $G$. Since 
	$[G,G]\cap Z(G)$ is a normal subgroup of $G$, we may assume that 
	$P\cap [G,G]\cap Z(G)\ne\{1\}$. Let $z\in P\cap [G,G]\cap Z(G)\setminus\{1\}$. 
 
    Let $\nu\colon G\to P$ be the transfer homomorphism. We compute  
    $\nu(z)$ with Lemma~\ref{lem:sigma}. For $i\in\{1,\dots,n\}$, let 
    $x_1,\dots,x_n\in P$ and $\sigma\in\Sym_n$ be such that 
	$zt_i=t_{\sigma(i)}x_i$. Since $z\in Z(G)$, 
	$t_i=t_{\sigma(i)}x_iz^{-1}$. By the uniqueness of Lemma~\ref{lem:sigma}, 
	$\sigma=\id$ and $x_i=z$ for all $i$. Therefore  
	\[
	\nu(z)=z^{|T|}=z^{(G:P)}. 
	\]

	Since $P$ is abelian, $[G,G]\subseteq\ker\nu$. Thus $\nu(z)=1$, a contradiction, since 
    $1\ne z\in P$ y $z^{(G:P)}=1$ implies that $z$ has order not divisible by $p$. 
\end{proof}




Another application:

\begin{proposition}
	\label{pro:center}
	If $G$ is a group such that $Z(G)$ has finite index $n$, then
	$(gh)^n=g^nh^n$ for all $g,h\in G$.	
\end{proposition}

\begin{proof}
	Note that we may assume that $\alpha=\id$, as $Z(G)$ is
	abelian. Let $g\in G$. By Lemma~\ref{lem:transfer} there are positive integers 
    $n_1,\dots,n_k$ such that $n_1+\cdots+n_k=n$ and elements 
	$t_1,\dots,t_k$ of a transversal of $Z(G)$ in $G$ such that 
	\[
		\nu(g)=\prod_{i=1}^k t_ig^{n_1}t_i^{-1}.
	\]
	Since $g^{n_i}\in Z(G)$ for all $i\in\{1,\dots,k\}$ (as $t_ig^{n_i}t_i^{-1}\in Z(G)$), 
	it follows that 
	\[
	\nu(g)=g^{n_1+\cdots+n_k}=g^n.
	\]
	Now Theorem~\ref{thm:transfer} implies the claim.
\end{proof}

The same idea implies the following property:

\begin{exercise}
\label{xca:K_central}
	If $G$ is a group and $K$ is a central subgroup of finite index $n$, then
	$(gh)^n=g^nh^n$ for all $g,h\in G$.	
\end{exercise}

\begin{proposition}
	\label{prop:semidirecto}
	Let $G$ be a finite group and $H$ a central subgroup of index $n$, where 
	$n$ is coprime with $|H|$. Then
	$G\simeq N\rtimes H$.
\end{proposition}

\begin{proof}
	Since $H$ is abelian, $H=H/[H,H]$. Let  
	$\nu\colon G\to H$ be the transfer map and $h\in H$. 
	By Lemma~\ref{lem:transfer}, 
	\[
		\nu(h)
		=\prod_{i=1}^m s_i^{-1}h^{n_i}s_i,
	\]
	where each $s_i^{-1}h^{n_i}s_i\in H$. Since 
	$h^{n_i}\in H\subseteq Z(G)$ for all $i$, it follows that 
	$s_i^{-1}h^{n_i}s_i=h^{n_i}$ for all $i$. Thus 
	\[
		\nu(h)
		=\prod_{i=1}^m s_i^{-1}h^{n_i}s_i
		=\prod_{i=1}^mh^{n_i}
		=h^{\sum_{i=1}^m n_i}=h^n.
	\] 
	The composition $f\colon H\hookrightarrow G\xrightarrow{\nu} H$ is a group homomorphism. 
	We claim that it is an isomorphism. It is injective: If $h^n=1$, then 
	$|h|$ divides both $|H|$ and $n$. Since $n$ and $|H|$ are
	coprime, $h=1$. It is surjectice: Since $n$ and $|H|$ are coprime, there exists 
	$m\in\Z$ such that $nm\equiv 1\bmod |H|$. If $h\in H$, then $h^m\in
	H$ and $\nu(h^m)=h^{nm}=h$. 
	
	Let $N=\ker f$. We claim that $G=N\rtimes H$. 
	By definition, $N$ is normal in $G$ and $N\cap
	H=\{1\}$. To show that $G=NH$ note that 
	$|NH|=|N||H|$ and $G/N\simeq H$.
\end{proof}

\begin{exercise}
	Let $H$ be a central subgroup of a finite group $G$. If $|H|$
	and $|G/H|$ are coprime, then $G\simeq H\times G/H$.
\end{exercise}

%\begin{proof}
%	Es consecuencia inmediata del corolario~\ref{corollary:semidirecto} pues
%	$H$ es normal por ser un subgrupo central.
%\end{proof}

% TODO: Transitivity of the transfer

% serre, 7.12
We now present a nice 
application to infinite groups taken from Serre's book 
\cite[7.12]{MR3469786}. 

\begin{theorem}
	Let $G$ be a torsion-free group that contains a finite-index subgroup isomorphic to  
	$\Z$. Then $G\simeq\Z$.
\end{theorem}

\begin{proof}
	We may assume that $G$ contains a finite-index normal subgroup isomorphic to $\Z$. Indeed, 
	if $H$ is a finite-index subgroup of $G$ such that $H\simeq\Z$, then 
	$K=\cap_{x\in G}xHx^{-1}$ is a non-trivial normal subgroup of $G$ (because $K=\Core_G(H)$ and 
	$G$ has no torsion) and hence $K\simeq\Z$ (because  
	$K\subseteq H$) and $(G:K)=(G:H)(H:K)$ is finite.
	The action of $G$ on $K$ by conjugation induces a group homomorphism  
	$\epsilon\colon G\to\Aut(K)$. Since $\Aut(K)\simeq\Aut(\Z)=\{-1,1\}$, 
	there are two cases to consider.
	
	Assume first that $\epsilon=\id$. Since $K\subseteq Z(G)$, let
	$\nu\colon G\to K$ be the transfer homomorphism. By
	Proposition~\ref{pro:center} (more precisely, 
	by Exercise \ref{xca:K_central}), $\nu(g)=g^n$, where $n=(G:K)$. Since
	$G$ has no torsion, $\nu$ is injective. Thus
	$G\simeq\Z$ because it is isomorphic to a subgroup of $K$.

	Assume now that $\epsilon\ne\id$. Let $N=\ker\epsilon\ne G$. Since
	$K\simeq\Z$ is abelian, $K\subseteq N$. The result proved in the previous paragraph 
	applied to $\epsilon|_N=1$ implies that $N\simeq\Z$, as 
	$N$ contains a finite-index subgroup isomorphic to $\Z$. Let $g\in G\setminus N$. 
	Since $N$ is normal in $G$, $G$ acts by conjugation on $N$ and hence 
	there exists a group homomorphism $c_g\in\Aut(N)\simeq\{-1,1\}$. Since
	$K\subseteq N$ y $g$ acts non-trivially on $K$, 
	\[
	c_g(n)=gng^{-1}=n^{-1}
	\]
	for all $n\in N$.  Since 
	$g^2\in N$, 
	\[
		g^2=gg^2g^{-1}=g^{-2}.
	\]
	Therefore $g^4=1$, a contradiction since $g\ne1$ and $G$ has no torsion.
\end{proof}

\subsection{Dietzman's theorem}

\begin{theorem}[Dietzmann]
	\index{Dietzmann's theorem}
	\label{thm:Dietzmann} 
	Let $G$ be a group and $X\subseteq G$ be a finite subset of $G$ closed by
	conjugation. If there exists $n$ such that $x^n=1$ for all $x\in X$, then
	$\langle X\rangle$ is a finite subgroup of $G$.
\end{theorem}

\begin{proof}
	Let $S=\langle X\rangle$. Since $x^{-1}=x^{n-1}$, every element of $S$ can be 
	written as a finite product of elements of $X$. 
	Fix $x\in X$. We claim that if $x\in X$ appears $k\geq 1$ times 
	in the word $s$, then we can write $s$ as a product of $m$
	elements of $X$, where the first $k$ elements are equal to $x$. Suppose that 
	\[
	s=x_1x_2\cdots x_{t-1}xx_{t+1}\cdots x_m,
	\]
	where $x_j\ne x$ for all $j\in\{1,\dots,t-1\}$. Then 
	\[
		s=x(x^{-1}x_1x)(x^{-1}x_2x)\cdots (x^{-1}x_{t-1}x)x_{t+1}\cdots x_m
	\]
	is a product of $m$ elements of $X$ since $X$ is closed under conjugation and 
	the first element is $x$. The same argument implies that $s$
	can be written as 
	\[
		s=x^ky_{k+1}\cdots y_m,
	\]
	where each $y_j$ belongs to $X\setminus\{x\}$.

	Let $s\in S$ and write $s$ as a product of $m$ elements of 
	$X$, where $m$ is minimal. We need to show that 
	$m\leq (n-1)|X|$. 
	If $m>(n-1)|X|$, 
	then at least one $x\in X$ appears exactly $n$ 
	times in the representation of 
	$s$. Without loss of generality, we write 
	\[
		s=x^nx_{n+1}\cdots x_m=x_{n+1}\cdots x_m,
	\]
	a contradiction to the minimality of $m$. 
\end{proof}

\subsection{Schur's commutator theorem}

\begin{theorem}[Schur]
\index{Schur's!commutator theorem}
\label{thm:Schur}
	Let $G$ be a group. 
	If $Z(G)$ has finite index in $G$, then $[G,G]$ is finite.
\end{theorem}

\begin{proof}
	Let $n=(G:Z(G))$ and  
	$X$ be the set of commutators of $G$. We claim that $X$ is finite, in fact
	$|X|\leq n^2$.
	A routine calculation shows that the map 
	\[
		\varphi\colon X\to G/Z(G)\times G/Z(G),\quad [x,y]\mapsto (xZ(G),yZ(G)),
	\]
	is well-defined. It is, moreover, 
	injective: if $(xZ(G),yZ(G))=(uZ(G),vZ(G))$, then $u^{-1}x\in Z(G)$, 
	$v^{-1}y\in Z(G)$. Thus 
	\begin{align*}
		[u,v]&=uvu^{-1}v^{-1}=uv(u^{-1}x)x^{-1}v^{-1}=xvx^{-1}(v^{-1}y)y^{-1}=xyx^{-1}y^{-1}=[x,y].
	\end{align*}
	Moreover, $X$ is closed under conjugation, as 
	\[
		g[x,y]g^{-1}=[gxg^{-1},gyg^{-1}]
	\]
	for all $g,x,y\in G$. Since $G\to Z(G)$, $g\mapsto g^n$ is a group
	homomorphism, Proposition~\ref{pro:center} implies that $[x,y]^n=[x^n,y^n]=1$ for
	all $[x,y]\in X$.  The theorem follows from applying Dietzmann's theorem. 
\end{proof}

\begin{exercise}
    Let $G$ be the group with generators $a,b,c$ and 
    relations $ab=ca$, $ac=ba$ and $bc=ab$. Prove the following statements:
    \begin{enumerate}
        \item $G$ is infinite and non-abelian.
        \item $Z(G)$ has finite index in $G$ and every conjugacy class of $G$ is finite.
        \item $[G,G]$ is finite. 
        \item The subgroup $N=\langle a^3\rangle$ of $G$ 
        generated by $a^3$ is central 
        and $G/N$ is finite.
    \end{enumerate}
\end{exercise}

We conclude the section with some results similar to that of Schur. 

\begin{theorem}[Niroomand]
\index{Niroomand's theorem}
\label{thm:Niroomand}
	If the set of commutators of a group $G$ is finite, then 
	$[G,G]$ is finite.
\end{theorem}

\begin{proof}
 	Let $C=\{[x_1,y_1],\dots,[x_k,y_k]\}$ be the (finite) set of commutators of $G$ and  
	\[
    H=\langle x_1,x_2,\dots,x_k,y_1,y_2,\dots,y_k\rangle.
    \]
    Since $C$ is a set of commutators of $H$, 
	it follows that 
	$[G,G]=\langle C\rangle\subseteq [H,H]$. To simplify the notation we write 
	$H=\langle h_1,\dots,h_{2k}\rangle$. 	
 	Since $h\in Z(H)$ if and only if $h\in C_H(h_i)$ for all 
	$i\in\{1,\dots,2k\}$, we conclude that $Z(H)=C_H(h_1)\cap\cdots\cap C_H(h_{2k})$. Moreover, if 
	$h\in H$, then $hh_ih^{-1}=ch_i$ for some $c\in C$. Thus the conjugacy class of each 
	$h_i$ contains at most as many elements as $C$. This implies that 
	\[
		|H/Z(H)|=|H/\cap_{i=1}^{2k} C_H(h_i)|\leq\prod_{i=1}^{2k} (H:C_H(h_i))\leq |C|^{2k}.
	\]
	Since $H/Z(H)$ is finite, $[H,H]$ is finite. Hence  
	$[G,G]=\langle C\rangle\subseteq [H,H]$ 
	is a finite group. 
\end{proof}

\begin{theorem}[Hilton--Niroomand]
	\index{Hilton--Niroomand theorem}
	\label{thm:HiltonNiroomand}
	Let $G$ be a finitely generated group. If $[G,G]$ is finite and $G/Z(G)$ is generated by
	$n$ elements, then  
	\[
	|G/Z(G)|\leq |[G,G]|^n. 
	\]
\end{theorem}

\begin{proof}
	Assume that $G/Z(G)=\langle x_1Z(G),\dots,x_nZ(G)\rangle$. Let 
	\[
		f\colon G/Z(G)\to [G,G]\times\cdots\times [G,G],
		\quad
		y\mapsto ([x_1,y],\dots,[x_n,y]).
	\]
	Note that $f$ is well-defined: If $y\in G$ y $z\in Z(G)$, then $[x_i,y]=[x_i,yz]$ for all $i$. 
	Then $f(yz)=f(y)$.
	 
	The map $f$ is injective. Assume that $f(xZ(G))=f(yZ(G))$. Then 
	$[x_i,x]=[x_i,y]$ for all $i\in\{1,\dots,n\}$. For each $i$ we compute  
	\begin{align*}
		[x^{-1}y,x_i] &= x^{-1}[y,x_i]x[x^{-1},x_i]\\
		&=x^{-1}[y,x_i][x_i,x]x=x^{-1}[x_i,y]^{-1}[x_i,x]x=x^{-1}[x_i,y]^{-1}[x_i,y]x=1.
	\end{align*}
	This implies that $x^{-1}y\in Z(G)$. Indeed, since  
	every $g\in G$ can be written as $g=x_kz$ for some $k\in\{1,\dots,n\}$ and some $z\in Z(G)$, 
	it follows that 
    \[
    [x^{-1}y,g]=[x^{-1}y,x_kz]=[x^{-1}y,x_k]=1.
    \]
    Since $f$ is injective, 
	$|G/Z(G)|\leq |[G,G]|^n$. 
\end{proof}

\begin{exercise}
    Prove Theorem~\ref{thm:HiltonNiroomand} from Theorem~\ref{thm:Niroomand}. 
\end{exercise}


\subsection{*Units in group algebras and Gardam's example}

\index{Trivial units in group algebras}
Let $K$ be a field and $G$ be a group. A unit $u\in K[G]$ is said
to be \textbf{trivial} if $u=\lambda g$ for some $\lambda\in K\setminus\{0\}$ and
$g\in G$.	

\begin{exercise}
\label{xca:non_trivial:C2andC5}
	Prove that $\C[C_2]$ and $\C[C_5]$ have non-trivial units.
\end{exercise}

The following question is usually attributed to Kaplansky. 

\begin{question}[Units in groups algebras]
	\label{question:units}
	Let $K$ be a field and $G$ be a torsion-free group. Is it true that all units of $K[G]$ are
	trivial?
\end{question}

Question \ref{question:units} was negatively answered by Gardam. 

\begin{definition}
\index{Promislow group}
    The \textbf{Promislow group} is the group
    \[
    P=\langle a,b:a^{-1}b^2a=b^{-2},b^{-1}a^2b=a^{-2}\rangle.
    \]
\end{definition}

The Promislow group $P$ is torsion-free 
and the subgroup 
$N=\langle a^2,b^2,(ab)^2\rangle$ is normal in $P$, 
free-abelian of rank three and $P/N\simeq C_2\times C_2$.
Moreover, the map $P\to\GL_2(\Q)$ given by  
\[
a\mapsto\begin{pmatrix}
1 & 0 & 0 & 1/2\\
0 & -1 & 0 & 1/2\\
0 & 0 & -1 & 0\\
0 & 0 & 0 & 1
\end{pmatrix},
\quad
b\mapsto\begin{pmatrix}
-1 & 0 & 0 & 0\\
0 & 1 & 0 & 1/2\\
0 & 0 & -1 & 1/2\\
0 & 0 & 0 & 1
\end{pmatrix}, 
\]
is a faithful representation. These facts appear in Passman's book \cite{MR798076}. 

\begin{theorem}[Gardam]
\label{thm:Gardam_char2}
\index{Gardam's theorem}
    Let $\F_2$ be the field of two elements. Consider the elements 
    $x=a^2$, $y=b^2$ and $z=(ab)^2$ of $P$ and let 
    \begin{align*}
        &p=(1+x)(1+y)(1+z^{-1}), 
        &&q = x^{-1}y^{-1}+x+y^{-1}z+z,\\
        &r = 1+x+y^{-1}z+xyz,
        &&s=1+(x+x^{-1}+y+y^{-1})z^{-1}.
    \end{align*}
    Then $u=p+qa+rb+sab$ is a non-trivial unit in $\F_2[P]$. 
\end{theorem}

\begin{proof}
    See \cite{MR4334981}. 
\end{proof}

\begin{exercise}
\index{Murray's theorem}
     Let $p$ be a prime number and $\F_p$ be the field of size $p$. 
     Use the technique 
     for proving Gardam's theorem to prove Murray's theorem
     on the existence 
     on non-trivial units in $\F_p[P]$. 
     Reference: \href{https://arxiv.org/abs/2106.02147}{arXiv:2106.02147}. 
\end{exercise}

Gardam also constructed non-trivial units in $\C[P]$. 

%\subsection{*Kapanskly's problems}

% Let $G$ be a group and $K$ be a field. If  
% $x\in G\setminus\{1\}$ is such that $x^n=1$, then, since 
% \[
% (1-x)(1+x+x^2+\cdots+x^{n-1})=0, 
% \] 
% it follows that $K[G]$ has non-trivial zero divisors. What happens in the case
% where $G$ is torsion-free?

% \begin{example}
% 	\label{example:k[Z]}
% 	Let $G=\langle x\rangle\simeq\Z$. Then $K[G]$ has no zero divisors. 
% 	Let $\alpha,\beta\in K[G]$ be non-zero elements and write 
% 	$\alpha=\sum_{i\leq n}a_ix^i$ with $a_n\ne 0$ and $\beta=\sum_{j\leq m}b_jx^j$
% 	with $b_m\ne 0$. Since the coefficient of $x^{n+m}$ of $\alpha\beta$ is non-zero,
% 	it follows that 
% 	$\alpha\beta\ne 0$.
% \end{example}

% % \begin{exercise}
% % \label{xca:non_trivial:C5}
% % 	Prove that $\C[C_5]$ has non-trivial units. 
% % \end{exercise}

% We have already mentioned the zero divisor problems 
% for group algebras; see Conjecture~\ref{conjecture:zero}. 

% \begin{question}[Zero divisors in groups algebras]
% 	\label{question:dominio}
% 	Let $G$ be a torsion-free group. Is it true that 
% 	$K[G]$ is a domain?
% \end{question}

% We mention some intriguing problems, generally known as Kaplansky's problems. 

\subsection{*The Alperin--Kuo theorem}

% We first start stating a general result known as the transitively of the transfer map. 

% \begin{theorem}
%     Let $G$ be a group and $H\subseteq K$ be subgroups of $G$ 
%     $(G:H)$ is finite. If $U\colon G\to K$, $W\colon K\to H$ and 
%     $V\colon G\to H$ are pretransfer maps, 
%     then $V(g)=W(U(g))\bmod [H,H]$ for all $g\in G$. 
% \end{theorem}

% \begin{proof}
%     See \cite[Theorem 10.8]{MR2426855}.
% \end{proof}

\begin{theorem}[Alperin--Kuo]
\index{Alperin--Kuo theorem}
\label{thm:AlperinKuo}
Let $G$ be a finite group and $A=[G,G]\cap Z(G)$. Then $g^{(G:A)}=1$ for all $g\in G$. 
\end{theorem}

\index{Transitivity of the transfer}
One way to prove Theorem~\ref{thm:AlperinKuo} uses non-trivial properties of 
the transfer map. More precisely, the proof of the Alperin--Kuo theorem 
combines the transitivity of the transfer (see \cite[Theorem 10.8]{MR2426855}) 
with the following theorem:

\begin{theorem}[Furtwr\"angler]
\index{Furtwr\"angler's theorem}
\label{thm:Furtwrangler}
    Let $G$ be a finite group. Then the transfer homomorphism 
    $G\to G^{(1)}/G^{(2)}$ is the trivial map. 
\end{theorem}

\begin{proof}
See \cite[Theorem 10.18]{MR2426855} for a ring-theoretical proof. 
\end{proof}

\subsection{*Passman's theorem}

We now describe some other well-known problems
in the theory of group rings.

\begin{definition}
\index{Ring!reduced}
A ring $R$ is \textbf{reduced} if for all $r\in R$ such that 
$r^2=0$ one has $r=0$.
\end{definition}

Integral domains and boolean rings are reduced. The ring $\Z/8$ of integers
modulo eight 
and $M_2(\R)$ are not reduced. 

\begin{example}
    The ring over the abelian group $\Z^n$ with multiplication  \[
    (a_1,\dots,a_n)(b_1,\dots,b_n)=(a_1b_1,\dots,a_nb_n)\]
    is reduced. 
\end{example}

The structure of 
reduced rings is described by the 
Andrunakevic--Rjabuhin theorem. It states
that a ring is reduced if and only if
it is a subdirect products of domains. See
\cite[3.20.5]{MR2015465} for a proof. 

\begin{question}[Reduced group algebras]
	\label{question:reduced}
	Let $K$ be a field and $G$ be a torsion-free group. Is it true that 
	$K[G]$ is reduced? 
\end{question}

Recall that if $R$ is a unitary ring, one proves that 
the Jacobson radical $J(R)$ is 
the set of elements $x$ such that
$1+\sum_{i=1}^n r_ixs_i$ is invertible 
for all $n$ and all $r_i,s_i\in R$.

\begin{question}[Semisimple group algebras]
	\label{question:J}
	Let $K$ be a field and $G$ be a torsion-free group. It is true that 
	$J(K[G])=\{0\}$ if $G$ is non-trivial?
\end{question}

\index{Idempotent}
Recall that an element $e$ of a ring is said to be \emph{idempotent} 
if $e^2=e$. Examples of idempotents are $0$ and $1$ and 
these are known as the \textbf{trivial idempotents}. 

\begin{question}[Idempotents in group algebras]
	\label{question:idempotente}
	Let $G$ be a torsion-free group and $\alpha\in K[G]$ be an idempotent. 
	Is it true that $\alpha\in\{0,1\}$?
\end{question}

\begin{exercise}
	Prove that if $K[G]$ has no zero-divisors and $\alpha\in K[G]$ is an
	idempotent, then $\alpha\in\{0,1\}$.
\end{exercise}

\begin{exercise}
    Let $K$ be a field of characteristic two. 
	Prove that $K[C_4]$ contains non-trivial zero divisors and every
	idempotent of $K[C_4]$ is trivial. What happens if the characteristic of $K$ is not two?
\end{exercise}

For completeness, let us 
restate Conjecture~\ref{conjecture:zero} as follows:

\begin{question}[Zero divisors in group algebras]
    \label{question:zero}
 	Let $K$ be a field and 
  $G$ be a torsion-free group. Is it true that 
 	$K[G]$ is a domain?
\end{question}

Our goal is the prove
the following implications:
\[
\ref{question:J}\Longleftarrow\ref{question:units}
\Longrightarrow\ref{question:reduced}
\Longleftrightarrow\ref{question:zero}
\]

We first prove that an affirmative solution to Question~\ref{question:units} 
yields a solution to Question~\ref{question:reduced}. 

\begin{theorem}
\label{thm:units=>reduced}
    Let $K$ be a field of characteristic $\ne2$ 
	and $G$ be a non-trivial group. Assume that $K[G]$ has only trivial units.
	Then $K[G]$ is reduced. 
\end{theorem}

\begin{proof}
	Let $\alpha\in K[G]$ be such that $\alpha^2=0$. We claim that 
	$\alpha=0$. Since $\alpha^2=0$, 
	\[
		(1-\alpha)(1+\alpha)=1-\alpha^2=1, 
	\]
	it follows that $1-\alpha$ is a unit of $K[G]$. Since units of $K[G]$ are 
	trivial, there exist $\lambda\in K\setminus\{0\}$ and $g\in G$ such that 
	$1-\alpha=\lambda g$. We claim that $g=1$. If not, 
	\[
		0=\alpha^2=(1-\lambda g)^2=1-2\lambda g+\lambda^2g^2,
	\]
	a contradiction. Therefore $g=1$ and hence $\alpha=1-\lambda\in K$. Since
	$K$ is a field, one concludes that $\alpha=0$.
\end{proof}

\begin{exercise}
    What happens in Theorem \ref{thm:units=>reduced} if $K$ is a field of characteristic two?
\end{exercise}

We now prove that an affirmative solution to Question~\ref{question:units} 
also yields a solution to Question~\ref{question:J}. 

\begin{theorem}
	Let $K$ be a field and $G$ be a non-trivial group. Assume that $K[G]$ has only trivial units. 
	If $|K|>2$ or $|G|>2$, then $J(K[G])=\{0\}$.
\end{theorem}

\begin{proof}
	Let $\alpha\in J(K[G])$. There exist $\lambda\in K\setminus\{0\}$ and $g\in
	G$ such that $1-\alpha=\lambda g$. We claim that $g=1$. Assume $g\ne 1$. 
	If $|K|\geq3$,
	then there exist $\mu\in K\setminus\{0,1\}$ such that
	\[
		1-\alpha\mu=1-\mu+\lambda\mu g 
	\]
	is a non-trivial unit, a contradiction.
	If $|G|\geq3$, there exists $h\in G\setminus\{1,g^{-1}\}$ such that
	\[
        1-\alpha h=1-h+\lambda gh
    \]
    is a non-trivial unit, a contradiction.  Thus
	$g=1$ and hence $\alpha=1-\lambda\in K$. Therefore $1+\alpha h$ is a
	trivial unit for all $h\ne 1$ and hence 	$\alpha=0$.
\end{proof}

\begin{exercise}
	Prove that if $G=\langle g\rangle\simeq\Z/2$, then 
	$J(\F_2[G])=\{0,g-1\}\ne\{0\}$. 
\end{exercise}


We now want to prove that an affirmative answer to 
Question~\ref{question:reduced} yields an affirmative answer to Question~\ref{question:zero}. We first need some preliminaries. 

\begin{proposition}
	\label{pro:FCabeliano}
	If $G$ is a torsion-free group such 
	that $\Delta(G)=G$, then $G$ is abelian.
\end{proposition}

\begin{proof}
	Let $x,y\in G=\Delta(G)$ and $S=\langle x,y\rangle$. The group $Z(S)=C_S(x)\cap C_S(y)$ has 
	finite index, say $n$, in $S$. By Proposition~\ref{pro:center}, 
	the map $S\to Z(S)$, $s\mapsto s^n$, is a group homomorphism. Thus  
	\[
		[x,y]^n=(xyx^{-1}y^{-1})^n=x^ny^nx^{-n}y^{-n}=1
	\]
	as $x^n\in Z(S)$. Since $G$ is torsion-free, $[x,y]=1$.
\end{proof}

\begin{lemma}[Neumann]
	\index{Neumann's!lemma}
	\label{lem:Neumann}
	Let $H_1,\dots,H_m$ be subgroups of $G$. 
	Assume there are finitely many elements
	$a_{ij}\in G$, $1\leq i\leq m$, $1\leq j\leq n$, such that 
	\[
		G=\bigcup_{i=1}^m\bigcup_{j=1}^n H_ia_{ij}.
	\]
	Then some $H_i$ has finite index in $G$.
\end{lemma}

\begin{proof}
	We proceed by induction on $m$. The case $m=1$ is trivial. 
	Let us assume that $m\geq2$. If $(G:H_1)=\infty$, there exists $b\in G$
	such that 
	\[
		H_1b\cap\left(
	\bigcup_{j=1}^nH_1a_{1j}\right)=\emptyset.
	\]
	Since $H_1b\subseteq\bigcup_{i=2}^m\bigcup_{j=1}^n H_ia_{ij}$, 
	it follows that 
	\[
		H_1a_{1k}\subseteq \bigcup_{i=2}^m\bigcup_{j=1}^n H_1a_{ij}b^{-1}a_{1k}.
	\]
	Hence $G$ can be covered by finitely many cosets of $H_2,\dots,H_m$. By the inductive hypothesis, 
	some of these $H_j$ has finite index in $G$.
\end{proof}

We now consider a projection operator of group algebras. If $G$ 
is a group and $H$ is a subgroup of $G$, let 
\[
	\pi_H\colon K[G]\to K[H],\quad
	\pi_H\left(\sum_{g\in G}\lambda_gg\right)=\sum_{g\in H}\lambda_gg.
\]

If $R$ and $S$ are rings, a $(R,S)$-bimodule is an abelian group
$M$ that is both a left $R$-module and a right $S$-module 
and the compatibility condition 
\[
(rm)s = r(ms)
\]
holds for all $r\in R$, $s\in S$ and $m\in M$.

\begin{exercise}
	Let $G$ be a group and $H$ be a subgroup of $G$. Prove that
	if $\alpha\in
	K[G]$, then $\pi_H$ is a $(K[H],K[H])$-bimodule homomorphism
	with usual left and right multiplications,
	\[
		\pi_H(\beta\alpha\gamma)=\beta\pi_H(\alpha)\gamma
	\]
	for all $\beta,\gamma\in K[H]$.
\end{exercise}

%\begin{proof}
%	Supongamos que $\alpha=\sum_{g\in G}\lambda_gg=\alpha_1+\alpha_2$, donde
%	$\alpha_1=\sum_{g\not\in H}\lambda_gg$ y $\alpha_2=\sum_{g\in
%	H}\lambda_gg=\pi_H(\alpha)$. Entonces
%	$\beta\alpha\gamma=\beta\alpha_1\gamma+\beta\alpha_2\gamma$, donde
%	$\beta\alpha_1\gamma\not\in K[H]$ y $\beta\alpha_2\gamma\in K[H]$.
%\end{proof}

\begin{lemma}
	\label{lem:escritura}
	Let $X$ be a left transversal of $H$ in $G$. Every $\alpha\in K[G]$ can be written
	uniquely as 
	\[
	\alpha=\sum_{x\in X}x\alpha_x,
	\]
	where $\alpha_x=\pi_H(x^{-1}\alpha)\in K[H]$.
\end{lemma}

\begin{proof}
	Let $\alpha\in K[G]$. Since $\supp\alpha$ is finite, $\supp\alpha$ is contained 
    in finitely many cosets of $H$, say $x_1H,\dots,x_nH$, where each 
	$x_j$ belongs to $X$. Write $\alpha=\alpha_1+\cdots+\alpha_n$,
	where $\alpha_i=\sum_{g\in x_iH}\lambda_gg$. If $g\in x_iH$, then 
	$x_i^{-1}g\in H$ and hence 
	\[
		\alpha=\sum_{i=1}^n x_i(x_i^{-1}\alpha_i)=\sum_{x\in X}x\alpha_x
	\]
	with $\alpha_x\in K[H]$ for all $x\in X$. For the uniqueness, note that 
	for each  $x\in X$ the previous exercise implies that  
	\[
		\pi_H(x^{-1}\alpha)
		=\pi_H\left(\sum_{y\in X}x^{-1}y\alpha_y\right)
		=\sum_{y\in X}\pi_H(x^{-1}y)\alpha_y=\alpha_x, 
	\]
	as  
	\[
		\pi_H(x^{-1}y)=\begin{cases}
		1 & \text{si $x=y$},\\
		0 & \text{si $x\ne y$}.
	\end{cases}\qedhere 
	\]
\end{proof}

\begin{lemma}
	\label{lem:ideal_pi}
	Let $G$ be a group and $H$ be a subgroup of $G$. If $I$ is a non-zero 
	left ideal
	of $K[G]$, then  $\pi_H(I)\ne\{0\}$.
\end{lemma}

\begin{proof}
	Let $X$ be a left transversal of $H$ in $G$ and $\alpha\in I\setminus\{0\}$. By Lemma
	\ref{lem:escritura} we can write $\alpha=\sum_{x\in
	X}x\alpha_x$ with $\alpha_x=\pi_H(x^{-1}\alpha)\in K[H]$ for all $x\in X$.
	Since $\alpha\ne0$, there exists $y\in X$ such that $0\ne
	\alpha_y=\pi_H(y^{-1}\alpha)\in\pi_H(I)$ ($y^{-1}\alpha\in I$ since $I$ is 
    a left ideal).
\end{proof}

\begin{exercise}
	Let $G$ be a group, $H$ be a subgroup of $G$ and $\alpha\in K[H]$. The following statements hold:
	\begin{enumerate}
		\item $\alpha$ is invertible in $K[H]$ if and only if $\alpha$ is
			invertible in $K[G]$.
		\item $\alpha$ is a zero divisor of $K[H]$ if and only if $\alpha$ is  
			a zero divisor of $K[G]$.
	\end{enumerate}
\end{exercise}

% \begin{sol}
% 	If $\alpha$ is invertible in $K[G]$, there exists $\beta\in K[G]$ such that 
% 	$\alpha\beta=\beta\alpha=1$. Apply $\pi_H$ and use that $\pi_H$ 
% 	is a $(K[H],K[H])$-bimodule homomorphism to obtain  
% 	\[
% 		\alpha\pi_H(\beta)=\pi_H(\alpha\beta)=\pi_H(1)=1=\pi_H(1)=\pi_H(\beta\alpha)=\pi_H(\beta)\alpha.
% 	\]
	
% 	Assume now that $\alpha\beta=0$ for some $\beta\in K[G]\setminus\{0\}$. Let $g\in G$
% 	be such that $1\in\supp(\beta g)$. Since $\alpha(\beta g)=0$, 
% 	\[
% 		0=\pi_H(0)=\pi_H(\alpha(\beta g))=\alpha\pi_H(\beta g),
% 	\]
% 	where $\pi_H(\beta g)\in K[H]\setminus\{0\}$, as $1\in\supp(\beta g)$. 
% \end{sol}

\begin{lemma}[Passman]
	\index{Passman's!lemma}
	\label{lem:Passman}
	Let $G$ be a group and 
	$\gamma_1,\gamma_2\in K[G]$ be such that $\gamma_1K[G]\gamma_2=\{0\}$.
	Then $\pi_{\Delta(G)}(\gamma_1)\pi_{\Delta(G)}(\gamma_2)=\{0\}$.
\end{lemma}

\begin{proof}
	It is enough to show that $\pi_{\Delta(G)}(\gamma_1)\gamma_2=\{0\}$, 
	as in this case
	\[
		\{0\}=\pi_{\Delta(G)}(\pi_{\Delta(G)}(\gamma_1)\gamma_2)=\pi_{\Delta}(\gamma_1)\pi_{\Delta(G)}(\gamma_2).
	\]
	Write $\gamma_1=\alpha_1+\beta_1$, where 
	\begin{align*}
		&\alpha_1=a_1u_1+\cdots+a_ru_r, && u_1,\dots,u_r\in\Delta(G),\\
		&\beta_1=b_1v_1+\cdots+b_sv_s, && v_1,\dots,v_s\not\in\Delta(G),\\
		&\gamma_2=c_1w_1+\cdots+c_tw_t,&& w_1,\dots,w_t\in G.
	\end{align*}
	The subgroup $C=\bigcap_{i=1}^rC_G(u_i)$ has finite index in $G$.
	Assume that 
	\[
		0\ne \pi_{\Delta}(\gamma_1)\gamma_2=\alpha_1\gamma_2. 
	\]
	Let $g\in\supp(\alpha_1\gamma_2)$. 
	If $v_i$ is a conjugate in $G$ of some 
	$gw_j^{-1}$, let $g_{ij}\in G$ be such that
	$g_{ij}^{-1}v_ig_{ij}=gw_j^{-1}$. If $v_i$ and $gw_j^{-1}$ 
	are not conjugate, 
	we take $g_{ij}=1$. 

	For every $x\in C$ it follows that
	$\alpha_1\gamma_2=(x^{-1}\alpha_1x)\gamma_2$. Since  
	\[
		x^{-1}\gamma_1x\gamma_2\in x^{-1}\gamma_1K[G]\gamma_2=0,
	\]
	it follows that
	\begin{align*}
		(a_1u_1+\cdots+a_ru_r)\gamma_2&=
		\alpha_1\gamma_2=x^{-1}\alpha_1x\gamma_2=-x^{-1}\beta_1x\gamma_2\\
		&=-x^{-1}(b_1v_1+\cdots+b_sv_r)x(c_1w_1+\cdots+c_tw_t).
	\end{align*}
	Now $g\in\supp(\alpha_1\gamma_2)$ implies that there exist $i,j$ such that
	$g=x^{-1}v_ixw_j$.
	Thus $v_i$ and $gw_j^{-1}$ are conjugate and hence
	$x^{-1}v_ix=gw_j^{-1}=g_{ij}^{-1}v_ig_{ij}$, that is
	$x\in C_G(v_i)g_{ij}$. This proves that 
	\[
		C\subseteq\bigcup_{i,j}C_G(v_i)g_{ij}. 
	\]
	Since $C$ has finite index in $G$, it follows that 
	$G$ can be covered by finitely many cosets of 
	the $C_G(v_i)$. Every $v_i\not\in\Delta(G)$, so 
	each $C_G(v_i)$ has infinite index in $G$, a contradiction 
	to Neumann's lemma.
\end{proof}


\begin{exercise}
    Let $K$ be a field and $G$ be a torsion-free abelian group. Prove that 
    $K[G]$ has no non-zero divisors. 
\end{exercise}


\begin{theorem}[Passman]
\index{Passman's!theorem}
\label{thm:Passman}
	Let $K$ be a field and $G$ be a torsion-free group. If 
	$K[G]$ is reduced, then $K[G]$ is a domain.
\end{theorem}

\begin{proof}
	Assume that $K[G]$ is not a domain. Let $\gamma_1,\gamma_2\in K[G]\setminus\{0\}$
	be such that $\gamma_2\gamma_1=0$. If $\alpha\in K[G]$, then
	\[
		(\gamma_1\alpha\gamma_2)^2=\gamma_1\alpha\gamma_2\gamma_1\alpha\gamma_2=0
	\]
	and thus $\gamma_1\alpha\gamma_2=0$, as $K[G]$ is reduced. In particular, 
	$\gamma_1K[G]\gamma_2=\{0\}$. Let $I$ be the left ideal of $K[G]$ generated 
	by $\gamma_2$. Since $I\ne\{0\}$, it follows
	from Lemma~\ref{lem:ideal_pi} that 
	$\pi_{\Delta(G)}(I)\ne\{0\}$. Hence 
	$\pi_{\Delta(G)}(\beta\gamma_2)\ne\{ 0\}$ for some $\beta\in K[G]$. 
	Similarly, 
	$\pi_{\Delta(G)}(\gamma_1\alpha)\ne\{ 0\}$ for some $\alpha\in K[G]$. Since 
	\[
		\gamma_1\alpha K[G]\beta\gamma_2\subseteq \gamma_1 K[G]\gamma_2=\{0\},
	\]
    it follows that $\pi_{\Delta(G)}(\gamma_1\alpha)\pi_{\Delta(G)}(\beta\gamma_2)=\{0\}$
    by Passman's lemma. Hence $K[\Delta(G)]$ has zero divisors, a contradictions
    since $\Delta(G)$ is an abelian group.
\end{proof}



\subsection{*The isomorphism problem for group algebras}

If $R$ is a commutative ring (with 1) 
and $G$ is a group, then one defines the group ring $R[G]$. More precisely,
$R[G]$ is the set of finite linear combinations
\[
    \sum_{g\in G}\lambda_gg
\]
where $\lambda_g\in R$ and $\lambda_g=0$ for all but finitely many $g\in G$. 
One easily proves that $R[G]$ is a ring with 
addition
\[
\left(\sum_{g\in G}\lambda_gg\right)+\left(\sum_{g\in G}\mu_gg\right)
=\sum_{g\in G}(\lambda_g+\mu_g)(g)
\]
and multiplication 
\[
\left(\sum_{g\in G}\lambda_gg\right)\left(\sum_{h\in G}\mu_hh\right)
=\sum_{g,h\in G}\lambda_g\mu_h(gh).
\]
Moreover, $R[G]$ is a left $R$-module with
\[
\lambda(\sum_{g\in G}\lambda_gg)=\sum_{g\in G}(\lambda\lambda_g)g.
\]

\begin{exercise}
    Let $G$ be a group. Prove that if 
    $\Z[G]\simeq\Z[H]$, then $R[G]\simeq R[H]$ for any commutative ring $R$.      
\end{exercise}

\begin{question}[The isomorphism problem]
\index{Isomorphism problem}
\label{question:IP}
    Let $G$ and $H$ be groups. Does $\Z[G]\simeq\Z[H]$ imply $G\simeq H$?
\end{question}   

Although there are several cases where 
the isomorphism problem has an affirmative answer (e.g. abelian groups, 
metabelian groups, nilpotent groups, nilpotent-by-abelian groups, simple groups, 
abelian-by-nilpotent groups), it is false in general. In~\cite{MR1847590},    
Hertweck found a counterexample of order $2^{21}97^{28}$.

\begin{question}[The modular isomorphism problem]
\label{question:MIP}
\index{Modular isomorphism problem}
    Let $p$ be a prime number. Let 
    $G$ and $H$ be finite $p$-groups and let $K$ be a field of characteristic $p$. 
    Does $K[G]\simeq K[H]$ imply $G\simeq H$?
\end{question}   

Question \ref{question:MIP} has an affirmative answer in several cases. However, 
this is not true in general. This question was recently answered by Garc\'ia, Margolis and
del R\'io \cite{MR4373245}. They found two non-isomorphic groups $G$ and $H$ both of order $512$ 
such that $K[G]\simeq K[H]$ for all field $K$ 
of characteristic two. 
