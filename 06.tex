\section{21/03/2024}

\subsection{Super-solvable groups}

\begin{definition}
\index{Group!super-solvable}
A group $G$ is said to be \textbf{super-solvable} if there exists a sequence 
\[
G=G_0\supseteq G_1\supseteq\cdots\supseteq G_n=\{1\}
\]
of normal subgroups of $G$ such that every 
quotient $G_{i-1}/G_i$ is cyclic. 
\end{definition}

In the previous definition, we do not require the group to be finite. Hence the quotients 
could be finite cyclic groups or isomorphic to $\Z$. 

\begin{example}
The dihedral group $\D_{n}$ of order $2n$ is super-solvable, as 
\[	
\D_{n}\supseteq \langle
r\rangle\supseteq \{1\}
\]
is a sequence of normal subgroups with cyclic factors. 
\end{example}

Every super-solvable group is solvable. See Exercise~\ref{xca:solvable}.

\begin{example}
The alternating group $\Alt_4$ solvable but not super-solvable. The only 
proper non-trivial normal subgroup of $\Alt_4$ is 
	\[
	\{\id,(12)(34),(13)(24),(14)(23)\}\simeq C_2\times C_2.
	\]
Thus $\Alt_4$ does not have a sequence of normal subgroups 
with cyclic factors. 
\end{example}

% \begin{exercise}
% \label{xca:Aff_supersolvable}
% Prove that $\Aff(\Z)$ is super-solvable. 
% \end{exercise}
% % aff(Z) es súper-resoluble

\begin{example}
The group $\SL_2(3)$ is solvable but not super-solvable. Here is a computer verification: 
\begin{lstlisting}
gap> IsSolvable(SL(2,3));
true
gap> IsSupersolvable(SL(2,3));
false
\end{lstlisting}
\end{example}

\begin{exercise}
\label{xca:super}
Prove the following statements: 
\begin{enumerate}
\item Every subgroup of a super-solvable group is super-solvable. 
\item Quotients of super-solvable groups are super-solvable. 
\end{enumerate}
\end{exercise}

% \begin{svgraybox}
% 	Sea $G$ un grupo súper-resoluble y sea 			
% 	\[ 
% 	G=G_0\supseteq G_1\supseteq \cdots\supseteq G_n=1 
% 	\] 
% 	una sucesión de subgrupos normales
% 	donde cada cociente $G_{i-1}/G_{i}$ es cíclico. 
% 	\begin{enumerate}
% 		\item Sea $H$ un subgrupo de $G$. Como $G$ es
% 			súper-resoluble, Sea 
% 			\[
% 			H=H\cap G_0\supseteq H\cap G_1\supseteq\cdots\supseteq H\cap G_n=1
% 			\]
% 			una sucesión de subgrupos de $H$. Cada $H\cap G_i$ es normal en $H$
% 			pues $G_i$ es normal en $G$. Fijemos $i\in\{1,\dots,n\}$ y sea
% 			$\pi_{i-1}\colon G_{i-1}\to G_{i-1}/G_{i}$ el morfismo canónico. La
% 			restricción de $\pi_{i-1}$ al subgrupo $H\cap G_{i-1}$ es un morfismo con
% 			núcleo $G_{i}\cap H$.  Al usar el teorema de isomorfismos vemos que 
% 			\[
% 			\frac{H\cap G_{i-1}}{H\cap G_{i}}\simeq \pi_{i-1}(H\cap G_i)\subseteq G_{i-1}/G_i
% 			\]
% 			es un grupo cíclico por ser subgrupo de un grupo cíclico. 
% 		\item Sea $K$ un subgrupo normal de $G$ y sea $\pi\colon G\to G/K$ el
% 			morfismo canónico. Para cada $i$ sea $Q_i=\pi(G_i)$. Cada $Q_i$ es
% 			normal en $Q_n=\pi(G_n)=G/K$ pues $G_i$ es normal en $G$. Como
% 			$G_{i-1}K=G_{i-1}(G_iK)$ para todo $i$, 
% 			el grupo
% 			\begin{align*}
% 			Q_{i-1}/Q_i
% 			&\simeq\frac{G_{i-1}/G_{i-1}\cap K}{G_i/G_i\cap K}
% 			\simeq \frac{G_{i-1}K/K}{G_{i}K/K}\\
% 			&\simeq\frac{ G_{i-1}K}{G_iK}
% 			\simeq\frac{ G_{i-1}(G_iK)}{G_iK}
% 			\simeq\frac{ G_{i-1}}{G_iK\cap G_{i-1}}
% 			\simeq\frac{ G_{i-1}/G_i}{G_iK\cap G_{i-1}/G_i}
% 			\end{align*}
% 			es cíclico por ser un cociente de un grupo cíclico.
% 	\end{enumerate}
% \end{svgraybox}

\begin{exercise}
\label{xca:directosuper}
Prove that the direct product of super-solvable groups is super-solvable. 
\end{exercise}

% \begin{svgraybox}
% 	Supongamos que $G$ admite una sucesión $G=G_0\supseteq G_1\supseteq
% 	\cdots\supseteq G_n=1$ de de subgrupos normales tales que cada cociente
% 	$G_{i-1}/G_i$ es cíclico, y que $H$ admite una sucesión $H=H_0\supseteq
% 	H_1\supseteq \cdots\supseteq H_m=1$ de subgrupos normales donde cada
% 	$H_{i-1}/H_i$ es cíclico. Consideramos la sucesión 
% 	\[
% 		1=G_0\times H_0\supseteq G_1\times H_0\supseteq\cdots\supseteq G_n\times H_0\supseteq G_n\times H_1\supseteq \cdots\supseteq G_n\times H_m=G\times H
% 	\]
% 	tiene factores cíclicos pues 
% 	cada $G_{i-1}\times H_0/G_i\times H_0\simeq G_{i-1}/G_i$ es cíclico y cada 
% 	$G_n\times H_{j-1}/G_n\times H_j$ también pues
% 	\[
% 	G_n\times H_{j-1}/G_n\times H_j
% 	\simeq \frac{GH_{j-1}/G}{GH_j/G}
% 	\simeq \frac{H_{j-1}/H_{j-1}\cap G}{H_j/H_j\cap G}\simeq H_{j-1}/H_j.
% 	\]
% \end{svgraybox}

\begin{exercise}
\label{xca:super_quotient}
Let $H$ and $K$ be normal subgroups of a group $G$ such that $G/K$ and $G/H$
are super-solvable. Prove that $G/H\cap K$ is super-solvable. 
\end{exercise}

% \begin{svgraybox}
% 	El producto directo $G/H\times G/K$ es súper-resoluble. Sea $\partial\colon
% 	G\to G/H\times G/K$, $g\mapsto (gH,gK)$.  Como $\ker\partial=H\cap K$, se
% 	tiene que $G/H\cap K\simeq\partial(G)$, que es súper-resoluble por ser un
% 	subgrupo de un grupo súper-resoluble.
% \end{svgraybox}

\begin{exercise}
\label{xca:Nciclico}
Let $N$ be a cyclic normal subgroup of $G$. If $G/N$ is super-solvable, then 
$G$ is super-solvable. 
\end{exercise}

% todo: arreglar 

% \begin{proof}
% 	Sea $\pi\colon G\to G/N$ el morfismo canónico y sea $Q=G/N$. Como $Q$ es
% 	súper-resoluble, tenemos una sucesión
% 	\[
% 		Q=Q_0\supseteq Q_1\supseteq \cdots\supseteq Q_n=\{1\}
% 	\]
% 	de subgrupos normales de $Q$ tales que cada cociente $Q_{i-1}/Q_i$ es
% 	cíclico. Cada elemento de la sucesión
% 	\[
% 	G=\pi^{-1}(Q)\supseteq\pi^{-1}(Q_1)\supseteq\cdots\supseteq \pi^{-1}(Q_n)=N\supseteq \{1\}
% 	\]
% 	es normal en $G$ (por la correspondencia) y dejamos como 
% 	ejercicio demostrar que cada cociente es cíclico. 
% % 	cada cociente es cíclico $N$ es cíclico. 
% % 	Queda como ejercicio demostrar 
% % 	y cada 
% % 	\[
% % 	\frac{\pi^{-1}(Q_j)}{\pi^{-1}(Q_{j+1})}
% % 		=\frac{Q_jN}{Q_{j+1}N}
% % 		\simeq\frac{Q_jN/N}{Q_{j+1}N/N}
% % 		\simeq\frac{Q_j(Q_{j+1}N)}{Q_{j+1}N}
% % 		\simeq\frac{Q_j/Q_{j+1}}{Q_{j+1}N\cap Q_j}
% % 	\]
% % 	es cíclico por ser cociente de un grupo cíclico.
% \end{proof}

\begin{theorem}
\label{thm:ZorCp}
Let $G$ be a super-solvable non-trivial group. Then $G$ admits a sequence 
\[
G=G_0\supseteq G_1\supseteq\cdots\supseteq G_n=\{1\}
\]
of normal subgroups 
such that every quotient $G_{i-1}/G_i$ is cyclic of prime order or isomorphic to 
$\Z$.
\end{theorem}

\begin{proof}
Let $G=G_0\supseteq G_1\supseteq\cdots\supseteq G_n=\{1\}$ be a sequence of normal subgroups
of $G$ such that every quotient $G_{i-1}/G_i$ is cyclic. Let 
$i\in\{1,\dots,n\}$ be such that $G_{i-1}/G_i\simeq C_n$ for some non-prime  
$n$ and let $\pi\colon G_{i-1}\to G_{i-1}/G_i$ be the canonical map. 
Let $p$ be a prime divisor of $n$ and $H$ be a subgroup of $G$ such that 
$\pi(H)$ is a subgroup of $G_{i-1}/G_i$ of order $p$. By the correspondence theorem, 
$G_{i}\subseteq H\subseteq G_{i-1}$. 

We claim that $H$ is normal in $G$. Let $g\in G$. Since $\pi(gHg^{-1})$ is a subgroup of order $p$ of 
the cyclic group $G_{i-1}/G_i$, $\pi(gHg^{-1})=\pi(H)$. Then 
$gHg^{-1}=G_{i}H\subseteq H$ and hence $gHg^{-1}=H$. 
% 	\[
% 	\frac{gHg^{-1}}{G_i}=\frac{G_{i}H}{G_{i}}\simeq \frac{H}{G_i\cap H}=\frac{H}{G_i}
% 	\]
% 	y entonces $gHg^{-1}\subseteq H$.  

Note that $H/G_i$ is cyclic of prime order, as 
\[
H/G_i=H/H\cap G_i\simeq \pi(H)\simeq C_p. 
\]
Moreover, $G_{i-1}/H$ is cyclic, as 
\[
G_{i-1}/H\simeq\frac{G_{i-1}/G_i}{H/G_i}
\]
is the quotient of a cyclic group. 
	
We have shown that by adding $H$ to our sequence of normal subgroups, 
we obtain a sequence with cyclic factors where 
$H/G_{i}$ is cylic of prime order. Repeating this procedure, we obtain the desired result. 
\end{proof}

Let us discuss an immediate application. 

\begin{corollary}
A finite super-solvable group admits a sequence 
of normal subgroups where each quotient is cyclic of prime order. 
\end{corollary}

% \begin{proof}
% 	Es consecuencia inmediata del teorema~\ref{theorem:ZorCp}.
% \end{proof}

We now discuss other properties of super-solvable groups. 

\begin{theorem}
\label{thm:super_structure}
Let $G$ be a super-solvable group. The following statement hold:  
\begin{enumerate}
\item If $N$ is minimal normal in $G$, then $N\simeq C_p$ for some prime number $p$.
\item If $M$ is maximal in $G$, then $(G:M)=p$ for some prime number $p$.
\end{enumerate}
\end{theorem}

\begin{proof}
Let us prove the first claim. Since $G$ is super-solvable, there exists a sequence 
\[
G=G_0\supseteq G_1\supseteq
G_2\supseteq\cdots\supseteq G_n=\{1\}
\]
of normal subgroups with cyclic factors. Since 
each $G_i\cap N$ is a normal subgroup of $G$ contained in $N$, 
the minimality implies that 
each $G_i\cap N$ is either trivial or equal to $N$. Moreover, $N\cap G_0=N$ and $N\cap
G_n=\{1\}$. Let $j$ be the smallest positive integer such that $N\cap G_j=\{1\}$. 
Since $N\subseteq G_{j-1}$ (because $N\cap G_{j-1}=N$), the composition 
	\[
	N\hookrightarrow G_{j-1}\to G_{j-1}/G_j
	\]
is an injective group homomorphism, as its kernel is equal $N\cap G_{j}=\{1\}$. 
Thus $N$ is cyclic, as it is isomorphic to a subgroup of the cyclic group $G_{i-1}/G_i$. 
If $G_{i-1}/G_i\simeq\Z$, then $N\simeq\Z$, a contradiction to the fact that $N$ is minimal normal. (For example, 
$2\Z$ is characteristic subgroup of $\Z$ and hence it is normal in $G$. Thus $N$ is cyclic and finite. Hence $N\simeq C_p$.)

We now prove the second claim. Let $M$ be a maximal subgroup of $G$. If $M$ is normal in $G$, 
then $G/M$ does not contain non-trivial proper subgroups. Then 
$G/M\simeq C_p$ for some prime number $p$. Assume that $M$ is not normal in $G$. 
Let $H=\cap_{g\in G}gMg^{-1}$ and $\pi\colon G\to G/H$ be the canonical map.  
Since $\pi(M)$ is maximal in 
	$\pi(G)=G/H$ and 
	\[
		(G:M)=(G/H:M/H)=(G/H:M/H\cap M)=(\pi(G):\pi(M)),
	\]
we may assume that $M$ does not contain non-trivial normal subgroups of $G$ (if needed, 
we just replace $G$ by $G/H$). Since $G$ is super-solvable, there exists a sequence 
$G=G_0\supseteq G_1\supseteq\cdots\supseteq G_n=\{1\}$ of normal subgroups of $G$ 
with factors either cyclic of prime order or isomorphic to $\Z$. Let 
$N=G_{n-1}$. Since $N$ is cyclic, every subgroup of $N$ is characteristic 
in $N$ and hence normal in $G$. In particular, $M\cap N$ is normal in 
$G$ and therefore $M\cap N=\{1\}$. Since $M\subseteq
NM\subseteq G$, the maximality of $M$ implies that either $M=NM$ or $G=NM$.
Since $N\subseteq NM=M$ yields a contradiction, we conclude that $G=NM$.

If $N\simeq C_p$ for some prime $p$, then $(G:M)=p$ and the proof is complete. 
Assume that $N\simeq\Z$. Let $H$ be a proper non-trivial subgroup of $N$. Since 
$H$ is characteristic in $N$, $H$ is normal in $G$. Since 
$M\subseteq HM\subseteq NM=G$, the maximality of $M$ implies that either $HM=M$ or 
$HM=G$. Since $HM=M$ implies $H\subseteq M\cap N=\{1\}$,
we may assume that $HM=G$. If $n\in N\setminus H$, then $n=hm$ for some 
$h\in H$ and $m\in M$. Then $h=n$, as $h^{-1}n\in N\cap M=\{1\}$, a contradiction. 
% We now prove the third claim. Since $G$ is super-solvable, 
% there exists a sequence
% 	\[
% 	G=G_0\supseteq G_1\supseteq\cdots\supseteq G_n=\{1\}
% 	\]
% of normal subgroups of $G$ such that each 
% $G_i/G_{i+1}$ is cyclic. For  
% 	$i\in\{0,\dots,n\}$, let $H_i=[G,G]\cap G_i$. Since $[G,G]$ the each 
%  $G_i$ are normal in $G$, one obtains a sequence 
% 	\[
% 	[G,G]=H_0\supseteq H_1\supseteq\cdots\supseteq H_n=\{1\}
% 	\]
% of normal subgroups of $G$. Since $H_i$ and $H_{i+1}$ are normal in $G$, 
% the group $G$ acts by conjugation on $H_i/H_{i+1}$. Thus there exists a group
% homomorphism 
% 	$\gamma\colon G\to\Aut(H_i/H_{i+1})$. Since $H_i/H_{i+1}$ is cyclic, 
% 	$\Aut(H_i/H_{i+1})$ is abelian. Thus $[G,G]\subseteq\ker \gamma$. Therefore 
% 	$[G,G]$ acts trivially by conjugation on $H_{i}/H_{i+1}$. Hence 
%  	\[
% 	H_i/H_{i+1}\subseteq Z([G,G]/H_{i+1}).
% 	\]

%  Finally, we prove the fourth claim. Since $G$ is non-abelian,
% 	$Z(G)\ne G$. Let $\pi\colon G\to G/Z(G)$ be the canonical map. The group 
% 	$G/Z(G)$ is super-solvable and the sequence 
% 	\[
% 	G/Z(G)=\pi(G)\supseteq \pi(G_1)\supseteq\cdots\supseteq \pi(1)=\{1\}
% 	\]
% is a sequence of normal subgroups of $G/Z(G)$ with cyclic quotients. 
% In particular, $1\ne \pi(G_1)$ is a proper normal subgroup of $G/Z(G)$. By the correspondence theorem, $\pi^{-1}(\pi(G_1))\ne G$ is a normal subgroup of 
% $G$ properly containing $Z(G)$. 
\end{proof}

\begin{exercise}
    Let $G$ be a super-solvable group. Prove the following statements:
    \begin{enumerate}
        \item The commutator subgroup $[G,G]$ is nilpotent. 
        \item If $G$ is non-abelian, there exists a normal subgroup $N\ne G$ such that
	$Z(G)\subsetneq N$.
    \end{enumerate}
\end{exercise}

There are solvable groups with a non-nilpotent derived subgroup. 

\begin{example}
The group $\Sym_4$ is solvable and $[\Sym_4,\Sym_4]=\Alt_4$ is not nilpotent.
\end{example}

\begin{proposition}
\label{pro:psuper}
Let $p$ be a prime number. Every finite $p$-group is super-solvable.
\end{proposition}

\begin{proof}
Let $G$ be a minimal counterexample. We may assume that $|G|=p^n$ for some 
$n>1$ (otherwise, if $n=1$, then $G$ is trivially super-solvable). 
The group $G$ is nilpotent and contains a normal subgroup $N$ of order $p$. 
Moreover, since $|G/N|=p^{n-1}$, the group $G/N$ is super-solvable. 
Since $N$ is cyclic and $G/N$ is super-solvable, 
$G$ is super-solvable by Exercise~\ref{xca:Nciclico}.
\end{proof}

% Como todo grupo finito nilpotente es producto directo de (finitos) subgrupos de
% Sylow, cada $p$-grupo es súper-resoluble y el producto directo de súper-resolubles es súper-resoluble, 
% se obtiene el siguiente resultado:

\begin{exercise}
\label{xca:nilpotent=>supersolvable}
Prove that finite nilpotent groups are super-solvable.
\end{exercise}

% \begin{proof}
% 	Todo grupo finito nilpotente es producto directo (finito) de subgrupos de
% 	Sylow. Como cada $p$-grupo es súper-resoluble por la
% 	proposición~\ref{proposition:psuper}, el resultado se obtiene
% 	inmediatamente del ejercicio~\ref{exercise:directosuper}.
% \end{proof}

\begin{theorem}
Super-solvable groups have maximal subgroups. 	
\end{theorem}

\begin{proof} 
We proceed by induction on the length of the super-solvable series. The claim holds for groups with a super-solvable series of length one, as in this case we are dealing with cyclic groups. So let 
$G$ be a group admitting a sequence
	\[
		G=G_0\supseteq\cdots\supseteq G_k=\{1\}
	\]
and suppose the theorem holds for super-solvable groups
with super-solvable series of length $<k$. Note that   
$G_{k-1}$ is normal in $G$. Let $\pi\colon G\to
	G/G_{k-1}$ be the canonical map. 
The sequence 
 	\[
		G/G_{k-1}=\pi(G)\supseteq \pi(G_1)\supseteq\cdots\supseteq\pi(G_{k-1})=\{1\}
	\]
has length 
$<k$ and proves the super-solvability of $\pi(G)$. By the inductive hypothesis, 
$G/G_{k-1}$ admits maximal subgroups. By the correspondence theorem, 
$G$ admits maximal subgroups. 
\end{proof}

Solvable or nilpotent groups do not always admit maximal subgroups. Can you give an example?

\begin{definition}
	\index{Group!satisfying the maximal condition on subgroups}
A group $G$ satisfies the \textbf{maximal condition on subgroups} if
for every non-empty subset $\mathcal{S}$ of subgroups contains a maximal 
element (i.e. a subgroup not contained in any other subgroup of $\mathcal{S}$). 
	%toda sucesión creciente
	%$S_1\subseteq S_2\subseteq S_3\subseteq\cdots$
	%de subgrupos es finita. 
	%%si todo subconjunto $\mathcal{S}$ 
	%%no vacío de subgrupos tiene un elemento maximal, es decir: existe
	%$M\in\mathcal{S}$ tal que $S\subseteq M$ para todo $S\in\mathcal{S}$.
\end{definition}

%\begin{lemma}
%	Un grupo $G$ satisface la la condición maximal para subgrupos si y sólo si
%	todo subconjunto $\mathcal{S}$ no vacío de subgrupos tiene un subgrupo
%	maximal (es decir, no contenido en ningún otro subgrupo de $\mathcal{S}$). 
%\end{lemma}

\begin{exercise}
\label{xca:MAX=fg}
A group satisfies the maximal condition on subgroups if and only if
every subgroup of $G$ is finitely generated. 
\end{exercise}

% \begin{proof}
% 	Supongamos que $G$ satisface la condición maximal para subgrupos y sea $H$
% 	un subgrupo de $G$.  Sea $\mathcal{S}$ el conjunto de subgrupos de $H$
% 	finitamente generados. Como $\mathcal{S}$ es no vacío (pues
% 	$1\in\mathcal{S}$), existe un elemento maximal $M\in\mathcal{S}$.  Sea
% 	$x\in H$. Como $\langle M,x\rangle\in\mathcal{S}$, $M=\langle M,x\rangle$ y
% 	luego $x\in M$. Como entonces $H=M$, $H$ es finitamente generado.
% 	%Supongamos que $G$ no es finitamente generado y satisface la condición maximal para subgrupos. Sea $1\ne g\in G$
% 	%y sea $S_1=\langle g_1\rangle$. Como $S_1\ne G$, existe $g_2\in G\setminus S_1$, y entonces 
% 	%$S_1\subseteq S_2=\langle x_1,x_2\rangle$. 

% 	Supongamos ahora que todo subgrupo de $G$ es finitamente generado. Si
% 	$\mathcal{S}$ es un subconjunto no vacío de subgrupos de $G$ sin elemento
% 	maximal, podemos construir una sucesión de subgrupos $S_1\subseteq
% 	S_2\subseteq\cdots$ que no se estabiliza (acá necesitamos utilizar el
% 	axioma de elección). Como la unión 
% 	\[
% 		S=\bigcup_{j\geq1}S_j 
% 	\]
% 	es un subgrupo de $G$, es finitamente generado y luego $S\subseteq S_k$
% 	para algún $k$ suficientemente grande, una contradicción.
% \end{proof}

% Una consecuencia inmediata. 

\begin{exercise}
Let $H$ be a subgroup of a group $G$. If $G$ satisfies 
the maximal condition on subgroups, then so does $H$. 
\end{exercise}

\begin{exercise}
\label{xca:max:G/N}
Let $G$ be a group and $N$ be a normal subgroup of $G$. If $G/N$ and $N$
satisfy the maximal condition on subgroups, then so does $G$.
\end{exercise}

% \begin{proof} 
% 	Sea $\pi\colon G\to G/N$ el morfismo canónico.  Sea $\mathcal{S}$ un
% 	subconjunto no vacío de subgrupos de $G$. El conjunto $\{S\cap
% 	N:S\in\mathcal{S}\}$ tiene un elemento maximal $A$ y el conjunto
% 	$\{\pi(S):S\in\mathcal{S},S\cap N=A\}$ tiene un elemento maximal $B$. Sea
% 	$S\in\mathcal{S}$ tal que $\pi(S)=B$ y $S\cap N=A$. Si $S$ no es maximal en
% 	$\mathcal{S}$, existe $T\in\mathcal{S}$ tal que $S\subseteq T$, $N\cap T=A$
% 	y $\pi(T)=B$. Sea $x\in T\setminus S$. Como $\pi(xN)=\pi(x)\in\pi(T)=B$,
% 	existe $y\in S$ tal que $xN=yN$. Luego $y^{-1}x\in N\cap T=A=N\cap S$, una
% 	contradicción pues $x\not\in S$. 
% \end{proof}

% TODO: agregar teorema de Huppert (ver por ejemplo Robinson, p. 268)
% corolario: G super si y sólo G/\Phi(G) super
% teorema de Iwasawa, Hall 342-345, 19.3
% teorema de Zappa-Ore, Duke 5 (1939), 431-460, Duke 6 (1940), 511-512

%\begin{definition}
%	\index{Grupo!que satisface la condición minimal para subgrupos}
%	Se dice que un grupo $G$ satisface la \textbf{condición minimal para
%	subgrupos} si todo subconjunto no vacío de subgrupos tiene un elemento
%	minimal.
%\end{definition}
%
%\begin{example}
%	El grupo $\Z$ no satisface la condición minimal para subgrupos pues
%	el conjunto $\{2^n\Z:n\in\N\}$ no posee elemento minimal. 
%\end{example}
%
%\begin{proposition}
%	Sea $G$ un grupo que satisface la condición minimal sobre subgrupos.
%	Entonces todo elemento de $G$ tiene orden finito.
%\end{proposition}
%
%\begin{proof}
%	Si existe $x\in G$ de orden infinito, la sucesión $\mathcal{S}$ de subgrupos 
%	\[
%	\langle x\rangle\supsetneq\langle x^2\rangle\supsetneq\langle
%	x^4\rangle\supsetneq\cdots\supsetneq\langle x^{2^k}\rangle\supsetneq\cdots
%	\]
%	tiene infinitos elementos y luego no posee un elemento minimal. 
%\end{proof}
%
%\begin{exercise}
%	\label{exercise:min:N}
%	Sea $G$ un grupo y sea $H$ un subgrupo de $G$.  Si $G$ satisface la
%	condición minimal para subgrupos entonces $H$ también. 
%\end{exercise}
%
%\begin{svgraybox}
%	Si $\mathcal{S}$ es un subconjunto no vacío de subgrupos de $H$, entonces
%	$\mathcal{S}$ posee un elemento minimal por ser un subconjunto no vacío de
%	subgrupos de $G$.
%\end{svgraybox}
%
%\begin{proposition}
%	\label{proposition:min:G/N}
%	Sea $G$ un grupo y sea $N$ un subgrupo normal de $G$.  Si $G/N$ y $N$
%	satisfacen la condición minimal para subgrupos entonces $G$ también. 
%\end{proposition}
%
%\begin{proof}
%	
%\end{proof}

\begin{proposition}
\label{pro:superfg}
Super-solvable groups satisfy the maximal condition on subgroups. In particular, 
every super-solvable group is finitely generated. 
\end{proposition}

\begin{proof}
We proceed by induction on the length of the super-solvable sequence. If the length
is one, the result holds as the group is cyclic. 
So assume the result holds for super-solvable groups with 
super-solvable series of length $\leq n-1$.  Let $G$
be a non-trivial super-solvable group and 
	\[
	G=G_0\supsetneq
	G_1\supsetneq\cdots\supsetneq G_n=\{1\}
	\]
a sequence of normal subgroups of $G$ with cyclic factors. Since 
$G_{1}$ is super-solvable (Exercise~\ref{xca:super}),
	$G_{1}$ satisfies the maximal condition on subgroups by the inductive
 hypothesis. By Exercise~\ref{xca:max:G/N}, $G$ satisfies the maximal
 condition on subgroups, as 
 $G/G_{1}$ is cyclic. 
\end{proof}

%\begin{proposition}\
%	\begin{enumerate}
%		\item Si un grupo súper-resoluble admite una serie de composición,
%			entonces es finito. 
%		\item Si un grupo súper-resoluble satisface la condición de minimal en
%			subgrupos entonces es finito.
%	\end{enumerate}
%\end{proposition}
%
%\begin{proof}
%	%Para probar la segunda afirmación obsevemos que todo cociente de $G$ es súper-resoluble 
%	%y que por el teorema~\ref{theorem:ZorCp} todo factor de la serie debe ser finito pues
%	%$\Z$ no satisface la condición minimal para subgrupos.
%\end{proof}

\begin{example}
The abelian group $\Q$ is nilpotent but not super-solvable, 
as it is not finitely generated. 
\end{example}

%	El grupo $\Sym_3$ es súper-resoluble pero no es nilpotente. 

If $G$ is a group and $x_1,\dots,x_{n+1}\in G$, 
let 
\[
[x_1,x_2\dots,x_{n+1}]=\left[ x_1,[x_2,\dots,x_{n+1}]\right],\quad
n\geq1.
\]

We will prove in Theorem~\ref{thm:super=fg} that nilpotent groups are super-solvable 
if and only if they are finitely generated. For this, we need two lemmas. 

\begin{lemma}
	\label{lem:G_n}
	Let $G$ be a finite generated group, say $G=\langle X\rangle$ for some finite set $X$.  
	For $n\geq2$, let 
 	\[
		G_n=\langle g[x_1,\dots,x_n]g^{-1}:x_1,\dots,x_n\in X,\,g\in G\rangle.
	\]
	Then $G_n=\gamma_n(G)$ for all $n\geq2$. 
\end{lemma}

\begin{proof}
	Note that each $G_n$ is normal in $G$. We proceed by induction on $n$. The case 
    $n=2$ is trivial. So let us assume that 
    $\gamma_{n-1}(G)=G_{n-1}$ for some $n\geq2$. Let $x_1,\dots,x_n\in X$. Since 
	$[x_1,\dots,x_n]\in\gamma_{n}(G)$, $G_{n-1}\subseteq\gamma_n(G)$. Let 
	$N=G_n$ and $\pi\colon G\to G/N$ be the canonical map. The group $G/N$ is finitely generated. Since 
	\[
	[\pi(x_1),[\pi(x_2),\dots,\pi(x_{n})]]=\pi([x_1,\dots,x_n])=1,
	\]
	we obtain that $\pi([x_2,\dots,x_{n}])\in Z(G/N)$. Hence 
	$\pi(g[x_2,\dots,x_n]g^{-1})=1$ for all $g\in G$. By the inductive hypothesis, 
 	\[
	\pi(\gamma_{n-1}(G))=\pi(G_{n-1})\subseteq Z(G/N).
	\]
	Since  
	\[
	\pi(\gamma_{n}(G))=\pi([G,\gamma_{n-1}(G)])=[\pi(G),\pi(\gamma_{n-1}(G))]=\{1\},
	\]
	we conclude that $\gamma_n(G)\subseteq N=G_n$.
\end{proof}

\begin{lemma}
	\label{lem:gamma_n/gamma_n+1}
	Let $G$ be a finitely generated group. Then 
 	$\gamma_n(G)/\gamma_{n+1}(G)$ is finitely generated. 
\end{lemma}

\begin{proof}
	Assume that $G=\langle X\rangle$ for some finite set $X$. Write 
	\[
	g[x_1,\dots,x_n]g^{-1}=[g,[x_1,\dots,x_n]][x_1,\dots,x_n]. 
	\]
	By Lemma~\ref{lem:G_n}, 
    $[g,[x_1,\dots,x_n]]\in \gamma_{n+1}(G)=G_{n+1}$. Then 
	\[
	g[x_1,\dots,x_n]g^{-1}\equiv [x_1,\dots,x_n]\bmod \gamma_{n+1}(G). 
	\]
	Hence $\gamma_{n}(G)/\gamma_{n+1}(G)$ is generated by the finite set 
	\[
	\{[x_1,\dots,x_n]\gamma_{n+1}(G):x_1,\dots,x_n\in X\}. \qedhere 
	\]
\end{proof}

\begin{theorem}
    \label{thm:super=fg}
    Let $G$ be a nilpotent group. Then $G$ is super-solvable if and only if 
    $G$ is finitely generated. 
\end{theorem}

\begin{proof}
    If $G$ is super-solvable, it is then finitely generated by 
    Proposition~\ref{pro:superfg}.  
    
    Now assume that the nilpotent group $G$ is finitely generated. By Lemma~\ref{lem:gamma_n/gamma_n+1}, 
    each quotient $\gamma_{n}(G)/\gamma_{n+1}(G)$ is finitely generated, say by 
    the elements $y_1,\dots,y_m$. Let $\pi\colon G\to G/\gamma_{n+1}(G)$ the canonical map. 
    For $j\in\{1,\dots,m\}$, let 
    \[
    K_j=\langle \gamma_{n+1}(G),y_1,\dots,y_j\rangle.
    \]
    Since $[G,K_j]\subseteq [G,\gamma_n(G)]=\gamma_{n+1}(G)$, 
    we obtain that $\pi(K_j)$ is central in $\pi(G)$. Thus $\pi(K_j)$ is normal
    in $\pi(G)$. Hence $K_j$ is normal in $G$. Each quotient $K_j/K_{j-1}$
    is cyclic and generated by $y_jK_{j-1}$. Therefore, in between $\gamma_n(G)$ and 
    $\gamma_{n+1}(G)$, we have constructed a sequence of normal subgroups of $G$ 
    with cyclic factors. Since $G$ is nilpotent, there exists an integer $c$ such that 
    $\gamma_{c+1}(G)=\{1\}$. Hence $G$ is super-solvable. 
\end{proof}

\begin{corollary}
	\label{cor:nilpotente=>max}
    Every finitely generated nilpotent group satisfies the maximal condition on subgroups. 
\end{corollary}

\begin{proof}
    This is an immediate consequence of Proposition~\ref{pro:superfg} and 
    Theorem~\ref{thm:super=fg}.  
\end{proof}

\begin{theorem}
    Let $G$ be a nilpotent finitely generated group. Then $T(G)$ is finite. 
\end{theorem}

\begin{proof}
    Since $G$ is nilpotent, $G$ satisfies the maximal condition on subgroups 
    (Corollary~\ref{cor:nilpotente=>max}). Thus 
	every subgroup of $G$ is finitely generated. Since 
    $T(G)$ is a subgroup (Theorem~\ref{thm:T(nilpotent)}), it is a torsion finitely generated group. 
	Hence $T(G)$ is finite by Theorem~\ref{thm:T(G)finito}.
\end{proof}

\subsection{*Huppert's super-solvable theorem}

\begin{theorem}[Huppert]
\index{Hupper's super-solvable theorem}
\label{thm:Huppert}
Let $G$ be a finite group such that all its maximal subgroups 
are of prime index. Then $G$ is super-solvable. 
\end{theorem}

\begin{proof}
    See \cite[Theorem 10.5.8]{MR414669}.
\end{proof}

\subsection{*Formanek's zero divisors theorem}

We start recalling a conjecture formulated by Kaplansky as \cite[Problem 6]{MR0096696} in 1957. 

\begin{conjecture}[Kaplansky]
\index{Kaplansky's zero divisors conjecture}
\label{conjecture:zero}
    Let $K$ be a field and $G$ be a torsion-free group. 
    Then $K[G]$ has no zero divisors. 
\end{conjecture}

In 1973 Formanek proved the following result:

\begin{theorem}[Formanek]
    \index{Formanek's zero divisors theorem}
    \label{thm:Formanek:zerodivisors}
        Let $G$ be a torsion-free super-solvable 
        group and $K$ be a field. Then $K[G]$ has no zero divisors. 
\end{theorem}

\begin{proof}
    See \cite[Thoerem 13.3.9]{MR798076}.
\end{proof}

Conjecture \ref{conjecture:zero} is also known to hold, for example, when $G$ admits a bi-order (proved by Malcev and independently by Neumann), when $G$ is polycyclic-by-finite (proved by Brown and Farkas--Snider), or when $G$ has the unique product property (proved by Cohen).  In full generality, the conjecture is still open. See~\cite[Chapter 13]{MR798076} for more information. 


