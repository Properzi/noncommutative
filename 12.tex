\section{24/05/2024}

\subsection{*The Deaconescu--Walls theorem}

Let $A$ be a group acting on automorphisms on a finite group $G$. Then 
$C_{G}(A)=\{g\in G:a\cdot g=g\,\,\forall a\in A\}$ acts by left multiplication 
on the set of 
$A$-orbits by 
\[
  c(A\cdot g)
  =\{c(a\cdot g):a\in A\}
  =\{(a\cdot c)(a\cdot g):a\in A\}
  =\{a\cdot (cg):a\in A\}
  =A\cdot (cg)
\]
for all $g\in G$ and $c\in C_G(A)$.

%The following theorem first appeared in~\cite{MR2164558}. 
%The proof presented here goes back to Isaacs, see~\cite{MR2922681}. 

\begin{theorem}[Deaconescu--Walls]
	\index{Deaconescu--Walls theorem}
	\label{thm:DeaconescuWalls}
	Let $A$ be a group acting by automorphisms on a finite group $G$. Let
	$C=C_{G}(A)$ and $N=C\cap [A,G]$,
	where $[A,G]$ is the subgroup of $G$ generated by $[a,g]=(a\cdot g)g^{-1}$,
	$a\in A$, $g\in G$.  Then $(C:N)$ divides the number of $A$-orbits of 
	$G$. 
\end{theorem}

\begin{proof}
  The group $C$ acts by left multiplication on the set $\Omega$ of 
  $A$-orbits of $G$. Let $X=A\cdot g\in\Omega$ and $C_X$ be the stabilizer of 
  $C$ in $X$. If $c\in C_X$, then $cX=X$. In particular, if $c\in C_X$, then 
  $cg=a\cdot g$ for some $a\in A$, that is $c=(a\cdot
  g)g^{-1}=[a,g]\in [A,G]$. Thus $C_X\subseteq N$.

  To show that $(C:N)$ divides $|\Omega|$, it is enough to show that 
  $(C:N)$ divides the size of each $C$-orbit. If $X\in\Omega$, then $C\cdot
  X$ has size 
  \[
	(C:C_X)=(C:N)(N:C_X).
  \]
  Hence $(C:N)$ divides the size of the orbit $C\cdot X$.
\end{proof}

\begin{corollary}
	\label{cor:Z(G)subset[G,G]}
  Let $G$ be a non-trivial finite group with $k(G)$ conjugacy classes. 
  If the order of $Z(G)$ is coprime with $k(G)$, then  
  $Z(G)\subseteq[G,G]$.
\end{corollary}

\begin{proof}
	The group $A=G$ acts on $G$ by conjugation. Since $C_G(A)=Z(G)$ and 
	$[A,G]=[G,G]$, Theorem~\ref{thm:DeaconescuWalls} implies that the index 
	$(Z(G):Z(G)\cap [G,G])$ divides $k(G)$. Since $k(G)$ and $|Z(G)|$ are coprime, we conclude that $Z(G)=Z(G)\cap [G,G]\subseteq [G,G]$. 
\end{proof}

\begin{definition}
	\index{Central automorphism}
 	Let $G$ be a group and $f\in\Aut(G)$. We say that $f$ is \textbf{central} if 
	$f(x)x^{-1}\in Z(G)$ for all $x\in G$.
\end{definition}

An automorphism $f$ of a group $G$ 
is central if and only if $f\in C_{\Aut(G)}(\Inn(G))$.

\begin{corollary}
	Let $G$ be a finite group with $k(G)$ conjugacy classes and $c(G)$
	central automorphisms. If $\gcd(|G|,k(G)c(G))=1$, then 
	$[G,G]=Z(G)$.
\end{corollary}

\begin{proof}
	By Corollary~\ref{cor:Z(G)subset[G,G]}, $Z(G)\subseteq [G,G]$. Conversely, let 
	$A=C_{\Aut(G)}(\Inn(G))$. Since $|G|$ and $k(G)c(G)$ are coprime 
	and $(C_G(A):C_G(A)\cap [A,G])$ divides $c(G)$ by 
	Theorem~\ref{thm:DeaconescuWalls}, we obtain that $C_G(A)=C_G(A)\cap [A,G]$. 
	Since 
	\[
		a\cdot [x,y]=[(a\cdot x)x^{-1}x,(a\cdot y)y^{-1}y]=[x,y]
	\]
	for all $a\in A$ and $x,y\in G$, 
    $[G,G]\subseteq C_G(A)$. Moreover, 
    $[A,G]\subseteq Z(G)$. Thus 
	\[
	[G,G]\subseteq C_G(A)=C_G(A)\cap [A,G]\subseteq [A,G]\subseteq Z(G).\qedhere 
	\]
\end{proof}

\begin{exercise}
    Let $p$ be a prime number and $G$ be a group with $p$ conjugacy classes. 
    Prove that either $Z(G)\subseteq[G,G]$ or $|G|=p$. 
\end{exercise}

% \begin{proof}
%   Hacemos actuar a $G$ en $G$ por conjugación.  Como cada elemento de $C=Z(G)$
%   es una clase de conjugación, $|C|\leq p$. Si $|C|=p$ entonces $G=C=Z(G)$
%   tiene orden $p$. Si no, $|C|$ es coprimo con $p$ y luego $C\subseteq
%   N=[G,G]$.
% \end{proof}


\subsection{*The Chermak--Delgado subgroup}

\begin{definition}
\index{Chermak--Delgado!measure}
Let $G$ be a finite group and $H$ a subgroup of $G$. 
The \textbf{Chermak--Delgado measure} of $H$ 
is the number 
$m_G(H)=|H||C_G(H)|$.
\end{definition}

\begin{example}
If $G$ is abelian and $H$ is a subgroup of $G$, then 
$m_G(H)=|H||G|$.
\end{example}

\begin{example}
Let $G=\Sym_3$. The subgroups of $G$ are 
	\[
		H_0=1,\quad
		H_1=\langle (23)\rangle,\quad
		H_2=\langle (12)\rangle,\quad
		H_3=\langle (13)\rangle,\quad
		H_4=\langle (123)\rangle,\quad
		H_5=\Sym_3.
	\]
	A direct calculation shows that 
	\[
		m_G(H_j)=\begin{cases}
			6 & \text{if $j\in\{0,5\}$},\\
			4 & \text{if $j\in\{1,2,3\}$},\\
			9 & \text{if $j=4$}.
		\end{cases}
	\]
\end{example}

\begin{lemma}
\label{lem:CD1}
Let $G$ be a finite group and $H$ be a subgroup of $G$. Then 
\[
m_G(H)\leq m_G(C_G(H)).
\]
If the equality holds, then $H=C_G(C_G(H))$.
\end{lemma}

\begin{proof}
Let $C=C_G(H)$. 
Since $H\subseteq C_G(C)$, 
\[
m_G(C)=|C||C_G(C)|\geq |C||H|=m_G(H). 
\]
If $m_G(H)=m_G(C_G(H))$, then $|H|=|C_G(C_G(H))|$ and 
hence $H=C_G(C_G(H))$, as $H\subseteq C_G(C_G(H))$. 
\end{proof}

\begin{lemma}
	Let $G$ be a finite group and 
	$H$ and $,K$ be subgroups of $G$. Let $D=H\cap K$ and
    $J=\langle H,K\rangle$. Then 
	\[
		m_G(H)m_G(K)\leq m_G(D)m_G(J).
	\]
	If the equality holds, then $J=HK$ and $C_G(D)=C_G(H)C_G(K)$.
	\label{lem:CD2}
\end{lemma}

\begin{proof}
	Let $C_H=C_G(H)$, $C_K=C_G(K)$, $C_D=C_G(D)$, and $C_J=C_G(J)$. Then
	$C_J=C_H\cap C_K$ and $C_H\cup C_K\subseteq C_D$. Since 
	\[
		|J|\geq |HK|=\frac{|H||K|}{|D|},
		\quad
		|C_D|\geq |C_HC_K|=\frac{|C_H||C_K|}{|C_J|},
	\]
	we obtain that 
	\[
		m_G(D)
		=|D||C_D|\geq \frac{|H||K|}{|J|}\frac{|C_H||C_K|}{|C_J|}
		=\frac{m_G(H)m_G(K)}{m_G(J)}.
	\]
	The second claim is clear. 
\end{proof}

\begin{definition}
\index{Lattice of subgroups}
Let $G$ be a finite group and $\mathcal{L}$ be a collection of subgroups of $G$. We say that $\mathcal{L}$ is a \textbf{lattice} if for every $H,K\in\mathcal{L}$ one has that
$H\cap K\in\mathcal{L}$ and $\langle H,K\rangle\in\mathcal{L}$. 
\end{definition}

Since $G$ is finite, it makes sense to consider the set $\mathcal{L}(G)$ of 
subgroups of $G$ $G$ where the Chermak--Delgado gets its largest value,
say $M_G$. 

\begin{exercise}
	\label{xca:M_S}
	Let $G$ be a finite group and $H$ be a subgroup of $G$. Prove that 
	$M_H\leq M_G$.
\end{exercise}

% \begin{svgraybox}
% 	Sabemos que existe algún subgrupo $K$ de $H$ tal que $M_H=m_H(K)$. Como
% 	$C_H(K)\subseteq C_G(K)$, 
% 	\[
% 		M_H=m_H(K)=|H||C_H(K)|\leq |H||C_G(K)|\leq m_G(H)\leq M_G.
% 	\]
% \end{svgraybox}

\begin{example}
	\label{exa:D8_CD}
    Let $G=\D_8=\langle r,s:r^4=s^2=1,srs=r^{-1}\rangle$ be the dihedral group
    of eight elements. In the subgroups 
    \[
		G,
		\quad
		Z(G)=\{1,r^2\},\quad
		A=\{1,r,r^2,r^3\},\quad
		B=\{1, s,sr^2,r^2\},\quad
		C=\{1,sr,sr^3,r^2\},
	\]
	the Chermak--Delgado measure is $16$ and this is the largest possible value. Thus and $M_G=16$ and $\mathcal{L}(G)=\{G,Z(G),A,B,C\}$. 
	\begin{lstlisting}
gap> ChermakDelgado := function(group, subgroup)
> return Size(subgroup)\
> *Size(Centralizer(group, subgroup));
> end;
function( group, subgroup ) ... end
gap> gr := DihedralGroup(IsPermGroup, 8);;
gap> r := gr.1;;
gap> s := gr.2;;
gap> ChermakDelgado(gr, Subgroup(gr, [r]));
16
gap> ChermakDelgado(gr, Subgroup(gr, [s*r,s*r^3]));
16
gap> ChermakDelgado(gr, Subgroup(gr, [s,s*r^2]));
16
gap> ChermakDelgado(gr, Subgroup(gr, [r^2]));
16
gap> List(AllSubgroups(gr), x->ChermakDelgado(gr, x));
[ 8, 16, 8, 8, 8, 8, 16, 16, 16, 16 ]
	\end{lstlisting}
\end{example}

\begin{theorem}
	Let $G$ be a finite group. The following statements hold: 
	\begin{enumerate}
		\item $\mathcal{L}(G)$ is a lattice. 
		\item If $H,K\in\mathcal{L}(G)$, then $\langle H,K\rangle=HK$.
		\item If $H\in\mathcal{L}(G)$, then $C_G(H)\in\mathcal{L}(G)$ and $C_G(C_G(H))=H$.
	\end{enumerate}
	\label{thm:lattice}
\end{theorem}

\begin{proof}
	If $H,K\in\mathcal{L}(G)$, then $m_G(H)=m_G(K)=M_G$. Let $D=H\cap K$ and $J=\langle
	H,K\rangle$. By Lemma~\ref{lem:CD2}, 
	\[
		M_G^2=m_G(H)m_G(K)\leq m_G(D)m_G(J).
	\]
	Since $m_G(D)\leq M_G$ and $m_G(J)\leq M_G$ (because $M_G$ is the largest possible value), we conclude that $m_G(D)=m_G(J)=M_G$. Hence $\mathcal{L}(G)$ is a lattice. 

	Since $m_G(H)m_G(K)=m_G(D)m_G(J)=M_G^2$, Lemma~\ref{lem:CD2} implies that 
	$J=HK$. 

	By Lemma~\ref{lem:CD1}, 
	\[
	M_G=m_G(H)\leq m_G(C_G(H)).
	\]
	Since $M_G$ is the largest possible value, $m_G(C_G(H))=M_G$. Thus 
    $C_G(H)\in\mathcal{L}(G)$.  By Lemma~\ref{lem:CD1}, $C_G(C_G(H))=H$.
\end{proof}

\index{Chermak--Delgado!subgroup}
Theorem~\ref{thm:lattice} implies the existence 
of the \textbf{Chermak--Delgado subgroup}.

\begin{corollary}
	\label{cor:ChermakDelgado}
	Let $G$ be a finite group. There exists a unique subgroup $M$ minimal 
    suc that $m_G(M)$ is the largest possible value among all the subgroups 
    of $G$. Moreover, $M$ is characteristic, abelian and $Z(G)\subseteq M$. 
\end{corollary}

% f(C_G(H))=C_G(f(H))$ para todo $H$ y todo $f\in\Aut(G)$.

\begin{proof}
	By Theorem~\ref{thm:lattice}, $\mathcal{L}(G)$ is a lattice. Let 
	\[
		M=\bigcap_{H\in\mathcal{L}(G)}H\in\mathcal{L}(G).
	\]
	By Theorem~\ref{thm:lattice},  
	\[
    C_G(M)\in\mathcal{L}(G)
    \text{ and }M=C_G(C_G(M))\supseteq Z(G).
    \]Since $C_G(M)\in\mathcal{L}(G)$, $M\subseteq C_G(M)$. Hence $M$ is abelian. Moreover, $M$ is characteristic in $G$ because $f(M)\in\mathcal{L}(G)$
	for all $f\in\Aut(G)$.
\end{proof}

\begin{example}
	Let $G=\D_8$ be the dihedral group of eight elements. The Chermak--Delgado subgroup of $G$ is $Z(G)\simeq C_2$. See Example~\ref{exa:D8_CD}.
\end{example}

\begin{theorem}[Chermak--Delgado]
	\index{Chermak--Delgado!theorem}
 	\label{thm:ChermakDelgado}
	Let $G$ be a finite group. Then $G$ has an abelian characteristic subgroup $M$ such that $(G:M)\leq (G:A)^2$ for every abelian subgroup 
	$A$ of $G$. 
\end{theorem}

\begin{proof}
	Let $M$ be the Chermak--Delgado subgroup of Corollary~\ref{cor:ChermakDelgado} 
    and $A$ be an abelian subgroup of 
	$G$. Then $A\subseteq C_G(A)$. Hence 
	\[
		m_G(M)\geq m_G(A)=|A||C_G(A)|\geq|A|^2
	\]
	and 
	\[
	(G:A)^2
	=\frac{|G|^2}{|A|^2}\geq\frac{|G|^2}{m_G(M)}
	=\frac{|G|}{|M|}\frac{|G|}{|C_G(M)|}
	=\frac{|G|}{|M|}
	=(G:M).\qedhere 
	\]
\end{proof}

\begin{corollary}
	Let $G$ be a non-abelian finite group and $H$ be a subgroup of $G$ such that 
	\[
	|H||C_G(H)|>|G|.
	\]
	Then $G$ is not simple. 
\end{corollary}

\begin{proof}
	Let $M$ be the Chermak--Delgado subgroup of $G$. 
	Then 
	\begin{equation}
		\label{equation:mG}
	m_G(M)\geq m_G(H)>|G|.
	\end{equation}
	This implies that $M\ne\{1\}$, since $m_G(M)=m_G(1)=|G|$. If $G$ is simple, then $G=M$ is abelian. 
\end{proof}

\begin{corollary}
	Let $G$ be a non-abelian finite group and $P\in\Syl_p(G)$. If $P$ is abelian and $|P|^2>|G|$, then $G$ is not simple. 
\end{corollary}

\begin{proof}
	Let $M$ be the Chermak--Delgado subgroup of $G$. Since $P$ is abelian, 
    \[
    (G:M)\leq (G:P)^2<|G|
    \]
    by Theorem~\ref{thm:ChermakDelgado}. Hence 
    $M\ne\{1\}$. If $G$ is simple, then $G=M$ is abelian. 
\end{proof}

We now discuss an application of the Wielandt zipper theorem 
to the Chermak--Delgado lattice. 

\begin{lemma}
	\label{lem:L(G)L(S)}
	Let $G$ be a finite group, $H\in\mathcal{L}(G)$ and  $S$ be a subgroup of $G$ such that 
	$HC_G(H)\subseteq S$. Then $H\in\mathcal{L}(S)$.
\end{lemma}

\begin{proof}
	Since $C_G(H)\subseteq S$, $C_G(H)=C_S(H)$. By Exercise~\ref{xca:M_S},
    $M_S\leq M_G$. Thus $M_G=M_S$, since 
	\[
		M_G=m_G(H)=|H||C_G(H)|=|H||C_S(H)|=m_S(H)\leq M_S.\qedhere 
	\]
\end{proof}

\begin{theorem}
	\label{thm:L(G)subnormal}
	Let $G$ be a finite group. Every $H\in\mathcal{L}(G)$ is subnormal in $G$.
\end{theorem}

\begin{proof}
	We proceed by induction on $|G|$. If $|G|=1$, the result is trivial. So assume the group $G$ is non-trivial. Let $H\in\mathcal{L}(G)$ and $K=HC_G(H)$. Since $H$ is normal in $K$, by the inductive hypothesis, 
    it is enough to show that 
	$K$ is subnormal in $G$. If $K=G$, the claim holds. So assume that 
	$K\ne G$. 

	Assume that $K$ is not subnormal in $G$. By the inductive hypothesis and 
    Wielandt's zipper theorem (Theorem~\ref{thm:zipper}), there exists a unique 
    maximal subgroup $M$ containing $K$. By Theorem~\ref{thm:lattice},
	$C_G(H)\in\mathcal{L}(G)$ and $K=HC_G(H)\in\mathcal{L}(G)$. By Lemma~\ref{lem:L(G)L(S)},
	$H\in\mathcal{L}(M)$. Hence $K\in\mathcal{L}(M)$. By the inductive hypothesis, $K$ is subnormal in $M$. We claim that $M$ is normal in $G$. Let $x\in G$. Since 
	$m_G(xKx^{-1})=m_G(K)$, the subgroup $xKx^{-1}\in\mathcal{L}(G)$. Hence 
	$K(xKx^{-1})\in\mathcal{L}(G)$. 
	
	If $K(xKx^{-1})=G$, then, since there exist $k_1,k_2\in K$ such that 
	$k_1(xk_2x^{-1})=x^{-1}$, we obtain that $x\in K$, since $x^{-1}=k_2k_1\in K$. This implies that $xKx^{-1}\subseteq K$. Therefore $K=G$, a contradiction.

	Since $K(xKx^{-1})\ne G$, there exists a maximal subgroup $N$ such that 
	$K(xKx^{-1})\subseteq N$. Since $K\subseteq N$, $N=M$ because $M$ is the unique
	maximal subgroup containing $K$. Since $xKx^{-1}\subseteq M$, $K\subseteq
	x^{-1}Mx$. Hence $x^{-1}Mx=M$, because $x^{-1}Mx$ is a maximal subgroup containing $K$ and $M$ is the only maximal subgroup containing $K$. 
\end{proof}

\begin{corollary}
	Let$G$ be a non-abelian finite Then $\mathcal{L}(G)=\{1,G\}$. 
\end{corollary}

\begin{proof}
	Let $K\in\mathcal{L}(G)$. Then $K$ is subnormal in $G$ by Theorem~\ref{thm:L(G)subnormal}. Hence $K\in\{1,G\}$. Now the claim follows from $m_G(1)=m_G(G)$. 
\end{proof}

\begin{exercise}
	Let $n\geq5$. Prove that $\mathcal{L}(\Sym_n)=\{1,\Sym_n\}$. 
\end{exercise}

% \begin{proof}
% 	Let $G=\Sym_n$ y sea $K\in\mathcal{L}(G)$. Por el
% 	teorema~\ref{thm:L(G)subnormal}, $K$ es subnormal en $G$. Si $K\ne G$
% 	entonces se tiene una sucesión estrictamente creciente de subgrupos 
% 	\[
% 	K=K_1\triangleleft
% 	K_2\triangleleft\cdots\triangleleft K_{n-1}\triangleleft K_n=G.
% 	\]
% 	Como $K_{n-1}$ es normal en $G$, $K_{n-1}\in\{1,\Alt_n\}$ y luego $K=1$. 
% 	El corolario queda demostrado al observar que $m_G(1)=m_G(G)$. 
% \end{proof}



\subsection{Miller's double cosets theorem}

\index{Double cosets}
Let $G$ be a group and $H$ and $K$ be subgroups of $G$. 
The group $L=H\times K$ acts on $G$ by
\[
(h,k)\cdot g=hgk^{-1},\quad h\in H,k\in K,g\in G.
\]
The orbits of this action are the set of the form 
\[
HgK=\{hgk:h\in H,\,k\in K\}.
\]
A set of the form $HgK$ for some $g\in G$ is called a \textbf{double coset} modulo $(H,K)$ 
with representative $g$. In particular, 
any two double cosets are either disjoint or equal, and $G$ decomposes
as a disjoint union 
\[
G=\bigcup_{i\in I}Hg_iK,
\]
for some set $I$. Let 
\[
L_g=\{(h,k)\in H\times K:hgk^{-1}=g\}=\{(h,g^{-1}hg)\in H\times K\}.
\]
Then
$|L_g|=|H\cap gKg^{-1}|$, 
because there is a bijection $L_g\to H\cap gKg^{-1}$.  
By the fundamental counting principle, 
\[
|HgK|=(L:L_g)=\frac{|H\times K|}{|H\cap gKg^{-1}|}=\frac{|H||K|}{|H\cap gKg^{-1}|}.
\]

We need a lemma. 

\begin{lemma}
\label{lem:Miller}
    Let $G$ be a finite group, $x\in G$, and $H$ and $K$ be subgroups of $G$. Then
    \[
    \#\{zK:zK\subseteq HxK\}=(H:xKx^{-1}\cap H).
    \]
\end{lemma}

\begin{proof}
    Let $L=xKx^{-1}\cap H$ and 
    \[
    \varphi\colon H/L\to\{yK:yK\subseteq HxK\},\quad 
    hL\mapsto hxK.
    \]

    The map $\varphi$ is well-defined. If $hL=h_1L$, then $h^{-1}h_1\in L$. Thus 
    $h^{-1}h_1=xkx^{-1}$ for some $k\in K$. This means that
    \[
    (h_1x)^{-1}(hx)=x^{-1}h_1^{-1}hx=k\in K,
    \]
    that is $\varphi(hL)=hxK=h_1xK=\varphi(h_1L)$. 

    The map $\varphi$ is surjective: If $zK$ is such that $zK\subseteq HxK$, then 
    $z=hxk$ for some $k\in K$. In particular, 
    $zK=hxK$. Now $\varphi(hL)=hxK=zK$.

    The map $\varphi$ is injective: If $hxK=h_1xK$, then 
    $x^{-1}h_1^{-1}hx\in K$. Moreover, 
    $h_1^{-1}h\in xKx^{-1}\cap H=L$. Thus $h_1L=hL$. 
\end{proof}

\begin{exercise}
\label{xca:Miller}
    Let $G$ be a finite group, $H$ and $K$ be subgroups of $G$, and $x\in G$. Prove 
    that 
    \[
    \#\{Hy:Hy\subseteq HxK\}=(K:xHx^{-1}\cap K).
    \]
\end{exercise}

\begin{theorem}[Miller]
\index{Miller' theorem}
    Let $G$ be a finite group and $H$ and $K$ be subgroups of $G$ 
    of the same index. Then there exists a common complete set
    of representatives for the right cosets of $H$ in $G$ and the 
    left cosets of $K$ in $G$. 
\end{theorem}

\begin{proof}
    Let $Hy$ be a right coset and $zK$ be a left coset. Note that 
    $Hy$ and $zK$ have a common representative
    if and only if $Hy\cap zK\ne\emptyset$, as 
    \[
    Hy=Hx\text{ and }zK=xK
    \Longleftrightarrow 
    xy^{-1}\in H\text{ and }z^{-1}x\in K
    \Longleftrightarrow x\in Hy\cap zK.
    \]

    The group $G$ decomposes as a  
    disjoint union of finitely many double cosets. Each doble coset
    $HxK$ is a disjoint union of finitely many right cosets of $H$ 
    and a disjoint union of finitely many left cosets of $K$. Thus 
    \[
    HxK=\bigcup_{i=1}^kHy_i=\bigcup_{j=1}^lz_jK, 
    \]
    where the unions are disjoint. 
    Since $|H|=|K|$, by applying cardinality, it follows that $k=l$. To prove the theorem
    it is enough to show that each $Hy_i$ intersects every $z_jK$. 
    
    Note that for each $i\in\{1,\dots,k\}$ there exists $j\in\{1,\dots,k\}$ such that
    $Hy_i\cap z_jK\ne\emptyset$. 
    Without loss of generality, we may assume (reordering if needed) that 
    $Hy_1\cap z_jK\ne\emptyset$ for all $j\in\{1,\dots,m\}$, where $1\leq m\leq k$. Then
    \[
    Hy_1\subseteq\bigcup_{j=1}^mz_jK. 
    \]
    Then
    \[
    Hy_1K\subseteq\bigcup_{j=1}^mz_jK\subseteq \bigcup_{j=1}^kz_jK=HxK.
    \]
    Since $Hy_1K$ and $HxK$ are double cosets with non-empty intersection, 
    they are equal. Thus 
    \[
    |HxK|=|Hy_1K|\leq \sum_{j=1}^m|z_jK|=m|K|.
    \]
    By Lemma~\ref{lem:Miller}, 
    \[
    k=\#\{z_jK:z_jK\subseteq HxK\}=(H:xKx^{-1}\cap H). 
    \]
    Therefore
    \[
    k|K|=\frac{|H||K|}{|H\cap xKx^{-1}|}=|HxK|\leq m|K|
    \]
    and hence $k=m$. 
\end{proof}

\begin{exercise}[Hall]
\label{xca:Hall:cosets}
    Let $G$ be a finite group and $H$ be a subgroup of $G$ with $(G:H)=n$. 
    Then there exists $x_1,\dots,x_n\in G$ such that 
    $\{Hx_1,Hx_2,\dots,Hx_n\}=\{x_1H,x_2H,\dots,x_nH\}$. 
\end{exercise}

\subsection{*Landau's Theorem}

In 1903, Landau demonstrated that there exists only a finite number of groups with finite conjugacy classes. The proof is entirely elementary and is based on the following lemma:

\begin{lemma}[Landau]
  \label{lem:Landau}
  \index{Landau's!lemma}
  For each $k \in \mathbb{N}$, the equation 
  \[
	\frac{1}{n_1}+\cdots+\frac{1}{n_k}=1
  \]
  has only finitely many solutions.
\end{lemma}

\begin{proof}
  Suppose $0 < n_1 \leq n_2 \leq \cdots \leq n_k$. Then $n_1 \leq k$.
  We prove by induction that 
  \[
    n_j \leq \frac{k+1-j}{1-\left(\frac{1}{n_1}+\cdots+\frac{1}{n_{j-1}}\right)}
  \]
  for all $j \in \{2,\dots,k\}$. Since for each $j \in \{2,\dots,k\}$, $n_j \leq n_2$, then $1 \leq \frac{1}{n_1}+\frac{k-1}{n_2}$ and hence $n_2 \leq \frac{k-1}{1-\frac{1}{n_1}}$. If we assume that the result holds for $j \geq 2$, say $n_p \geq n_j$ for all $p \geq j$, then
  \[
	1 \leq \sum_{i=1}^{j-1}\frac{1}{n_i}+\frac{k-j+1}{n_j},
  \]
  which implies the inequality we wanted to prove.
\end{proof}

\begin{theorem}[Landau]
\index{Landau's!theorem}
  Let $k\geq1$. There exists only a finite number of finite groups that have exactly $k$ conjugacy classes.
\end{theorem}

\begin{proof}
  Let $G$ be a group with $k$ conjugacy classes, say $C_1,\dots,C_k$, and let $1=g_1,\dots,g_k$ be representatives of these classes. When we decompose $G$ as $G=C_1\cup\cdots\cup C_k$, we have 
  \[
    |G|=|C_1|+\cdots+|C_k|=(G:C_G(g_1))+\cdots(G:C_G(g_k)).
  \]
  For each $j \in \{1,\dots,k\}$, let $n_j=|C_G(g_j)|$. Then 
  \[
	1=\frac{1}{n_1}+\cdots+\frac{1}{n_k}.
  \]
  As we saw in Lemma~\ref{lem:Landau}, this equation has only finitely many solutions. In particular, $n_k=|G|$ is bounded by a function of $k$.
\end{proof}

Landau's method allows tackling certain classification results. Let us 
see some examples.

\begin{example}
  Let $G$ be a finite group that has two conjugacy classes. Since $G\setminus\{1\}$ is a conjugacy class, $|G|-1$ divides $|G|$, and thus $|G|=2$.
\end{example}

\begin{example}
  Let $G$ be a finite non-abelian group with three conjugacy classes. The solutions of the equation $1/n_1+1/n_2+1/n_3=1$ with $n_1\leq n_2\leq n_3$ are $(3,3,3)$, $(2,3,6)$, and $(2,4,4)$. The only possibility is $(2,3,6)$. Then $G\simeq\Sym_3$.
\end{example}

Now let us see a bound that can be easily obtained from Landau's method.

\begin{theorem}[Neumann]
\index{Neumann's!theorem}
If $G$ is a finite group of order $n$ with $k$ conjugacy classes, then
\[
k\geq\frac{\log\log n}{\log 4}.
\]
\end{theorem}

\begin{proof}
We proceed as we did in the proof of Landau's theorem. Let $C_1,\dots,C_k$ be the conjugacy classes, and let $1=g_1,\dots,g_k$ be representatives of these classes. When we decompose $G$ as a disjoint union of conjugacy classes, we have 
\[
n=|G|=|C_1|+\cdots+|C_k|=(G:C_G(g_1))+\cdots(G:C_G(g_k)).
\]
For each $j \in \{1,\dots,k\}$, let $n_j=|C_G(g_j)|$. Then 
\[
	1=\frac{1}{n_1}+\cdots+\frac{1}{n_k}.
\]

We claim that 
\[
\max_{1\leq i\leq k}n_i\leq k^{2^{k-1}}.
\]
Without loss of generality, we can assume that $n_1\leq n_2\leq\cdots\leq n_k$. Then $n_1\leq k$, because otherwise, 
\[
\sum_{i=1}^k\frac{1}{n_i}<\sum_{i=1}^k\frac{1}{k}=1,
\]
a contradiction. Let $r\in\{1,\dots,k-1\}$. We write
\[
\sum_{i=r+1}^k\frac{1}{n_i}=1-\sum_{i=1}^r\frac{1}{n_i}=\frac{x}{n_1\cdots n_r}
\]
for some positive integer $x$. Then
\[
\frac{k-r}{n_{r+1}}\geq\frac{1}{n_1\cdots n_r}
\]
and hence $n_{r+1}\leq (k-r)n_1\cdots n_r<kn_1\cdots n_r$.

To complete the proof, we need to prove that 
\begin{equation}
    \label{eq:Neumann}
    n_r\leq k^{2^{r-1}}
\end{equation}
for all $r$. We proceed by induction on $r$. The case $r=1$ is trivial. Suppose then that the result is valid
for all $j\leq r$. By the inductive hypothesis, 
\[
n_{r+1}\leq kn_1\cdots n_r\leq k\prod_{j=1}^k2^{2^{j-1}}=k^{2^k}.\qedhere
\]
\end{proof}

%Landau's method opens the door to tackling certain classification results. Let's %see some examples.

\begin{problem}[Brauer]
\index{Brauer problem}
Find good bounds for the order $n$ of a group with a fixed number $k$ of conjugacy classes. It is expected that the bounds will be considerably better than those obtained from Landau's method.
\end{problem}

\subsection{*Burnside's cyclic numbers theorem}

We now mention some problems and results related to the number of isomorphism classes of finite groups of a given order. This classification problem is obviously almost as old as group theory itself. When taking the first steps in group theory, we encounter some easy-to-prove results:
\begin{itemize}
    \item There exists a unique finite group of prime order, and it is cyclic.
    \item There are two groups of order four, both abelian.
    \item There are two groups of order six, one of which is non-abelian.
    \item Groups of order $p^2$ are abelian.
\end{itemize}

With the Sylow theorems, we can go a bit further. It is easy to prove, for example, that there exists a unique group of order 15, and it is cyclic. The same result can be shown for other orders, such as 455 and 615.

A natural question arises. For which values of $n$ does a unique group (which will obviously be isomorphic to $C_n$) of order $n$ exist? The answer was given by Burnside.

\begin{definition}
\index{Cyclic number}
A number $n\in\mathbb{N}$ is called \textbf{cyclic} if $C_n$ is the only group (up to isomorphism) of order $n$.
\end{definition}

Some examples of cyclic numbers are: $2$, $3$, $15$, and $615=3\cdot 5\cdot 41$.

\begin{theorem}[Burnside]
\index{Burnside's cyclic number theorem}
    Let $n\in\mathbb{N}$. Then $n$ is cyclic if and only if $n$ and $\phi(n)$ are coprime.
\end{theorem}

\begin{proof}
    Suppose $n$ is cyclic. Without loss of generality, we can assume that $n$ is square-free (because otherwise, if $n=p^am$ with $m\in\mathbb{N}$, $p$ prime such that $\gcd(p,m)=1$, and $a\geq2$, the group $C_m\times C_p^a$ has order $n$ and is not cyclic). We then write
    \[
        n=p_1\cdots p_k
    \]
    with the $p_j$ distinct primes and $\phi(n)=(p_1-1)\cdots(p_k-1)$. If $\gcd(n,\phi(n))\ne1$, there exist distinct primes $p$ and $q$ such that $p$ divides $q-1$. The group $G= C_m\times (C_p\rtimes C_q)$ has order $n=pqm$ and is not cyclic.

    Suppose $\gcd(n,\phi(n))=1$ and $n$ is not cyclic. Let $G$ be a group of minimum order $n$ that is not cyclic.
    Without loss of generality, we can assume that $n$ is square-free: if $n=p^\alpha m$ with $p$ prime, $m$ coprime with $p$ and $\alpha\geq2$, then, as $\phi(n)=p^{\alpha-1}(p-1)\phi(m)$, $p$ divides 
    $\gcd(n,\phi(n))$. Then
    \[
        n=p_1\cdots p_k,
    \]
    with the $p_j$ distinct primes.

    \begin{claim}
    Every subgroup of $G$ and every quotient of $G$ is cyclic.
    \end{claim}

    If $m$ divides $n$, then $\gcd(m,\phi(m))=1$ since $n$ and $\phi(n)=(p_1-1)\cdots(p_k-1)$ are coprime. Therefore, every subgroup and every proper quotient is cyclic by the minimality of $n$.

    \begin{claim}
    $Z(G)=\{1\}$.
    \end{claim}

    For each $i\in\{1,\dots,k\}$, let $x_i\in G$ be an element of order $p_i$. If $G$ were abelian, then $G$ would be cyclic: $x_1\cdots x_k$ would be an element of order $n$. Hence $Z(G)\ne G$. Now, if $1<|Z(G)|<n$, then $G/Z(G)$ would be cyclic (since every quotient of $G$ is), and then $G$ would be abelian.

    \begin{claim}
    If $M$ is a maximal subgroup of $G$ and $x\in M\setminus\{1\}$, then $M=C_G(x)$. In particular, if $M$ and $N$ are distinct maximal subgroups, then $M\cap N=\{1\}$.
    \end{claim}

    Since $Z(G)\ne\{1\}$, $C_G(x)\ne G$. And since $M$ is cyclic, $M\subseteq C_G(x)$. Therefore, by maximality, $M=C_G(x)$. If $M$ and $N$ are two maximal subgroups and $x\in M\cap N\setminus\{1\}$, then $M=N=C_G(x)$.

    \begin{claim}
    If $M$ is a maximal subgroup, then $M=N_G(M)$.
    \end{claim}

    Let $x\in N_G(M)\setminus\{1\}$ and let $\alpha\in\Aut(M)$ be given by $y\mapsto xyx^{-1}$. Since $M$ is cyclic, if $m=|M|$, then $|\Aut(M)|$ has order $\phi(m)$. On the other hand, since $|x|$ divides $n$, $|\alpha|$ divides $n$. Hence $|\alpha|$ divides $\gcd(n,\phi(m))=1$. This means that $x\in C_G(M)$, i.e., $N_G(M)\subseteq C_G(M)$. Since $M\subseteq N_G(M)\subseteq C_G(M)$, $M=N_G(M)=C_G(M)$.

    Let $M_1,\dots,M_l$ be the representatives of the conjugacy classes of maximal subgroups of $G$. For each $j\in\{1,\dots,l\}$, let $m_j=|M_j|$. Since $M_{j}=N_G(M_j)$ for each $j$, the orbit of $M_j$ has $n/m_j$ elements.

    Since for each $g\in G\setminus\{1\}$ there exists a unique maximal subgroup $M$ such that $g\in M$, we have
    \begin{equation}
    \label{eq:particion}
    n=1+\sum_{j=1}^l \frac{n}{m_j}(m_j-1).
    \end{equation}
    If $l=1$ then $n=m_1$, a contradiction. If $l>1$ then, since for each $j$ we have $m_j\geq2$, rewriting~\eqref{eq:particion}, we have
    \begin{align*}
    \frac{1}{n}+l-1=\sum_{j=1}^l\frac{1}{m_j}\leq\frac{l}{2}.
    \end{align*}
    From this inequality, we obtain $nl\leq 2n-2<2n$ and then $l<2$, a contradiction. 
\end{proof}

Similarly, abelian and nilpotent numbers can be defined. These numbers are classified, and an elementary proof can be found in \cite{MR1786236}. There is also the notion of a solvable number. Thanks to the Feit--Thompson theorem, every odd number is a solvable number. These numbers are also classified, although the proof is much more difficult as it relies on a very deep theorem of Thompson and the famous Feit--Thompson theorem.

\subsection{*How many finite groups are there?}

In \cite{MR2410121}, the function $\operatorname{gnu}(n)$ is defined, which returns the number of isomorphism classes of groups of order $n$. For example, \[
\operatorname{gnu}(1)=\operatorname{gnu}(2)=\operatorname{gnu}(3)=\operatorname{gnu}(5)=1\text{ and }\operatorname{gnu}(4)=\operatorname{gnu}(6)=2.
\]
The name comes from \emph{\textbf{g}roups \textbf{nu}mber}.

Burnside's theorem can be reformulated as follows:
\[
\operatorname{gnu}(n)=1 \Longleftrightarrow \text{$n$ is cyclic} \Longleftrightarrow \gcd(n,\phi(n))=1.
\]

In \cite{MR2410121}, Conway, Dietrich, and O'Brien characterized the $n\in\mathbb{N}$ such that $\operatorname{gnu}(n)=2$, $\operatorname{gnu}(n)=3$, $\operatorname{gnu}(n)=4$.

In the Encyclopedia of Integer Sequences, the sequence
\[
\operatorname{gnu}(1),\operatorname{gnu}(2),\operatorname{gnu}(3),\operatorname{gnu}(4)\dots
\]
is \lstinline{A000001}, see \url{http://oeis.org/A000001} for more information.

There exists a powerful database of small groups. It was 
written by Besche, Eick and O'Brien and is now a fundamental tool in group theory \cite{MR1935567}. In particular, this database allows us to easily compute some values of the function $\operatorname{gnu}$.

The function \lstinline{NrSmallGroups} returns the number of isomorphism classes of groups of a certain order. We then define the function $\operatorname{gnu}$ and compute some examples:

\begin{lstlisting}
gap> gnu := NrSmallGroups;;
gap> gnu(16);
14
gap> gnu(32);
51
gap> gnu(64);
267
gap> gnu(27);
5
gap> gnu(81);
15
gap> gnu(128);
2328
gap> gnu(512);
10494213
\end{lstlisting}

It is known that $\operatorname{gnu}(1024)=49487365422$, although this value cannot be obtained directly with the computer, as there are way too many 
groups of order 1024. The groups of order 1024 were classified by Besche, Eick, and O'Brien, and the announcement was made in \cite{MR1826989}.

More than 99\% of the groups of order $<2000$ are of order $1024$. In fact, as we saw, there are 49487365422 groups of order 1024, and the number of isomorphism classes of groups of order $n\ne1024$ with $n<2016$ is 423164131.

\begin{lstlisting}
gap> Sum([1..1023], gnu);
423164131
\end{lstlisting}

The classification of groups is a difficult problem. A particularly difficult case is that of groups of order $p^3$, where $p$ is a prime number. For example, there are exactly 2 groups of order $2^3$ and 5 groups of order $3^3$. For general $p$, the question is open. For example, there are 14 groups of order $2^4$, 51 groups of order $2^5$, and 267 groups of order $2^6$. The sequences are found in the Encyclopedia of Integer Sequences, sequences \lstinline{A000679}, \lstinline{A000880}, and \lstinline{A000881}.

% Finally, we mention some references on the subject. The classic book on the classification of finite simple groups is \cite{MR1915964}. For groups of small order, see \cite{MR1250465} and \cite{MR1357169}.

\begin{lstlisting}[language=gap]
gap> Sum([1..1023], gnu)+Sum(List([1025..2015], gnu));
423164131
\end{lstlisting}

These numbers give us approximately 99.15\%.

These observations naturally suggest the following conjecture, which seems to be part of mathematical folklore:

\begin{conjecture}
Almost every finite group is a $2$-group.
\end{conjecture}

The numerology we did allows us to avoid having to make precise what \textquotedblleft almost every finite group\textquotedblright\ means. Similar problems appear in Chapter 22 of the book \cite{MR2382539}.

\begin{problem}
Calculate $\operatorname{gnu}(2048)$.
\end{problem}

It is known that $\operatorname{gnu}(2048)>1774274116992170$, which is the number of subgroups of order 2048 of a certain class.

\begin{conjecture}
\label{conjecture:gnu}x
Let $n\geq1$. The sequence
\[
\operatorname{gnu}(n),\operatorname{gnu}^2(n),\operatorname{gnu}^3(n)\dots
\]
stabilizes at 1.
\end{conjecture}

In Conjecture \ref{conjecture:gnu}, 
$\operatorname{gnu}^{1}(n)=\operatorname{gnu}(n)$ and 
$\operatorname{gnu}^{k+1}(n)=\operatorname{gnu}(\operatorname{gnu}^k(n))$ for $k\geq1$. 

It is not difficult to verify that the conjecture is true for $n<2000$.

The following conjecture appeared independently in several places. Apparently, the first more or less explicit appearance was around 1930 and was due to Miller. Independently, MacHale formulated it forty years later.

\begin{conjecture}
The function $\Z_{\geq1}\to\Z_{\geq1}$, 
$n\mapsto\operatorname{gnu}(n)$, is surjective.
\end{conjecture}

For more information about this conjecture, we refer to \cite[\S21.6]{MR2382539}.

Although there are many conjectures about the behavior of the function that counts the number of isomorphism classes of finite groups, there are several results. The following one is completely elementary:

\begin{theorem}
If $n\geq1$, then $\operatorname{gnu}(n)\leq n^{n\log_2 n}$.
\end{theorem}

\begin{proof}
If $G$ is a group, let
\[
d(G)=\min\{k:\text{there exist }g_1,\dots,g_k\in G\text{ such that }G=\langle g_1,\dots,g_k\rangle\}.
\]
We are going to prove that if $|G|=n$, then $d(G)\leq\log_2 n$. Let
\[
\{1\}=G_0\subsetneq G_1\subsetneq\cdots\subsetneq G_r=G
\]
be a maximal sequence of subgroups. For each $i\in\{1,\dots,r\}$ let
$g_i\in G_i\setminus G_{i-1}$. It is easily shown
that
\[
G_i=\langle g_1,\dots,g_i\rangle
\]
for all $i$. 
Indeed, if there exists some $i$ such that $G_i\ne\langle g_1,\dots,g_i\rangle$, then
there exists $g\in G_i\setminus\langle g_1,\dots,g_i\rangle$ and then
\[
\langle g_1,\dots,g_i\rangle\subsetneq \langle G_i,g\rangle\subsetneq G_{i+1}
\]
which contradicts the maximality of the sequence of subgroups. In particular, $G$ is generated 
by $r$ elements.

By Lagrange's theorem, 
\[
n=|G|=\prod_{i=1}^r(G_i:G_{i-1})\geq 2^r
\]
and then $r\leq\lfloor \log_2 n\rfloor$. 
As $G$ is isomorphic to a 
subgroup of $\Sym_n$ by Cayley's theorem, then 
\begin{align*}
\operatorname{gnu}(n)
&\leq\text{number of subgroups of order $n$ of $\Sym_n$}\\
&\leq\text{number of subgroups of $\Sym_n$ generated by $\lfloor\log_2n\rfloor$ elements}\\
&\leq\text{number of subsets of $\Sym_n$ of $\lfloor\log_2n\rfloor$ elements}.
\end{align*}
Since the number of subsets of $\Sym_n$ of $\lfloor\log_2n\rfloor$ elements
is
\[
\binom{n!}{\lfloor\log_2n\rfloor}\leq(n!)^{\log_2n}
\]
since $\binom{n}{k}\leq n^k$, we conclude that $\operatorname{gnu}(n)\leq n^{n\log_2n}$.
\end{proof}
% The proof is simple and elementary and can be found in the second
% chapter of the book \cite{MR2382539}.

In the case of $p$-groups, it can be shown by elementary methods 
that
\[
\operatorname{gnu}(p^n)\leq p^{\frac16(n^3-n)},
\]
see \cite[Theorem 5.1]{MR2382539}. Much more sophisticated bounds are known:

\begin{theorem}[Higman--Sims]
\index{Higman--Sims Theorem}
If $p$ is a prime number and $n\geq1$, then
\[
p^{\frac{2}{27}n^3-O(n^2)}\leq\operatorname{gnu}(p^n)\leq p^{\frac{2}{27}n^3+O(n^{\frac{8}{3}})}.
\]
\end{theorem}

The proof of the above theorem results from combining the works of Higman \cite{MR113948}
and Sims \cite{MR169921}. A modern presentation can be found in the book
\cite{MR2382539}.

In \cite{MR123605}, Higman conjectured that $\operatorname{gnu}(p^n)$ is a polynomial function of $p$ and $p$ modulo $N$ for a certain
finite number of integers $N$. 

\begin{conjecture}[Higman]
\index{Higman's PORC conjecture}
Let $n\geq1$. Then there exist $N$ polynomials
\[
P_{0}(X),P_{1}(X),\dots,P_{N-1}(X)
\]
such that
if $p\equiv i\bmod N$, then $\operatorname{gnu}(p^n)=P_{i}(p)$.
\end{conjecture}

PORC stands for \textbf{P}olynomial \textbf{O}n \textbf{R}esidue \textbf{C}lasses.

It is known that the conjecture is true for $n\leq7$, although the problem remains open for $n\geq8$.
In \cite{MR2921623}, a work of over seventy pages,
du Sautoy and Vaughan--Lee constructed a family of groups
of order $p^{10}$ suggesting that the PORC conjecture might not be true. Nevertheless, Higman's PORC conjecture remains open.

\begin{theorem}[Pyber]
\index{Pyber's Theorem}
If $n\geq1$, then
\[
\operatorname{gnu}(n)\leq n^{\frac{2}{27}\mu(n)^2+O\left(\mu(n)^{\frac{5}{3}}\right)},
\]
where $\mu(n)$ is the largest exponent
appearing in the prime factorization of $n$.
\end{theorem}

The proof appears in \cite{MR1200081} and in the case of non-solvable groups uses the classification of simple groups. A detailed presentation can be found in the book
\cite{MR2382539} by Blackburn, Neumann, and Venkataraman.
